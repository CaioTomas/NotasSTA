\documentclass[12pt,a4paper]{article}
\usepackage[utf8]{inputenc}
\usepackage{amssymb,amsthm,amsmath}
\usepackage{graphicx}
\usepackage{verbatim}
\usepackage{booktabs}
\usepackage{xfrac}
\usepackage[pdfstartview=FitH,backref,colorlinks,bookmarksnumbered,bookmarksopen,linktocpage,urlcolor=blue,linkcolor=cyan]{hyperref}
\usepackage{pdfpages}
\usepackage{enumitem}
\usepackage{pifont}

\newcommand{\cmark}{\ding{51}}%
\newcommand{\xmark}{\ding{55}}%
\newlist{todolist}{itemize}{2}
\setlist[todolist]{label=$\square$}

\newcommand{\done}{\rlap{$\square$}{\raisebox{2pt}{\large\hspace{1pt}\cmark}}%
\hspace{-2.5pt}}
\newcommand{\wontfix}{\rlap{$\square$}{\large\hspace{1pt}\xmark}}

\title{A fazer -- observações gerais}
\author{}
\date{\today}

\begin{document}

\maketitle

\begin{todolist}
    % pra marcar o item como feito é só colocar \item[\done]
    \item introduzir o livro como um texto sobre continuações/extensões analíticas (?)
    \item juntar as partes que tratam do logaritmo em um mesmo lugar ou deixar ``on-demand'',
    ao longo do texto (?)
    \item ir acrescentando os termos ao índice remissivo
    \item colocar enunciados mais auto-contidos (?)

\end{todolist}

\end{document}