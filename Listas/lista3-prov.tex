\chapter*{Lista 3}
\addcontentsline{toc}{chapter}{Lista 3}
\markboth{Lista 2}{Lista 3}
%%%%%%%%%%%%%%%%%%%%%%%%%%%%%%%%%%%%%%%%%%%%%%%%



% Inicio da Lista de Exercícios 
\begin{enumerate}[leftmargin=*]

	\item Calcule a $\int_{\partial V} f$, onde $V$ é cada um dos conjuntos do exercício anterior ($V$ e a $\partial V$ tem orientação compatível) e a função $f$ é dada por \\[0.2cm]
	a) $f(x,y)=\left(\displaystyle\frac{-y}{x^2+y^2},\frac{x}{x^2+y^2}\right)$;\\[0.3cm]
	b) $f(x,y)=\left(\displaystyle\frac{x}{x^2+y^2},\frac{-y}{x^2+y^2}\right)$;\\
	\item Calcule $\int_{\partial V}x^n\, dy$ e $\int_{\partial V}y^n\, dx$, onde $n\geq 1$ e $V$ é dado por \\
	a) o quadrado $[0,1]\times [0,1]$;\\
	b) o disco $\mathbb{D}\equiv \{z\in\mathbb{C}: |z|< 1\}$;\\ 
	levando em conta que $V$ e $\partial V$ estão compativelmente orientados.
	\item Seja $V$ como enunciado do Teorema de Green. Mostre que a área de $V$ é dada por 
	$$
	\int_{\partial V} x\, dy.
	$$
	\item Use o exercício anterior para calcular a área de\\
	a) $V=\{(x,y)\in\mathbb{R}^2: \frac{x^2}{a^2}+\frac{y^2}{b^2}\leq 1\}$;\\
	b) $V=\{(x,y)\in\mathbb{R}^2: 1\leq x^2-y^2\leq 9, \ 1\leq xy\leq 4\}$.
	\item Calcule 
	\\[0.2cm]
	a) $\displaystyle\int_{\partial V} (x^2-y^2)\, dx+ 2xy \, dy$\ ;\\[0.5cm]
	b) $\displaystyle \int_{\partial V} 2xy\, dx+ (y^2-x^2) \, dy$\ ,\\[0.5cm]
	onde $V$ é \\
	a) o retângulo delimitado pelas retas $y=x, y=-x+4, y=x+2, y=-x$;\\
	b) $\{(x,y)\in\mathbb{R}^2: 1\leq x^2-y^2\leq 9, \ 1\leq xy\leq 4\}$ ($V$ e $\partial V$ tem fronteira compatível). 
	\item Calcule 
	$$
	\int_{\partial V} \frac{y}{(x-1)^2+y^2}\ dx + \frac{x-1}{(1-x)^2+y^2}\ dy,
	$$
	onde $V$ é a região limitada pelos círculos $x^2+y^2=3$ e $(x-1)^2+y^2=9$ ($V$ e $\partial V$ tem fronteira compatível). 
	\item Calcule 
	$$
	\int_{\partial V} \frac{x^2y}{(x^2+y^2)^2}\ dx - \frac{x^3}{(x^2+y^2)^2}\ dy,
	$$
	onde $V$ é a região interior a elipse $\frac{x^2}{4}+\frac{y^2}{9}=1$ orientada no sentido anti-horário. 



	
	
	\item Em quais pontos do plano complexo cada uma das as seguintes funções  
	têm derivada (no sentido complexo)
	
	\begin{itemize}
		\item $f(z)= \overline{z}^2$;
		\item $f(z) = f(x + iy) = x^3-3xy^2+i(3x^2y -y^3)$;
		\item $f(z)= |z|$.
	\end{itemize}
	
	\item Em quais pontos do plano complexo a função 
	$$
	f(z)=f(x+iy) = \frac{x}{x^2+y^2}- i\frac{y}{x^2+y^2}
	$$
	tem derivada ? Para tais pontos calcule $f'(z)$.
	
	
	\item Seja $g:\mathbb{C}\setminus\{0\}\to \mathbb{C}$ a função definida por 
	$$
	g(z)=\frac{\overline{z}}{z}.
	$$
	Pergunta: existe o limite de $g(z)$ quando $z$ tende a zero ?
	
	\item Mostre {\bf usando as equações de Cauchy-Riemann} que a função 
	$f:\mathbb{C}\to\mathbb{C}$ dada por $f(z)=z^3+2z+1$ é inteira.

	\item Existe algum subconjunto do plano complexo onde $f(z)=|z|$ é analítica ?

	\item Seja $U\subset \mathbb{C}$ um domínio (aberto e conexo). Suponha que $f:U\to\mathbb{C}$ seja uma função 
	holomorfa. Defina $g:U\to\mathbb{C}$ por $g(z)=\overline{f(z)}$. A função $g$ é holomorfa no domínio $U$ ?
	
	
	\item Seja $f:\mathbb{C}\to\mathbb{C}$ uma função inteira {\bf não constante}. Mostre que as seguintes funções não 
	são inteiras:
	\begin{itemize}
		\item[a)] $|f(z)|$;
		\item[b)] $\overline{f(z)}$;
		\item[c)] $\text{Im}(f(z))$;
		\item[d)] $\Re(f(z))$.
	\end{itemize}
	
	
	
	\item Seja $\mathbb{H}\equiv\{z\in\mathbb{C}: \text{Im}(z)>0\}$ o semi-plano superior. Considere a função 
	\[ 
		f(z)=\frac{z-1}{z+1}.
	\] 
	Mostre que $\text{Im}(f(z))>0$ para todo $z\in\mathbb{H}$. Conclua que 
	$f$ define uma função que leva $\mathbb{H}$ em algum subconjunto de $\mathbb{H}$. Verifique se as equações 
	de Cauchy-Riemann são satisfeitas em $\mathbb{H}$, em caso afirmativo, determine a derivada de $f$ em cada ponto de
	seu domínio.
	
	
	
	\item Se $U\subset\mathbb{C}$ é um aberto do plano complexo é verdade que se $f'(z)=0$ para todo $z\in U$ então 
	$f(z)\equiv c$ para alguma constante $c\in\mathbb{C}$ ?
	
	
	
	
	
	\item A conclusão do exercício acima continua verdadeira se supormos apenas que a parte imaginária 
	de $f$ é constante ?
	
	
	
	
	\item Examine a função analítica $f(z) = z^2$.
	
	\begin{itemize}
		\item Determine as funções $u,v:\mathbb{R}^2\to\mathbb{R}$ tais que
		$z^2=u(x,y)+iv(x,y)$.
		
		\item 	Determine as seguintes {\it curvas equipotenciais} para $u$ e $v$, isto é,  
		as curvas $u(x,y)=1$ e $v(x,y)=1$.
		
		\item 	Determine os pontos de interseção das curvas, determinadas no exercício anterior, 
		no primeiro quadrante.
		
		\item  Mostre que nos pontos de interseção, estas curvas se tocam ortogonalmente,
		isto é, seus vetores tangentes neste ponto são ortogonais.
	\end{itemize}
	
	
	
\end{enumerate}
