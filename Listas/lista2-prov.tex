\chapter*{Lista 2}
\addcontentsline{toc}{chapter}{Lista 2}
\markboth{Lista 2}{Lista 2}
%%%%%%%%%%%%%%%%%%%%%%%%%%%%%%%%%%%%%%%%%%%%%%%%



% Inicio da Lista de Exercícios 
\begin{enumerate}[leftmargin=*]
	\item Neste exercício $z_0$ é um número complexo arbitrário fixado. 
	Esboce os conjuntos abaixo, diga se são fechados, abertos
	ou nenhum deles, esboce sua fronteira, diga quais são domínios e quais são limitados:
	\\
	a) $\Re(z)\geq \Re(z_0)$;\\
	b) $\Im(z_0)> \Re(z)$;\\
	c) $\Re(z^2)\geq 1$;\\
	d) $\Im(zz_0)>0$;\\
	e) $|z-z_0|<|\overline{z}-z_0|$;\\
	f) $|z-z_0|\leq |z-\overline{z_0}|$;\\
	g) $1\leq |z-\overline{z_0}|\leq 3$;\\
	h) $\Im(z^2)\leq 1$.
	\item Para cada um dos conjuntos abaixo sua fronteira é descrita 
	por uma curva suave por partes. Esboce o conjunto, sua fronteira e dê uma aplicação que a descreva\\
	a) $V=\{z\in\mathbb{C}: |z|\leq 1,\ \Re(z)\geq 1/2\}$;\\
	b) $V=\{z\in\mathbb{C}: 1/2\leq |z|\leq 1, \ \Re(z)\geq 0\}$;\\
	c) $V=\{z\in\mathbb{C}: 1/3\leq |z|\leq 1,\  \Re(z)\geq \Im(z)\geq 0\}$.
	
\end{enumerate}