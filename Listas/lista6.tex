% !TeX spellcheck = pt_BR
\chapter*{Lista 6}
\addcontentsline{toc}{chapter}{Lista 6}
\markboth{Lista 6}{Lista 6}
%%%%%%%%%%%%%%%%%%%%%%%%%%%%%%%%%%%%%%%%%%%%%%%%




% Inicio da Lista de Exercícios 
\begin{enumerate}[leftmargin=*]


	\item 
	Seja $\gamma:[a,b]\to\mathbb{R}^2$ um caminho suave inteiramente contido em 
	um aberto conexo $U\subset\mathbb{R}^2$. 
	Seja $F:U\subset\mathbb{R}^2\to\mathbb{R}^2$ um campo vetorial contínuo 
	dado por $F(x,y)=(u(x,y),v(x,y))$. Fazendo a identificação natural
	de $U$ com um subconjunto do plano complexo podemos observar que o campo vetorial $F$
	induz uma função complexa $f:U\to\mathbb{C}$ contínua. 
	Usando a definição escreva a integral de linha de $F$ e em seguida 
	a integral complexa de $f$. Explique quais são as diferenças entre
	estes conceitos.  
	
	
	\item A função $f:\mathbb{C}\to\mathbb{C}$ dada por $f(z)=z^2$
	induz um campo de vetores $F$ em $\mathbb{R}^2$. Descreva este campo 
	em coordenadas, em seguida calcule a integral de linha (real),
	e a integral complexa deste campo, ao longo do círculo unitário
	no sentido anti-horário.  

	\item Seja $U\subset \mathbb{C}$ um aberto conexo contendo a origem e 
	$h:U\to\mathbb{C}$ uma função contínua (não necessariamente holomorfa).
	Para cada $r>0$ fixado considere a curva suave 
	$\gamma_r:[0,2\pi]\to\mathbb{C}$ dada por $\gamma_r(t)=re^{it}$. 
	Mostre que 
	\[
	\lim_{r\to 0} \int_{\gamma_r} \frac{h(z)}{z}\, dz  =2\pi i h(0).
	\]
	(Dica: use a técnica apresentada na prova do Teorema de Cauchy-Goursat.)
	
	\item Sejam $R$ um retângulo contido em um domínio estrelado $\Omega\subset\mathbb{C}$ e  
	$f:\Omega\to\mathbb{C}$ uma função holomorfa. Mostre que
	\[
		\int_{\partial R} f(z)\, dz = 0.
	\]  


	
	\item O objetivo deste exercício é provar que a transformada de Fourier da
	função $f:\mathbb{R}\to\mathbb{R}$ dada por $f(x)=e^{-\pi x^2}$
	é ela mesma. 
	
	Primeiro lembramos que a transformada de Fourier desta função
	é definida para cada $\xi\in\mathbb{R}$ pela expressão
	\[
	\mathcal{F}(f)(\xi)
	=
	\lim_{R\to \infty} 
	\int_{-R}^{R} e^{ -\pi x^2  } e^{ 2\pi i x\xi }\, dx
	\equiv 
	\int_{-\infty}^{\infty} e^{ -\pi x^2  } e^{ 2\pi i x\xi }\, dx.
	\]
	Para calcular a transformada de Fourier acima vamos 
	precisar usar o seguinte fato bem-conhecido
	\[
	\lim_{R\to\infty} \int_{-R}^{R} e^{-\pi x^2}\, dx = 1
	\]
	e seguir os seguintes passos:
		\begin{itemize}
			\item Mostre que basta considerar $\xi\geqslant 0$, isto é, 
			$\mathcal{F}(f)(\xi)=\mathcal{F}(f)(-\xi)$, para todo $\xi\geqslant 0$.
			
			\item Para cada $R,\xi>0$, considere o contorno $\gamma_{R}$ consistindo do 
			retângulo no plano complexo cujos os vértices são os pontos 
			$R, R+i\xi, -R+i\xi,-R$. Faça o esboço deste contorno;
			
			\item Defina a função $g(z)=e^{-\pi z^2}$ e mostre que é possível 
			usar o exercício anterior para calcular a integral 
			$\int_{\gamma_{R}} g(z)\, dz$ para cada $R>0$;
			
			\item Sejam $I_1(R)$ e $I_2(R)$ as 
			integrais da função $g$ ao longo dos segmentos de reta
			unindo os pontos $R$ à $R+i\xi$ e $-R$ à $-R+i\xi$, respectivamente.
			Mostre que existem constantes $C_1,C_2>0$ tais que 
			$|I_1(R)|\leqslant C_1 e^{-\pi R^2}$
			e que $|I_2(R)|\leqslant C_2 e^{-\pi R^2}$.
			
			\item Conclua que 
			\[\mathcal{F}(f)(\xi) = f(\xi). \]
		\end{itemize}
	
	
	\item Calcular $\int_{\gamma} f(z)\, dz$, onde
		\begin{itemize}
			\item[a)] $f(z)=z\overline{z}$ e $\gamma(t)=e^{it}, 0\leq t\leq 2\pi$.
			\item[b)] $\displaystyle f(z)=\frac{z+1}{z}$ e $\gamma(t)= 3e^{it}, 0\leq t\leq 2\pi$.
			\item[c)] $\displaystyle f(z)=\frac{z+1}{z}$ e 
						$\gamma(t) = \frac{1}{4}e^{it}, 0\leq t\leq 2\pi$.
			\item[d)] $\displaystyle f(z)=\frac{z+1}{z}$ e 
						$\gamma(t) = 5i+e^{it}, 0\leq t\leq 2\pi$.			
			\item[e)] $\displaystyle f(z)=\frac{1}{z^2-2}$ e $\gamma(t) = 2+e^{it}, 0\leq t\leq 2\pi$. 
			\item[f)] $\displaystyle f(z)=\frac{1}{z^2-2}$ e $\gamma(t) = 2e^{it}, 0\leq t\leq 2\pi$. 
			\item[g)] $\displaystyle f(z)= \pi e^{\pi\overline{z}}$ e $\gamma$ é o quadrado
						de vértices $0,1,1+i$ e $i$, orientado no sentido anti-horário.
		\end{itemize}
	


\end{enumerate}
