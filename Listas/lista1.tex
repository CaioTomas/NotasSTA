\chapter*{Lista 1}
\addcontentsline{toc}{chapter}{Lista 1}
\markboth{Lista 1}{Lista 1}
%%%%%%%%%%%%%%%%%%%%%%%%%%%%%%%%%%%%%%%%%%%%%%%%




% Inicio da Lista de Exercícios 
\begin{enumerate}[leftmargin=*]

	\item Reduza à forma $x+iy$
	\begin{itemize}
		\item[a)] $(1-5i)^2-4i$;
		\item[b)] $-i(-1+i)+2$;
		\item[c)] $(3+i)(1-11i)$;
		\item[d)] $(\sqrt{2}-i\sqrt{5})(\sqrt{5}+i\sqrt{2})$;
	\end{itemize}
	\item Novamente reduza à forma $x+iy$ 
	\begin{itemize}
		\item[a)] $\displaystyle\frac{3-4i}{2i}$;\\
		\item[b)] $\displaystyle\frac{2+5i}{-1+i\sqrt{3}}$;\\
		\item[c)] $\displaystyle\frac{(2-2i)^2}{1+i}$;\\
		\item[d)] $\displaystyle\frac{z-\overline{z}i}{\overline{z}-zi}$;\\[-0.3cm]
	\end{itemize}
	\item Faça o esboço e identifique os seguintes conjuntos 
	\begin{itemize}
		\item[a)] $|z|=|z-2|$;
		\item[b)] $|z|=|\overline{z}-1|$;
		\item[c)] $a|z|=|z-1|$, com $a\in\mathbb{R}\setminus\{0,1\}$;
		\item[d)] $\Re(z)=\Im(z-1)$;
		\item[e)] $\Im(z-1)=|z+1|$;
		\item[f)] $|\overline{z}|=|z-1|$;
	\end{itemize}
	\item Usando a primeira desigualdade triangular, mostre que $||z_1|-|z_2||\leq |z_1-z_2|$.
	\item Deduza a desigualdade $|z_1+z_2+z_3|\leq |z_1|+|z_2|+|z_3|$.
	\item Mostre que, se $z_2\neq -z_3$ então $\displaystyle\left|\frac{z_1}{z_2+z_3}\right|\leq \frac{|z_1|}{||z_2|-|z_3||}$.\\
	\item Resolva as equações\\ a) $z-\overline{z}=1$;\\ b) $z+\overline{z}i=2+i$;\\ c) $z+2\overline{z}=1-i$.
	\item Mostre que $|z_1+z_2|< |1+\overline{z_1}z_2|$, desde que $|z_1|< 1$ e $|z_2|<1$.
	\item Encontre todas as soluções das equações\\ a) $z^2=1-i\sqrt{3}$;\\ b) $z^5=-1$;\\ c) $\overline{z}^3=1$;\\ d) $z^7=-(1+i)$.
	\item Seja $P(x)=ax^2+bx+c$ um polinômio de grau 2 com coeficientes reais e suponha que $\Delta=b^2-4ac<0$. Então 
	as soluções da equação $P(x)=0$ são números complexos com parte imaginária não nula. Se $z_1$ e $z_2$ são estas soluções,
	mostre que $z_1=\overline{z_2}$. Mais geralmente, se $P(x)=a_nx^n+a_{n-1}x^{n-1}+\ldots+a_1x+a_0$ é um polinômio de 
	grau $n>0$ arbitrário, com coeficientes reais, e se $z_0\in \mathbb{C}$ é tal que $P(z_0)=0$, então tem-se que 
	$P(\overline{z_0})=0$. 
	\item Considere a equação $az^2+bz+c=0$, onde $a,b$ e $c\in \mathbb{C}$. Deduza uma expressão para suas raízes.
	\item Deduza a fórmula $1+z+z^2+\ldots+z^{n-1}=\displaystyle\frac{1-z^n}{1-z}$, se $z\neq 1$.
	\item Use o exercício anterior para mostrar que se $\omega\neq 1$, satisfaz a equação $\omega^n=1$, então 
	$1+\omega+\omega^2+\ldots+ \omega^{n-1}=0$.
	\item Demonstre a fórmula De Moivre 
	$$
	(\cos\theta+i\,\text{sen}\, \theta)^n=\cos n\theta+i\,\text{sen}\, n\theta.
	$$
	\item Usa o exercício anterior para mostrar que \\
	a) $\cos 2\theta =\cos^2\theta-\sen^2\theta$;\\
	b) $\sen 2\theta =2\,\sen\theta\cos\theta$;\\
	c) $\cos 3\theta =\cos^3\theta-3\cos\theta\, \sen^2\theta$;\\
	d) $\sen 3\theta =3\cos^2\theta\,\sen\theta-\sen^3\theta$.
	\item Use o exercício anterior para deduzir expressões para 
	$\sen 4\theta$ e $\cos 4\theta$.
	\item Calcule $(2+i)(3+i)$ e deduza a igualdade 
	$$
	\frac{\pi}{4}=\arctan\left( \frac{1}{2}\right)+\arctan\left( \frac{1}{3}\right).
	$$
	\item Calcule $(5-i)^4(1+i)$ e deduza a igualdade 
	$$
	\frac{\pi}{4}=4\arctan\left( \frac{1}{5}\right)-\arctan\left( \frac{1}{239}\right).
	$$
	\item Mostre que três pontos $a,b$ e $c$ no plano complexo são colineares, se e somente se, 
	$$
	\det\begin{bmatrix}
	    	1\ &a\ &\overline{a}\ \\
	    	1\ &b\ &\overline{b}\ \\
	    	1\ &c\ &\overline{c}\ 
	    \end{bmatrix}=0.
	$$


	\item Uma maneira de definir uma ordem ``$<$'' em um corpo $\mathscr{C}$ 
	consiste um dar um subconjunto $\mathscr{C}^+$ de 
	$\mathscr{C}$ tal que: 
	\begin{itemize}
		\item[i)] se $x,y\in\mathscr{C}^+$ então $x+y\in\mathscr{C}^+$ e $xy\in\mathscr{C}^+$ 
		(o conjunto $\mathscr{C}^+$ é chamado de conjunto dos números positivos);
		\item[ii)] dado $x\in\mathscr{C}$ então apenas uma das possibilidades se verifica: 
		ou $x\in\mathscr{C}^+$, ou $x=0$ ou $-x\in\,\mathscr{C}^+$.  
	\end{itemize}
	Em seguida dizemos que $x<y$ se e somente se $y-x\in \mathscr{C}^{+}$.
	Note que segue das propriedades i) e ii) que o quadrado de 
	qualquer elemento não-nulo de $\mathscr{C}$ é um elemento do subconjunto dos positivos.
	De fato, se $x\in\mathscr{C}^+$ então $x^2\in\mathscr{C}^+$ por i). Por outro lado, se $-x\in\mathscr{C}$ então 
	 $(-x)^2=(-x)(-x)=x^2\in \mathscr{C}^+$, por i) novamente. Conclua que o corpo $\mathbb{C}$ 
	 não pode admitir nenhuma ordem do tipo definido acima.
	
	
	

\end{enumerate}
%  Fim da Lista de Exercícios



