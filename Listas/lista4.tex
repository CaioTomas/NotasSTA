% !TeX spellcheck = pt_BR
\chapter*{Lista 4}
\addcontentsline{toc}{chapter}{Lista 4}
\markboth{Lista 4}{Lista 4}
%%%%%%%%%%%%%%%%%%%%%%%%%%%%%%%%%%%%%%%%%%%%%%%%



% Inicio da Lista de Exercícios 
\begin{enumerate}[leftmargin=*]


\item Represente no plano cartesiano o número complexo 
$\displaystyle\exp\left( \frac{1+i}{1-i} \right)$.

\item Encontre o domínio onde a seguinte função está bem definida:
$$
\exp\left(  \frac{1}{z^n+1}  \right).
$$
\item Calcular a derivada da função do exercício anterior.
	



\item Mostre que a função $f:\mathbb{C}\setminus\{0\}\to\mathbb{C}$, dada por 
$$
f(z)=\displaystyle e^{\frac{1}{z}},
$$
não tem limite quando $z\to 0$.


	\item Nos itens abaixo determine quais limites existem e neste caso seu valor
\begin{itemize}
	\item[a)] $\displaystyle\lim_{z\to 0}\frac{e^z-1}{z}$;\\[0.2cm]
	\item[b)] $\displaystyle\lim_{z\to 0}\frac{\sen |z|}{z}$;\\[0.3cm]
	\item[c)] $\displaystyle\lim_{z\to 1}\frac{\overline{z}-1}{z-1}$;\\[0.3cm]
	\item[d)] $\displaystyle\lim_{z\to -1}\log(z)$, onde $\log(z)$ denota o ramo principal do logaritmo.
\end{itemize}
	\item É verdade que $|\sen z|\leq 1$, para todo $z\in\mathbb{C}$ ? Por quê ?
	
	
	\item Seja $f:\mathbb{C}\to\mathbb{C}$ uma função tal que 
	$f(z)\in \mathbb{C}^{*}\setminus L_{\pi}$ para todo $z\in \mathbb{C}$.
	Para cada $z\in\mathbb{C}$ fixado, mostre que
	\[
		\sqrt{-f(z)}
		=
		\begin{cases}
			-i \sqrt{f(z)},&\ \text{se}\ \Im(f(z))>0;
			\\[0.2cm]
			\, i \sqrt{f(z)},&\ \text{se}\ \Im(f(z))<0,
		\end{cases}
	\]
	onde a aplicação $(\mathbb{C}^{*}\setminus L_{\pi})\ni z\longmapsto \sqrt{z}\in\mathbb{C}$
	denota o ramo principal da raíz quadrada de $z$.

	\item Determine um aberto não vazio $U\subset \mathbb{C}$ tal que a expressão 
	      $$
	      f(z)=\frac{1}{i}\log(z+\sqrt{z^2-1}),
	      $$
	      onde $\log$ denota o ramo principal do logaritmo, esteja bem definida em $U$. Em seguida, 
	      mostre que $\cos(f(z))=z$ para todo $z\in U$. Faça o mesmo para a expressão 
	      $$
	      g(z)=\frac{1}{i}\log(z-\sqrt{z^2-1}),
	      $$

	\item Suponha que $f:U\to\mathbb{C}$ seja uma função holomorfa tal que $\sen(f(z))=z$, para 
	todo $z\in U$. Usando a regra da cadeia prove que $(f'(z))^2=1/(1-z^2)$. Mostre também que $\pi/2\notin f(U)$.

	\item Mostre que para quaisquer $a,b,c\in\mathbb{C}$ a equação $az^2+bz+c=0$ admite duas soluções (contadas com multiplicidade) 
	se $a\neq 0$.

	\item Considerando todos os ramos possíveis do logaritmo mostre que existem apenas 5 valores possíveis para $3^{\frac{1}{5}}$.  

	\item Considerando todos os ramos possíveis do logaritmo mostre que existem infinitos valores possíveis para $3^{\sqrt{2}}$.

	\item Encontre um ramo para a função $f(z)=\sqrt{1+z}$, em seguida para $g(z)=\sqrt{1-z}$ e por último
	um ramos para $h(z)=\sqrt{1-z^2}$. É verdade que $h(z)=f(z)g(z)$, em qualquer ponto onde todas estas funções estão definidas ?

	\item Encontre um ramo para $\log(\log(z))$.

	\item Calcule o módulo de $z^{\frac{1}{2}}$ e mostre que se $|z|\leq 1$ então $|z^{\frac{1}{2}}|\leq 1$.

	\item Encontre todas as soluções do problema $\sqrt{z+1}=5$, onde $\sqrt{\cdot}$ é um ramo arbitrário da raiz quadrada.

	\item É verdade que para quaisquer números complexos $a,b\in \mathbb{C}$ temos $|a^b|=|a|^{|b|}$ ? 

	\item Considere o polinômio $f(z)=z^n+a_{n-1}z^{n-1}+\ldots+a_1z+a_0$. Mostre que se $|z|$ é suficientemente grande então 
	$|f(z)-z^n|\leq \frac{1}{2}|z|^n$.

	\item Quais das seguintes desigualdades são verdadeiras para quaisquer números complexos:
	$$
	|e^z|\leq |z|, \ |z|\leq |e^z|,\ e^{|z|}\leq |z|,\ |e^z|\leq e^{|z|}\quad ?
	$$

	\item Calcule $\displaystyle\lim_{z\to 0}\frac{z}{\sen z}$.
	
	
	\item Prove a Proposição \ref{prop-ramos-raizes-raizes-nesimas}.
	Dica: re-leia o argumento do Exemplo \ref{exemplo-raiz-menos-um} e use a fórmula
	para as raízes de um número complexo $w$ dada em \ref{eq-sol-ZN=w}. 
	
\end{enumerate}