\chapter*{Lista 2}
\addcontentsline{toc}{chapter}{Lista 2}
\markboth{Lista 2}{Lista 2}
%%%%%%%%%%%%%%%%%%%%%%%%%%%%%%%%%%%%%%%%%%%%%%%%



% Inicio da Lista de Exercícios 
\begin{enumerate}[leftmargin=*]
	\item Neste exercício $z_0$ é um número complexo arbitrário fixado. 
	Esboce os conjuntos abaixo, diga se são fechados, abertos
	ou nenhum deles, esboce sua fronteira, diga quais são domínios e quais são limitados:
	\\
	a) $\Re(z)\geq \Re(z_0)$;\\
	b) $\Im(z_0)> \Re(z)$;\\
	c) $\Re(z^2)\geq 1$;\\
	d) $\Im(zz_0)>0$;\\
	e) $|z-z_0|<|\overline{z}-z_0|$;\\
	f) $|z-z_0|\leq |z-\overline{z_0}|$;\\
	g) $1\leq |z-\overline{z_0}|\leq 3$;\\
	h) $\Im(z^2)\leq 1$.
	\item Para cada um dos conjuntos abaixo sua fronteira é descrita 
	por uma curva suave por partes. Esboce o conjunto, sua fronteira e dê uma aplicação que a descreva\\
	a) $V=\{z\in\mathbb{C}: |z|\leq 1,\ \Re(z)\geq 1/2\}$;\\
	b) $V=\{z\in\mathbb{C}: 1/2\leq |z|\leq 1, \ \Re(z)\geq 0\}$;\\
	c) $V=\{z\in\mathbb{C}: 1/3\leq |z|\leq 1,\  \Re(z)\geq \Im(z)\geq 0\}$.

	\item Calcule a $\int_{\partial V} f$, onde $V$ é cada um dos conjuntos do exercício anterior ($V$ e a $\partial V$ tem orientação compatível) 
	e a função $f$ é dada por \\[0.2cm]
	a) $f(x,y)=\left(\displaystyle\frac{-y}{x^2+y^2},\frac{x}{x^2+y^2}\right)$;\\[0.3cm]
	b) $f(x,y)=\left(\displaystyle\frac{x}{x^2+y^2},\frac{-y}{x^2+y^2}\right)$;\\
	\item Calcule $\int_{\partial V}x^n\, dy$ e $\int_{\partial V}y^n\, dx$, onde $n\geq 1$ e $V$ é dado por \\
	a) o quadrado $[0,1]\times [0,1]$;\\
	b) o disco $\mathbb{D}\equiv \{z\in\mathbb{C}: |z|< 1\}$;\\ 
	levando em conta que $V$ e $\partial V$ estão compativelmente orientados.
	\item Seja $V$ como enunciado do Teorema de Green. Mostre que a área de $V$ é dada por 
	$$
	\int_{\partial V} x\, dy.
	$$
	\item Use o exercício anterior para calcular a área de\\
	a) $V=\{(x,y)\in\mathbb{R}^2: \frac{x^2}{a^2}+\frac{y^2}{b^2}\leq 1\}$;\\
	b) $V=\{(x,y)\in\mathbb{R}^2: 1\leq x^2-y^2\leq 9, \ 1\leq xy\leq 4\}$.
	\item Calcule 
	\\[0.2cm]
	a) $\displaystyle\int_{\partial V} (x^2-y^2)\, dx+ 2xy \, dy$\ ;\\[0.5cm]
	b) $\displaystyle \int_{\partial V} 2xy\, dx+ (y^2-x^2) \, dy$\ ,\\[0.5cm]
	onde $V$ é \\
	a) o retângulo delimitado pelas retas $y=x, y=-x+4, y=x+2, y=-x$;\\
	b) $\{(x,y)\in\mathbb{R}^2: 1\leq x^2-y^2\leq 9, \ 1\leq xy\leq 4\}$ ($V$ e $\partial V$ tem fronteira compatível). 
	\item Calcule 
	$$
	\int_{\partial V} \frac{y}{(x-1)^2+y^2}\ dx + \frac{x-1}{(1-x)^2+y^2}\ dy,
	$$
	onde $V$ é a região limitada pelos círculos $x^2+y^2=3$ e $(x-1)^2+y^2=9$ ($V$ e $\partial V$ tem fronteira compatível). 
	\item Calcule 
	$$
	\int_{\partial V} \frac{x^2y}{(x^2+y^2)^2}\ dx - \frac{x^3}{(x^2+y^2)^2}\ dy,
	$$
	onde $V$ é a região interior a elipse $\frac{x^2}{4}+\frac{y^2}{9}=1$ orientada no sentido anti-horário. 


\end{enumerate}