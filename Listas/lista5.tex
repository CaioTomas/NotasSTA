\chapter*{Lista 5}
\addcontentsline{toc}{chapter}{Lista 5}
\markboth{Lista 5}{Lista 5}
%%%%%%%%%%%%%%%%%%%%%%%%%%%%%%%%%%%%%%%%%%%%%%%%



% Inicio da Lista de Exercícios 
\begin{enumerate}[leftmargin=*]


	\item Calcule $\displaystyle \lim_{n\to\infty} \frac{n!}{n^n}$.

	\item Mostre que, se $|\alpha|<1$ então $\displaystyle\lim_{n\to\infty}n\alpha^n=0$.

	\item Calcule $\displaystyle \lim_{n\to\infty}\left[i+\left(\frac{2+3i}{5}\right)^n \right]$.

	\item Existe o seguinte limite: $\displaystyle \lim_{n\to\infty}\left[i+\left(\frac{2+3i}{5}\right)^n \right]$ ?



	\item Calcule
		\begin{eqnarray*}
			\lim_{n\to\infty} \frac{\log n}{n},
			&\qquad\displaystyle\lim_{n\to\infty} \sqrt[n]{n},	
			&\qquad\lim_{n\to\infty}\frac{\log(ni)}{ni}
			%
			\\[0.3cm]
			%
			\lim_{n\to\infty} \sqrt[n]{ni},
			&\qquad\displaystyle\lim_{n\to\infty} \sqrt[ni]{ni},		
			&\qquad\lim_{n\to\infty} \sqrt[ni]{n}.
		\end{eqnarray*}
		
	\item Para cada $\alpha\in\mathbb{C}$ diga se cada uma das sequências converge ou diverge e, 
			se convergir, determine o limite:
			$$
			\begin{array}{llll}
				\alpha^n,
				&\qquad\displaystyle n\alpha^n,	
				&\qquad\displaystyle \frac{\alpha^n}{n}
				&\qquad\displaystyle \sqrt{n}\left(\sqrt{n+\alpha}-\sqrt{n}\right)
			\end{array}	
			$$
 
 	\item Suponha que $|\alpha|<|\beta|<1$. Existe o limite $\displaystyle \sqrt[n]{\alpha^n+\beta^n}$ ?
 
 	\item Suponha que $1<|\alpha|=|\beta|$. Mostre que, se a sequência $\alpha^n-\beta^n$ 
 		é limitada, então $\alpha=\beta$.

 	\item Existem os seguintes limites: 
 			$$
			\begin{array}{lll}
				\displaystyle \lim_{n\to\infty} \cos(n\pi i)
				&\qquad\text{e}
				&\qquad\displaystyle \lim_{n\to\infty} ni\,\text{sen}\left(\frac{\pi i}{n}\right) ?	
			\end{array}	
			$$

	\item Mostre que as séries abaixo divergem
			$$
			\begin{array}{lll}
				\displaystyle \sum_{n=1}^{\infty}\frac{1}{ni}
				&\qquad\text{e}
				&\qquad\displaystyle \sum_{n=1}^{\infty}\frac{1}{n+i}.
			\end{array}	
			$$

	\item Determine o raio de convergência de cada uma das séries abaixo:
	 \begin{equation*}
		\begin{array}{llll}
			&\displaystyle\sum_{n=0}^{\infty}\frac{n}{(3i)^n}(z-1)^n,
			&\qquad \displaystyle \sum_{n=0}^{\infty}\frac{11^{n+2i}}{n!} z^n,
			&\qquad \displaystyle \sum_{n=0}^{\infty}\frac{n^{2i}}{2^n} (z-\pi)^n,
			%
			\\[1.0cm]
			%
			&\displaystyle\sum_{n=0}^{\infty} \frac{5}{(4+3i)^n} z^n,
			&\qquad \displaystyle \sum_{n=0}^{\infty} \frac{7n}{(5+i)^n} (z+2)^n,
			&\qquad \displaystyle \sum_{n=1}^{\infty} \frac{1}{\log (ni)} z^n,
			%
			\\[1.0cm]
			%
			&\displaystyle\sum_{n=0}^{\infty}\frac{i^n}{2^{ni}}z^n,
			&\qquad \displaystyle \sum_{n=1}^{\infty} \frac{(-1)^{n+1}}{n\sqrt{3i}} z^n,
			&\qquad \displaystyle \sum_{n=0}^{\infty} \frac{3^{ni}}{i(2n)!} z^n,
			%
			\\[1.0cm]
			%
			&\displaystyle\sum_{n=0}^{\infty}\frac{(n!)^2}{(2n)!}z^n,
			&\qquad \displaystyle \sum_{n=1}^{\infty} \left(1+\frac{1}{n}\right)^n z^n,
			&\qquad \displaystyle \sum_{n=1}^{\infty} \left(1+\frac{1}{n}\right)^{n^2} z^n,
			%
			\\[1.0cm]
			%
			&\displaystyle\sum_{n=1}^{\infty} \left(1-\frac{1}{n}\right)^n z^n,
			&\qquad \displaystyle \sum_{n=1}^{\infty} n^{\log n} z^n,
			&\qquad \displaystyle \sum_{n=1}^{\infty} \frac{1}{(\log n)^n} z^n,
		\end{array}
	 \end{equation*}	

	\item Mostre usando o produto de séries que se $f(z)=\displaystyle\sum_{n=0}^{\infty}a_n z^n$
	então
	$$
	\frac{1}{1-z}f(z) = \sum_{n=0}^{\infty}(a_0+\ldots+a_n)z^n.
	$$
	
	\item Mostre que 
		\begin{itemize}
		\item $\displaystyle \sum_{n=0}^{\infty} (n+1)z^n =\frac{1}{(1-z)^2}$.
		\item $\displaystyle \sum_{n=1}^{\infty} nz^n =\frac{z}{(1-z)^2}$.
		\item $\displaystyle \sum_{n=1}^{\infty} n^2z^n =\frac{z+z^2}{(1-z)^2}$.
		\item $\displaystyle \sum_{n=1}^{\infty} n^3z^n =\frac{z+4z^2+z^3}{(1-z)^4}$.
		\item $\displaystyle \sum_{n=1}^{\infty} n^4z^n =\frac{z+11z^2+11z^3+z^4}{(1-z)^5}$.
		\end{itemize}
	
	\item Seja $P(t)=a_0+a_1t+\ldots+a_kt^k$ um polinômio e considere a sequência 
			$\{P(n)\}_{n\in\mathbb{N}}$. Então 
			$$
				f(z) = \sum_{n=1}^{\infty} P(n)z^n
			$$
		é uma função racional. \\
		{\bf Dica:} comece considerando monômios, isto é, séries da forma $\sum_{n=0}^{\infty}n^kz^n$.
		Para estes casos inspiri-se no exercício anterior.
		
	
	\item Considere uma série de potências $\sum_{n=0}^{\infty}a_nz^n$ 
		na qual os coeficientes se repetem ciclicamente, $a_{n+k}=a_{n}$,
		onde $n$ é qualquer e $k$ um inteiro positivo fixado. Calcule
		seu raio de convergência e sua soma.


	\item Seja 
	$$
		f(z) = 1+\sum_{n=1}^{\infty} \frac{\alpha(\alpha-1)\ldots(\alpha-n+1)}{n!}\ \ z^n,
		\qquad\text{onde}\ \alpha\in\mathbb{R}. 
	$$
	Mostre que $f$ converge para todo $|z|<1$ e que $f'(z)=\frac{\alpha f(z)}{1+z}$. 
	Conclua daí que $[(1+z)^{\alpha}f(z)]'=0$ e que $f(z)=(1+z)^{\alpha}$. 
	Este resultado pode ser generalizado para $\alpha\in\mathbb{C}$ ?
		


	\item Use os resultados do apêndice sobre o raio de convergência, do livro texto, 
		para calcular o raio de convergência das seguintes séries:
		\begin{itemize}
			\item $\displaystyle \sum_{n=0}^{\infty}\frac{(n!)^2}{(2n)!}\, z^{2n}$.
			\item $\displaystyle \sum_{n=1}^{\infty}\left( 1+ \frac{1}{n}\right)^{n} z^{n^2}$.
			\item $\displaystyle \sum_{n=1}^{\infty}\left( 1+ \frac{1}{n}\right)^{n^2} z^{n^2}$.
		\end{itemize}		 

	
\end{enumerate}