% !TeX spellcheck = pt_BR
\chapter[Semana 3]{}
\chaptermark{}



\hfill%
\begin{minipage}{12cm}
	\begin{flushright}
		\rightskip=0.5cm
		\textit{``At the basis of the distance concept lies, for example, 
		the concept of a convergent point sequence and their defined limits,
		and one can, choosing these ideas as those fundamental to the point 
		set theory, eliminate the notions of distance ... Thirdly, 
		we can associate with each point of the set certain parts of the space
		called neighborhoods, and these can again be made building stones of the 
		theory with the elimination of the distance concept. Here the view
		of a set is in consideration of the association between elements and
		subsets.''}
		\\[0.1cm]
		\rightskip=0.5cm
		---F. Hausdorff, 1949
	\end{flushright}
\end{minipage}




\section{Limites, Continuidade de Funções Complexas}

Uma função complexa é uma função $f:U\subset \mathbb{C}\to\mathbb{C}$ definida em algum 
subconjunto $U\subset \mathbb{C}$ e tomando valores em $\mathbb{C}$
que associa cada ponto $z\in U$ um número complexo $f(z)$.
Podemos também descrever tais funções usando coordenadas, isto é,
para $z=x+iy\in U$ temos 
$f(z)=f(x+iy)=u(x,y)+iv(x,y)$, onde $u(x,y)=\Re(f(x+iy))=$ e $v(x,y)=\Im(f(x+iy))$.


Por exemplo, a função $f:\mathbb{C}\to\mathbb{C}$ dada por $f(z)=z^2$ pode ser
descrita em coordenadas por 
$f(x+iy) = (x+iy)^2 = x^2-y^2+2ixy$.
Em geral, é prudente evitar trabalhar com coordenadas. 
Mas isto não quer dizer que não devemos usá-las, mas sim 
avaliar cuidadosamente se uso delas facilitará ou viabilizará o estudo 
do problema. Um dos resultados mais importantes da próxima seção
às chamadas Condições de Cauchy-Riemann, fornecem um exemplo 
importante de como a descrição de uma função complexa em termos de suas coordenadas
pode ser poderosa. 

\bigskip 

A noção de limite associado a uma função $f:A\subset \mathbb{C}\to\mathbb{C}$ 
consiste simplesmente numa tradução da Definição \ref{definicao-limite-R2}.
Por questão de simplicidade, como na seção anterior, vamos considera apenas 
funções definidas em algum conjunto $A\subset\mathbb{C}$
cujo o interior é não-vazio. 

\begin{definicao}\label{def-limite-func-complexa}
\index{Limite!de uma função complexa}
Sejam $A\subset \mathbb{C}$ um conjunto cujo interior é não-vazio, $f:A\to\mathbb{C}$ uma 
função e $z_0\in \overline{A}$. Dizemos que um número complexo $w_0$ 
é o limite de $f(z)$ quando $z$ tende a $z_0$ se, dado qualquer $\varepsilon>0$ é possível
encontrar $\delta>0$ tal que se $z\in A$ e  $0<|z-z_0|<\delta$ então  temos $|f(z)-w_0|<\varepsilon$.
O limite será denotado por
\[
\lim_{z\to z_0} f(z) = w_0\qquad \text{ou} \qquad f(z)\xrightarrow{\ z\to z_0\ }w_0.
\]
\end{definicao}
É comum referir-se ao fato de que $f(z)\xrightarrow{\ z\to z_0\ }w_0$, dizendo simplesmente
que $f(z)$ converge para $w_0$, quando $z$ tende a $z_0$.

A intuição deste conceito de limite é totalmente análoga a de limite de funções em $\mathbb{R}^2$.
Ou seja, dado qualquer $\varepsilon>0$ podemos encontrar um $\delta\equiv \delta(\varepsilon,z_0)$ 
tal que todo ponto $z$ que está $\delta$-próximo de $z_0$ é enviado por $f$ $\varepsilon$-próximo de $w_0$.

\begin{exemplo}
Seja $f:\mathbb{C}\to\mathbb{C}$ dada por $f(z)=z^2$.  Seja $z_0=i$. Então, segundo a definição acima 
temos 
\[
\lim_{z\to i } f(z) = -1.
\]
De fato, dado $\varepsilon>0$, tome $\delta = \sqrt{1+\varepsilon} -1$. Desta forma, sempre que tomamos
$z$ tal que $|z-(-i)|<\delta$, temos
\begin{align*}
|z^2-(-1)|=|z^2+1|= |(z+i)(z-i)| 
&= |z+i|\cdot |z-i|
\\
&<
(\sqrt{1+\varepsilon} -1) |z-i|
\\
&=
(\sqrt{1+\varepsilon} -1) |z+i-2i|
\\
&\leqslant
(\sqrt{1+\varepsilon} -1) (|z+i|+|2i|)
\\
&<
(\sqrt{1+\varepsilon} -1) (\sqrt{1+\varepsilon} -1+2)
\\
&=
(\sqrt{1+\varepsilon} -1) (\sqrt{1+\varepsilon} +1)
\\
&=
(\sqrt{1+\varepsilon})^2-1^2
\\
&=
\varepsilon.
\end{align*}
\end{exemplo}

Uma propriedade muito importante e naturalmente esperada de qualquer definição de limite é que o mesmo
seja único. Neste caso é simples verificar a unicidade do limite, basta aplicar a desigualdade triangular como
segue. Se $f(z)\to w_0$, quando $z\to z_0$ e $f(z)\to w_1$, quando $z\to z_0$, então dado $\varepsilon>0$
sabemos que existe $\delta>0$ tal que se $0<|z-z_0|<\delta$ 
então $|f(z)-w_0|<\varepsilon/2$ e $|f(z)-w_1|<\varepsilon/2$.
Logo $|w_0-w_1| = |w_0-f(z)+f(z)-w_1|\leq |f(z)-w_0|+|f(z)-w_1|<\varepsilon/2+\varepsilon/2 = \varepsilon$.
Para a desigualdade acima ser válida para todo $\varepsilon$ é necessário que $|w_0-w_1|=0$, o que queríamos
demonstrar.


\bigskip

O limite de uma função complexa pode ser expresso também em termos de suas coordenadas. Em outras palavras, 
uma função complexa tem limite se, e somente se, suas partes real e imaginária
têm limite. Mais precisamente, se escrevemos $f(z)=u(x,y)+iv(x,y)$, onde $z=x+iy$ e se existe 
o limite $\lim_{z\to z_0}f(z)=w_0$, onde $z_0=x_0+iy_0$ e $w_0=u_0+iv_0$, então temos
\begin{align*}
\lim_{(x,y)\to (x_0,y_0)} u(x,y) = u_0
\qquad\text{e}\qquad 
\lim_{(x,y)\to (x_0,y_0)} v(x,y) = v_0.
\end{align*}

Para verificar que ambas igualdades vamos aplicar o Lema \ref{lema-re-im-modulo} que 
garante
\[
|u(x,y)-u_0|\leqslant |f(z)-w_0|
\qquad \text{e}\qquad 
|v(x,y)-v_0|\leqslant |f(z)-w_0|.
\]
Assim, dado $\varepsilon>0$, existe $\delta>0$ tal que 
se $z\in A$ e  $|z-z_0|<\delta$ (ou  $\|(x,y)-(x_0,y_0)\|<\delta$), 
então $|f(z)-w_0|<\varepsilon$. Logo segue das últimas duas desigualdades que $|u(x,y)-u_0|<\varepsilon$
e $|v(x,y)-v_0|<\varepsilon$ o que prova a afirmação sobre os limites das funções coordenadas $u$ e $v$.
A recíproca pode ser provada de forma análoga simplesmente substituindo o argumento do Lema \ref{lema-re-im-modulo}
pela desigualdade triangular.

As propriedades do limite de uma função complexa são análogas às do limite de funções reais, já que 
tudo que são usadas em ambas definições são as estruturas de distância (que na reta é definida pelo valor absoluto e no 
plano complexo pela norma) e de corpo. Relacionamos abaixo algumas das mais básicas e úteis:


\begin{lema}\label{lema-limite-f-limitada}
Seja  $A\subset \mathbb{C}$ conjunto de interior não-vazio, $f:A\to\mathbb{C}$ uma função complexa
e $z_0\in \overline{A}$. Suponha 
\[
\lim_{z\to z_0} f(z) = w_0.
\]
Então existe algum $r>0$ tal que $f$ é limitada no conjunto $V(A,r)\equiv \{z\in A: 0<|z-z_0|<r\}$,
isto é, existe uma constante $M$ (que pode depender de $z_0$ e $r$) tal que   $|f(z)|\leq M$
para todo $z\in V(A,r)$.
\end{lema}
\begin{proof}
Pela definição de limite, dado $\varepsilon=1$ existe $\delta>0$ tal que para todo $z\in A$ com 
$0<|z-z_0|<\delta$, temos $|f(z)-w_0|<1$. Pela segunda desigualdade triangular segue que 
$|f(z)|-|w_0|\leqslant |f(z)-w_0|<1$, o que implica que $|f(z)|\leqslant 1+|w_0|$. 
Para finalizar a prova basta tomar $M\equiv 1+|w_0|$ e $r=\delta$.
\end{proof}


\begin{proposicao}\label{prop-propriedades-limite-funcao-complexa}
Sejam $A\subset \mathbb{C}$ conjunto de interior não-vazio, $f_1,f_2:A\to\mathbb{C}$ duas funções complexas
e $z_0\in \overline{A}$. Se 
\[
\lim_{z\to z_0} f_1(z) = w_1\qquad\text{e}\qquad \lim_{z\to z_0} f_2(z) = w_2
\]
então 
\begin{enumerate}
\item $\displaystyle \lim_{z\to z_0} cf_1(z) = c \lim_{z\to z_0} f_1(z) = cw_1$, onde $c$ é uma constante complexa;
\item $\displaystyle \lim_{z\to z_0} (f_1(z)+f_2(z)) = \lim_{z\to z_0} f_1(z) + \lim_{z\to z_0}f_2(z) = w_1+w_2$;
\item $\displaystyle \lim_{z\to z_0} (f_1(z)\cdot f_2(z)) = \lim_{z\to z_0} f_1(z) \cdot \lim_{z\to z_0}f_2(z) = w_1\cdot w_2$;
\item se $w_1\neq 0$ então $\displaystyle \lim_{z\to z_0}\frac{1}{f_1(z)} = \frac{1}{w_1}$.
\end{enumerate}
\end{proposicao}

\begin{proof}
Vamos provar os itens $\mathit{2}$ e $\mathit{3}$. A prova dos itens restantes é semelhante. 

Prova do item $\mathit{2}$. Pela definição de limite, dado $\varepsilon>0$ existem $\delta_1,\delta_2>0$ tais que se $|z-z_0|<\delta_1$
então $|f_1(z)-w_1|<\varepsilon/2$ e se $|z-z_0|<\delta_2$ então $|f_2(z)-w_2|<\varepsilon/2$.
Portanto, se $|z-z_0|<\delta\equiv \min\{\delta_1,\delta_2\}$ temos da desigualdade triangular que  
\[
|f_1(z)+f_2(z) - (w_1+w_2)|\leqslant |f_1(z)-w_1|+|f_2(z)-w_2| <\frac{\varepsilon}{2}+\frac{\varepsilon}{2} = \varepsilon,
\]
mostrando que $f_1(z)+f_2(z)\xrightarrow{\ z\to z_0\ }w_1+w_2$.

Prova do item $\mathit{3}$. Pela definição de limite dado $\varepsilon>0$ sabemos que existe $\delta_1>0$
tal que se $|z-z_0|<\delta_1$, então $|f_1(z)-w_1|<\varepsilon/2|w_2|$. Pelo Lema \ref{lema-limite-f-limitada}
existem $\delta_2>0$ e uma constante $M>0$ tal que se $|z-z_0|<\delta_2$, então $|f_1(z)|\leqslant M$.
Também pela definição de limite podemos encontrar $\delta_3>0$ tal que se $|z-z_0|<\delta_3$ 
então $|f_2(z)-w_2|<\varepsilon/2M$. Aplicando a desigualdade triangular temos
\begin{align*}
|f_1(z)f_2(z) - w_1w_2 | &=  |f_1(z)f_2(z)- f_1(z)w_2+f_1(z)w_2-w_1w_2|
\\
&\leqslant 
|f_1(z)f_2(z)- f_1(z)w_2|+|f_1(z)w_2-w_1w_2|
\\
&=
|f_1(z)|\, |f_2(z)-w_2| + |w_2|\, |f_1(z)-w_1|
\\
&\leqslant
M \frac{\varepsilon}{2M}+|w_2| \frac{\varepsilon}{2|w_2|}
=
\varepsilon.
\end{align*}
Mostrando assim que $f_1(z)\cdot f_2(z)\xrightarrow{\ z\to z_0\ }w_1\cdot w_2$.
\end{proof}

De maneira semelhante às funções de $\mathbb{R}ˆ2$ definimos o conceito de continuidade.  

\begin{definicao}[Função Contínua]
\label{def-func-complexa-continua}
\index{Função!contínua}
Sejam $A\subset \mathbb{C}$ um aberto e $f:A\to\mathbb{C}$ uma função. Dizemos que 
$f$ é contínua em $z_0\in A$ se 
\[
\lim_{z\to z_0} f(z) = f(z_0).
\]
Se $f$ é contínua em todo ponto $z_0\in A$, dizemos que $f$ é contínua em $A$.
\end{definicao}

Note que a continuidade de uma função complexa é equivalente a continuidade de 
suas funções coordenadas. Isto é, $f:A\to\mathbb{C}$ é contínua se, e somente se,
as funções $z\longmapsto \Re(z)$ e $z\longmapsto \Im(z)$ são contínuas em $A$.

Podemos também verificar diretamente à partir da definição 
que $f:\mathbb{C}\to\mathbb{C}$ dada por $f(z)=\overline{z}$ é uma função contínua em $\mathbb{C}$.
Usando a proposição anterior podemos também verificar facilmente que 
a função $g:\mathbb{C}^{*}\to\mathbb{C}$ dada por $g(z) = 1/z$ é contínua em $\mathbb{C}^{*}$.

A seguir vamos mostrar que uma versão da  Proposição \ref{prop-propriedades-limite-funcao-complexa}
para o caso de funções contínuas. Além disto vamos incluir nesta nova proposição uma afirmação sobre
a continuidade de composições e para isto vamos relembrar o conceito de composição de funções. 
Sejam $A,B\subset \mathbb{C}$ conjuntos abertos, $f:B\to\mathbb{C}$ e $g:A\to\mathbb{C}$
funções tais que $g(A)\subset B$. Sob estas condições podemos definir a função composta $f\circ g: A\to\mathbb{C}$
pela expressão $f\circ g(z) \equiv f(g(z))$, para cada $z\in A$.

\begin{proposicao}
Sejam $A,B\subset \mathbb{C}$ abertos e $f_1:B\to\mathbb{C}$, $f_2:B\to\mathbb{C}$ e $g:A\to\mathbb{C}$ 
funções complexas tais que $g(A)\subset B$. Suponha que $f_1$ e $f_2$ são contínuas em $z_0\in B$ e
que $f_1$ é contínua em $g(z_1)$, onde $z_1\in A$. Então 
\begin{enumerate}
	\item são contínuas em $z_0\in B$ as funções:
		\begin{itemize}
			\item  $cf_1:B\to\mathbb{C}$; 
			\item  $f_1+f_2:B\to\mathbb{C}$;
			\item $f_1\cdot f_2: B\to\mathbb{C}$.
		\end{itemize}

	\item se $f_1(z_0)\neq 0$ então a função $\displaystyle\frac{1}{f_1}:B\setminus\{z\in B: f_1(z)=0\} \to \mathbb{C}$
	é contínua em $z_0$;
	
	\item a função $f_1\circ g :A\to\mathbb{C}$ é contínua em $z_1$.
\end{enumerate}
\end{proposicao}

\begin{proof}
As provas dos itens $\mathit{1}$ e $\mathit{2}$ são idênticas aos respectivos itens da Proposição \ref{prop-propriedades-limite-funcao-complexa} por isto faremos apenas a prova do item $\mathit{3}$.
Dado $\varepsilon>0$ segue da continuidade de $f_1$ no ponto $g(z_1)$, 
que existe $\delta_1>0$ tal que se $|w-g(z_1)|<\delta_1$
então $|f_1(w)-f_1(g(z_1))|<\varepsilon$. Pela continuidade de $g$ em $z_1$ podemos encontrar $\delta_2>0$
tal que se $|z-z_1|<\delta_2$ então $|g(z)-g(z_1)|<\delta_1$. Assim,
tomando $\delta=\delta_2$ temos que se 
 se $|z-z_1|<\delta$ então $|f_1(g(z))-f_1(g(z_1))|<\varepsilon$.
\end{proof}


\section{A Derivada Complexa}

Nesta seção vamos definir a derivada complexa de uma função $f:A\subset\mathbb{C}\to\mathbb{C}$.
Para facilitar vamos trabalhar apenas com funções definidas em subconjuntos abertos de $\mathbb{C}$.
A noção de derivada complexa é uma noção que diferencia o cálculo de variáveis reais e complexas.
A partir dela veremos surgir uma teoria de cálculo diferencial completamente nova e rica de propriedades
interessantes. As novidades oriundas desta nova noção de derivada 
fazem desta, uma teoria extremamente útil em diversas aplicações em matemática pura e aplicada, 
física, engenharia e muitas outras áreas.

\begin{definicao}\label{def-derivada-complexa}
\index{Derivada!complexa}
Sejam $A\subset\mathbb{C}$ um aberto e $f:A\to\mathbb{C}$ uma função. Dizemos que $f$
tem derivada no sentido complexo em $z_0\in A$ se existe o seguinte limite 
\[
f'(z_0) \equiv \lim_{z\to z_0} \frac{f(z)-f(z_0)}{z-z_0}.
\]
\end{definicao}

A diferença marcante aqui é que apesar da derivada ser definida como no caso de funções reais,
pelo limite de quocientes de Newton, agora o quociente de Newton é tomado no sentido da divisão 
de números complexos. Esta que parece uma pequena diferença terá consequências fantásticas!

Observamos que analogamente ao caso de funções reais a derivada complexa
pode também ser obtida  pelo seguinte limite 
\[
f'(z_0) = \lim_{h\to 0} \frac{f(z_0+h)-f(z_0)}{h}.
\]

Abaixo analisamos a existência da derivada complexa de duas funções bastante simples e importantes
neste texto.

\begin{exemplo}
A função $f:\mathbb{C}\to\mathbb{C}$ tem derivada complexa em qualquer $z_0\in\mathbb{C}$, além do mais
$f'(z_0)=2z_0$. 
Para verificar este fato, basta observar que 
\[
\lim_{z\to z_0}\frac{ f(z)-f(z_0)}{z-z_0} 
=
\lim_{z\to z_0} \frac{z^2-z_0^2}{z-z_0}
=
\lim_{z\to z_0} \frac{(z-z_0)(z+z_0)}{z-z_0}
=
\lim_{z\to z_0} z+z_0
=
2z_0.
\]
\end{exemplo}
 
O próximo exemplo mostra como a derivada complexa pode ser radicalmente diferente da derivada 
no sentido real. 

\begin{exemplo}
Considere a função $f:\mathbb{C}\to\mathbb{C}$ dada por $f(z)=\overline{z}$, para todo $z\in\mathbb{C}$.
Então não existe nenhum ponto do plano complexo onde esta função tenha derivada.

Para verificar este fato vamos considerar $z_0$ um ponto arbitrário fixado e o quociente de Newton 
\begin{align*}
\frac{f(z)-f(z_0)}{z-z_0}
&=
\frac{\overline{z}-\overline{z_0}}{z-z_0}
\\
&=
\frac{\overline{z}-\overline{z_0}}{z-z_0} \cdot \frac{\overline{z}-\overline{z_0}}{\overline{z}-\overline{z_0}}
=
\frac{(\overline{z}-\overline{z_0})^2}{|z-z_0|^2} 
\\
&=
\frac{(x-x_0+i(y-y_0))^2}{(x-x_0)^2+(y-y_0)^2}
\\
&=
\frac{(x-x_0)^2-(y-y_0)^2+2i(x-x_0)(y-y_0)}{(x-x_0)^2+(y-y_0)^2}.
\end{align*}
Agora vamos fazer $z$ se aproximar de $z_0$ por retas paralelas aos eixos coordenados que passando por 
$z_0$.

Primeiro consideramos  $z$ da forma $z=t+iy_0$, onde $t\in\mathbb{R}$. Neste caso obtemos, das identidades acima,
para $t\neq x_0$ a seguinte igualdade
\begin{align*}
\frac{f(t+iy_0)-f(z_0)}{(t+iy_0)-z_0}
&=
\frac{(t-x_0)^2-(y_0-y_0)^2+2i(t-x_0)(y_0-y_0)}{(t-x_0)^2+(y_0-y_0)^2}
\\
&=
\frac{(t-x_0)^2}{(t-x_0)^2}=1.
\end{align*}

Por outro lado, se consideramos $z$ da forma $z=x_0+it$, onde $t\in\mathbb{R}$ temos para $t\neq y_0$
\begin{align*}
\frac{f(x_0+it)-f(z_0)}{(x_0+it)-z_0}
&=
\frac{(x_0-x_0)^2-(t-y_0)^2+2i(x_0-x_0)(t-y_0)}{(x_0-x_0)^2+(t-y_0)^2}
\\
&=
-\frac{(t-y_0)^2}{(t-y_0)^2}=-1.
\end{align*}
Como os limite quando $z$ tende a $z_0$ do quociente de Newton ao longo destes dois caminhos é distinto
segue que não pode existir a derivada de $f$ no ponto $z_0$. Mas já que $z_0$ é arbitrário, concluímos que
esta função não tem derivada em nenhum ponto do plano complexo.
\end{exemplo}

O exemplo acima, mostra como o conceito de derivada complexa é novo. Além do mais ele mostra
que em $\mathbb{C}$ é muito simples construir um função que é contínua em todos pontos
mas não é derivável em ponto algum. Existem funções com propriedades semelhantes definidas da reta na reta,
mas são muito mais complicadas de serem construídas. 


\begin{proposicao}\label{prop-diferenciavel-continua}
Sejam $A\subset\mathbb{C}$ um aberto e  $f:A\to\mathbb{C}$ uma função derivável em $z_0\in A$.
Então $f$ é contínua em $z_0$.
\end{proposicao}
\begin{proof}
Basta observar que 
\begin{align*}
\lim_{z\to z_0} (f(z)-f(z_0))
&=
\lim_{z\to z_0} (f(z)-f(z_0))\, \frac{z-z_0}{z-z_0}
\\
&=
\lim_{z\to z_0}  (z-z_0)\, \frac{f(z)-f(z_0)}{z-z_0}
\\
&=
0\cdot f'(z_0)=0.
\end{align*}
\end{proof}


\begin{proposicao}\label{prop-derivada}
Sejam $A\subset\mathbb{C}$ um aberto, $f$ e $g$ funções diferenciáveis em $z_0\in A$. 
Então temos 
\begin{enumerate}
	\item $(cf)'(z_0) = cf'(z_0)$, para qualquer constante $c\in\mathbb{C}$;
	\item $(f+g)'(z_0) = f'(z_0)+g'(z_0)$;
	\item $(f\cdot g)'(z_0) = f'(z_0)g(z_0)+f(z_0)g'(z_0)$;
	\item $\displaystyle\left( \frac{1}{f} \right)'(z_0) = -\frac{f'(z_0)}{f(z_0)^2}$, se $f(z_0)\neq 0$.
	\item $\displaystyle\left( \frac{f}{g} \right)'(z_0) = \frac{f'(z_0)g(z_0)-f(z_0)g'(z_0)}{g(z_0)^2}$, se $g(z_0)\neq 0$.
\end{enumerate}
\end{proposicao}

\begin{proof}
A prova dos itens $\mathit{1},\ \mathit{2}$ e $\mathit{3}$ são semelhantes a das proposições anteriores.
Por isto vamos apresentar somente a prova dos itens $\mathit{4}$ e $\mathit{5}$.

Para provar $\mathit{4}$ primeiro observamos que podemos reescrever o 
quociente de Newton de $1/f$ como segue já que  $f(z_0)\neq 0$. 
\begin{align*}
\frac{ \displaystyle \frac{1}{f(z)}  -  \frac{1}{f(z_0)}  }{z-z_0}
=
\frac{ \displaystyle  \frac{ f(z_0)-f(z) }{ f(z)f(z_0) }    }{z-z_0}
=
- \frac{1}{f(z)f(z_0)}\cdot   \frac{f(z)-f(z_0)}{z-z_0},
\end{align*}
Como estamos assumindo que $f$ é derivável em $z_0$ segue da Proposição \ref{prop-diferenciavel-continua}
que $f$ é contínua em $z_0$. Desta forma podemos tomar o limite, quando $z$ tende a $z_0$ na expressão 
à direita da igualdade acima obtendo 
\[
\lim_{z\to z_0} \frac{ \displaystyle \frac{1}{f(z)}  -  \frac{1}{f(z_0)}  }{z-z_0}
 = 
- \lim_{z\to z_0} \frac{1}{f(z)f(z_0)}\cdot   \frac{f(z)-f(z_0)}{z-z_0}
=
-\frac{f'(z_0)}{f(z_0)^2}.
\]

A prova do item $\mathit{5}$. Já que estamos assumindo que $g$ é derivável em $z_0$ e também $g(z_0)\neq 0$
podemos aplicar os itens $\mathit{3}$ e $\mathit{4}$ para concluir que 
\begin{align*}
\left( \frac{f}{g} \right)'(z_0) 
=
\left( f \cdot \frac{1}{g}\right)'(z_0)
&=
f'(z_0) \frac{1}{g(z_0)} + f(z_0)\left( \frac{1}{g}\right)'(z_0)
\\[0.3cm]
&=
 \frac{f'(z_0)}{g(z_0)}  -f(z_0)\frac{g'(z_0)}{g(z_0)^2}
\\[0.2cm]
&=
\frac{f'(z_0)g(z_0)-f(z_0)g'(z_0)}{g(z_0)^2}. 
\end{align*} 
\end{proof}

\begin{teorema}[Regra da Cadeia]
\label{teo-regra-da-cadeia}
\index{Regra da Cadeia}
Sejam $A, B$ abertos de $\mathbb{C}$ e $f:B\to\mathbb{C}$ e $g:A\to\mathbb{C}$ funções tais que $g(A)\subset B$.
Se $g$ é derivável em $z_0$ e $f$ é derivável em $f(g(z_0))$, então 
\[
(f\circ g)'(z_0) = f'(g(z_0))g'(z_0).
\]
\end{teorema}

\begin{proof}
Defina a função $h:B\to\mathbb{C}$ como segue
\[
h(w)
=
\begin{cases}
\displaystyle\frac{f(w)-f(g(z_0))}{w-g(z_0)}- f'(g(z_0)),&\text{se}\ w\neq g(z_0);\\
0,&\text{se}\ w=g(z_0).
\end{cases} 
\]
Observe que $h$ está bem definida em $B$ e é contínua em $g(z_0)$
uma vez que $f$, por hipótese, é derivável em $g(z_0)$. 
Como também é continuidade $g$ em $z_0$, temos
\begin{align}\label{eq-aux1-regra-cadeia}
\lim_{z\to z_0} (h\circ g)(z) = h(g(z_0))=0.
\end{align}
Para todo $w\in B$ temos a seguinte igualdade (válida também para $w=g(z_0)$)
\[
f(w)-f(g(z_0))= ( h(w)+f'(g(z_0)))\,(w-g(z_0)).
\]
Substituindo $w$ por $g(z)$ na identidade acima e dividindo ambos os lados por $z-z_0$ obtemos:
\[
\frac{ f(g(z)) - f(g(z_0))  }{ z-z_0 } 
=  
\big(  h(g(z))+ f'(g(z_0))   \big) 
\,\frac{g(z)-g(z_0)}{z-z_0}.
\]
Lembrando de \eqref{eq-aux1-regra-cadeia}  
podemos concluir  da expressão do lado direito da igualdade acima que existe a derivada de $f\circ g$
em $z_0$ e temos
\begin{align*}
(f\circ g)'(z_0)
&=
\lim_{z\to z_0}
\frac{ f(g(z)) - f(g(z_0))  }{ z-z_0 } 
\\[0.2cm]
&=  
\lim_{z\to z_0}
\big(  h(g(z))+ f'(g(z_0))   \big) \frac{g(z)-g(z_0)}{z-z_0}
\\[0.2cm]
&=
f'(g(z_0))g'(z_0).
\qedhere
\end{align*}
\end{proof}


\section{As Condições de Cauchy-Riemann}

Nesta seção vamos analisar algumas das consequências da existência da derivada complexa.
Em particular, estaremos interessados nas propriedades funções coordenadas de uma função 
que tem derivada complexa em uma certa região do plano complexo. Um dos resultados
principais são as condições de Cauchy-Riemann e sua recíproca enunciados na sequência.

\begin{proposicao}[Condições de Cauchy-Riemann]
\label{prop-cond-Cauchy-Riemann}
\index{Condições!de Cauchy-Riemann}
Seja $A\subset \mathbb{C}$ um aberto, $z_0=x_0+iy_0\in A$ e $f:A\to\mathbb{C}$
uma função da forma $f(x+iy)=u(x,y)+iv(x,y)$  derivável em $z_0$. Então
\[
\frac{\partial}{\partial x}u(x_0,y_0)=\frac{\partial}{\partial y}v(x_0,y_0)
\quad\text{e}\quad
\frac{\partial}{\partial x}v(x_0,y_0)=-\frac{\partial}{\partial y}u(x_0,y_0).
\]
\end{proposicao}

\begin{proof}
Primeiro passo é expressar em coordenadas o quociente de Newton de $f$ como segue:
\begin{align*}
\frac{f(z)-f(z_0)}{z-z_0}
&=
\frac{u(x,y)-u(x_0,y_0)+i(v(x,y)-v(x_0,y_0))}{x-x_0+i(y-y_0)}
\\[0.3cm]
&=
\frac{u(x,y)-u(x_0,y_0)+i(v(x,y)-v(x_0,y_0))}{x-x_0+i(y-y_0)}\cdot \frac{x-x_0-i(y-y_0)}{x-x_0-i(y-y_0)}
\\[0.3cm]
&=
\frac{u(x,y)-u(x_0,y_0)+i(v(x,y)-v(x_0,y_0))}{(x-x_0)^2+(y-y_0)^2}\cdot\big(x-x_0-i(y-y_0)\big)
\\[0.3cm]
&=
\frac{(u(x,y)-u(x_0,y_0))(x-x_0)+(v(x,y)-v(x_0,y_0))(y-y_0)}{(x-x_0)^2+(y-y_0)^2}
\\
&\quad +i
\frac{(v(x,y)-v(x_0,y_0))(x-x_0)-(u(x,y)-u(x_0,y_0))(y-y_0)}{(x-x_0)^2+(y-y_0)^2}.
\end{align*}

Já que estamos assumindo que $f$ é diferenciável em $z_0$ o limite, quando $z\to z_0$, 
do quociente de Newton acima existe e é independente do caminho.
Desta forma podemos tomá-lo ao longo dos segmentos de reta passando por $z_0$ e paralelos aos eixos 
coordenados. 

Tomando o limite, na igualdade acima, quando $z\to z_0$ ao longo da reta $t+iy_0$, com $t\to x_0$ obtemos
\begin{align*}
f'(z_0)
&=
\lim_{t\to x_0} 
\frac{f(t+iy_0)-f(z_0)}{t+iy_0-(x_0+iy_0)}
\\[0.3cm]
&=
\lim_{t\to x_0} 
\frac{(u(t,y_0)-u(x_0,y_0))(t-x_0)+(v(t,y_0)-v(x_0,y_0))(y_0-y_0)}{(t-x_0)^2+(y_0-y_0)^2}
\\
&\quad +i \lim_{t\to x_0} 
\frac{(v(t,y_0)-v(x_0,y_0))(t-x_0)-(u(t,y_0)-u(x_0,y_0))(y_0-y_0)}{(t-x_0)^2+(y_0-y_0)^2}
\\[0.3cm]
&=
\lim_{t\to x_0} 
\frac{u(t,y_0)-u(x_0,y_0)}{t-x_0} 
+i \lim_{t\to x_0} 
\frac{v(t,y_0)-v(x_0,y_0)}{t-x_0}
\\[0.3cm]
&=
\frac{\partial }{\partial x}u(x_0,y_0) + i\frac{\partial }{\partial x}v(x_0,y_0).
\end{align*}

Repetindo o procedimento acima, mas agora fazendo $z$ tender a $z_0$ pela reta 
$x_0+it$, com $t\to y_0$ temos

\begin{align*}
f'(z_0)
&=
\lim_{t\to y_0} 
\frac{f(x_0+it)-f(z_0)}{x_0+it-(x_0+iy_0)}
\\[0.3cm]
&=
\lim_{t\to y_0} 
\frac{(u(x_0,t)-u(x_0,y_0))(x_0-x_0)+(v(t,y_0)-v(x_0,y_0))(t-y_0)}{(x_0-x_0)^2+(t-y_0)^2}
\\
&\quad +i \lim_{t\to y_0} 
\frac{(v(x_0,t)-v(x_0,y_0))(x_0-x_0)-(u(x_0,t)-u(x_0,y_0))(t-y_0)}{(x_0-x_0)^2+(t-y_0)^2}
\\[0.3cm]
&=
\lim_{t\to y_0} 
\frac{v(x_0,t)-v(x_0,y_0)}{t-y_0} 
-i \lim_{t\to x_0} 
\frac{u(t,y_0)-u(x_0,y_0)}{t-y_0}
\\[0.3cm]
&=
\frac{\partial }{\partial y}v(x_0,y_0) -i \frac{\partial }{\partial y}u(x_0,y_0).
\end{align*}

Dos dois últimos conjuntos de igualdades segue o resultado.
\end{proof}

Uma maneira muito prática de lembrar das condições de Cauchy-Riemann consistem em lembrar que 
$f$ vista como função de $\mathbb{R}^2$ em $\mathbb{R}^2$ tem derivada total representada
por sua matriz Jacobiana (matriz dos gradientes das funções coordenadas) $2\times 2$ como mostrado abaixo:
\[
Df(x_0,y_0)
=
\begin{pmatrix}
\displaystyle \frac{\partial}{\partial x}u(x_0,y_0) & \displaystyle \frac{\partial}{\partial y} u(x_0,y_0)
\\[0.5cm] 
\displaystyle \frac{\partial}{\partial x}v(x_0,y_0) & \displaystyle \frac{\partial}{\partial y} v(x_0,y_0)
\end{pmatrix}.
\]
Desta forma para que este Jacobiano represente um número complexo e necessário que ele seja da forma 
\[
\begin{pmatrix}
a&b\\
-b&a
\end{pmatrix}
\]
o que se traduz precisamente nas condições de Cauchy-Riemann.

\bigskip 

Como vimos acima as condições de Cauchy-Riemann são necessárias para que $f$ seja derivável em um 
ponto $z_0$ em seu domínio. Mas o exemplo abaixo mostra que elas podem, em geral, não ser suficientes
para garantir a existência da derivada complexa

\begin{exemplo}
A função $f:\mathbb{C}\to\mathbb{C}$ dada por 
\[
f(x+iy)
=
\begin{cases}
0,& \text{se}\ xy=0;
\\
1,& \text{se}\ xy\neq 0.
\end{cases}
\]
satisfaz as condições de Cauchy-Riemann em $z_0=0$, pois as derivadas parciais das funçoões coordenadas $u$ e $v$
são nulas na origem. Porém ela não pode ter derivada complexa na origem
já que ela não é nem mesmo contínua neste ponto.
\end{exemplo}

Uma condição razoavelmente geral que a garante a existência da derivada complexa de uma função $f$
satisfazendo as condições de Cauchy-Riemann será dada na próxima proposição. Mas para prová-la
vamos precisar do seguinte lema auxiliar

\begin{lema}\label{lema-aux-reciproca-CR}
Sejam $A\subset\mathbb{R}^2$ aberto e $F:A\to\mathbb{R}$ uma função admitindo derivadas parciais contínuas 
em $(x_0,y_0)\in A$. Então temos 
\begin{align}\label{eq-aux1-lema-aux-reciproca-CR}
F(x,y)-F(x_0,y_0)
&=
(x-x_0)\left(  \frac{\partial }{\partial x}F(x_0,y_0) +H(x-x_0,y-y_0) \right)
\\
&\quad +
(y-y_0)\left(  \frac{\partial }{\partial y}F(x_0,y_0) +K(x-x_0,y-y_0) \right),
\end{align}
onde 
\begin{align*}
\lim_{(x,y)\to (x_0,y_0)} H(x-x_0,y-y_0)=0
\quad\text{e}\quad
\lim_{(x,y)\to (x_0,y_0)} H(x-x_0,y-y_0)=0.
\end{align*}
\end{lema}

\begin{proof}
A prova deste lema é baseada no Teorema do Valor Médio para funções definidas na reta.
Vamos considerar $x=x_0+h$ e $y=y_0+k$. Como $A$ é aberto existe $\delta>0$ tal que 
se $\|(h,k)\|<\delta$ então $(x_0+h,y_0+k)\in A$. 

Note que 
\begin{align}\label{eq-aux1-lema-prep-reciproca-CR}
F(x_0+h,y_0+k) -F(x_0,y_0) 
&= 
F(x_0+h,y_0+k)-F(x_0,y_0+k) 
\nonumber \\
&\quad + F(x_0,y_0+k)-F(x_0,y_0).
\end{align}
Aplicando o Teorema do Valor Médio a primeira diferença, do lado direito da igualdade acima, podemos
garantir que existe $0<t<1$ tal que 
\[
\frac{F(x_0+h,y_0+k)-F(x_0,y_0+k)}{h} = \frac{\partial }{\partial x}F(x_0+th,y_0+k).
\]
Como estamos assumindo que as derivadas parciais de $F$ são contínuas, temos que 
a diferença que aparece abaixo tende a zero, quando $(h,k)\to (0,0)$
\[
H(h,k) = \frac{\partial}{\partial x}F(x_0+th,y_0+k)-\frac{\partial}{\partial x}F(x_0,y_0).
\]
Isolando o termo $\frac{\partial}{\partial x}F(x_0+th,y_0+k)$ na igualdade acima 
e o substituindo na igualdade anterior, obtemos a seguinte identidade
\begin{align}\label{eq-aux2-lema-prep-reciproca-CR}
F(x_0+h,y_0+k)-F(x_0,y_0+k) = h \left( \frac{\partial}{\partial x}F(x_0,y_0) + H(h,k)  \right).
\end{align}

Similarmente, temos para a segunda diferença em \eqref{eq-aux1-lema-prep-reciproca-CR}
podemos mostrar que existe $0<t'<1$ tal que
\[
\frac{F(x_0,y_0+k)-F(x_0,y_0)}{t'}
=
\frac{\partial }{\partial x}F(x_0,y_0+t'k).
\]
Pela continuidade das derivadas parciais temos  
\[
K(h,k) = \frac{\partial }{\partial y}F(x_0,y_0+t'k)-\frac{\partial}{\partial y}F(x_0,y_0)
\]
tende a zero quando $(h,k)\to (0,0)$. Procedendo de maneira análoga obtemos 
\begin{align}\label{eq-aux3-lema-prep-reciproca-CR}
F(x_0,y_0+k)-F(x_0,y_0) = k\left( \frac{\partial }{\partial x}F(x_0,y_0) + K(h,k)  \right).
\end{align}
Para encerrar a prova basta somar as igualdades 
\eqref{eq-aux2-lema-prep-reciproca-CR} e \eqref{eq-aux3-lema-prep-reciproca-CR}.
\end{proof}


\begin{proposicao}\label{prop-reciproca-CR}
Sejam $A\subset\mathbb{C}$ um aberto e $f:A\to\mathbb{C}$ uma função complexa
dada por $f(x+iy)=u(x,y)+iv(x,y)$, admitindo derivadas parciais 
\[
\frac{\partial }{\partial x}u(x,y),\quad 
\frac{\partial }{\partial y}u(x,y),\quad 
\frac{\partial }{\partial x}v(x,y),\quad 
\frac{\partial }{\partial y}v(x,y),
\]
em todos os pontos de algum disco $D(z_0,r)\subset A$ e contínuas em $z_0=x_0+iy_0$.
Se as Condições de Cauchy-Riemann são satisfeitas em $z_0$ então $f$ possui derivada em $z_0$.
Além do mais, temos que $f'(z_0)$ pode ser descrita de duas formas alternativas:
\[
f'(z_0) 
=
\frac{\partial}{\partial x}u(x_0,y_0) +i \frac{\partial}{\partial x}v(x_0,y_0) 
\]
ou
\[
f'(z_0)
=
\frac{\partial }{\partial y}v(x_0,y_0) -i \frac{\partial }{\partial y}u(x_0,y_0).
\]
\end{proposicao}

\begin{proof}
Uma das ideias principais desta prova é escrever a função $f$ em termos de suas coordenadas e em
seguida usar o Lema \ref{lema-aux-reciproca-CR}. 

Para cada $z\in A$ temos 
\begin{align*}
f(z)-f(z_0)
&=
u(x,y)-u(x_0,y_0)+i(v(x,y)-v(x_0,y_0))
\\[0.3cm]
&=
(x-x_0)\left( \frac{\partial}{\partial x}u(x_0,y_0) +H_1(x-x_0,y-y_0)  \right)
\\
&\quad + 
(y-y_0)\left( \frac{\partial}{\partial y}u(x_0,y_0) +K_1(x-x_0,y-y_0)  \right)
\\
&\quad +
i (x-x_0)\left( \frac{\partial}{\partial x}v(x_0,y_0) +H_2(x-x_0,y-y_0)  \right)
\\
&\quad +
i (y-y_0)\left( \frac{\partial}{\partial y}v(x_0,y_0) +K_2(x-x_0,y-y_0)  \right).
\end{align*}
Usando as condições de Cauchy-Riemann podemos reescrever a igualdade acima como segue
\begin{align*}
f(z)-f(z_0)
&=
(z-z_0)\left(  \frac{\partial}{\partial x}u(x_0,y_0) +i \frac{\partial}{\partial x}v(x_0,y_0)   \right)
\\
&\quad +
(H_1(x-x_0,y-y_0) +i H_2(x-x_0,y-y_0)) (x-x_0)
\\
&\quad +
(K_1(x-x_0,y-y_0) +i K_2(x-x_0,y-y_0)) (y-y_0).
\end{align*}

Próximo passo é dividir ambos os lados da igualdade acima por $z-z_0$, ficando com a seguinte identidade
\begin{align}\label{eq-aux1-reciproca-CR}
\frac{f(z)-f(z_0)}{z-z_0}
&=
\frac{\partial}{\partial x}u(x_0,y_0) +i \frac{\partial}{\partial x}v(x_0,y_0) 
\nonumber\\
&\quad +
(H_1(x-x_0,y-y_0) +i H_2(x-x_0,y-y_0)) \frac{x-x_0}{z-z_0}
\nonumber\\
&\quad +
(K_1(x-x_0,y-y_0) +i K_2(x-x_0,y-y_0)) \frac{y-y_0}{z-z_0}.
\end{align}

Pelo Lema \ref{lema-re-im-modulo} temos para todo $z\neq z_0$
\[
\left| \frac{x-x_0}{z-z_0} \right|\leqslant 1
\qquad\text{e}\qquad 
\left| \frac{y-y_0}{z-z_0} \right|\leqslant 1.
\]
Estas duas estimativas junto com o fato que as quatro funções $H_1,H_2,K_1$ e $K_2$
tendem a zero quando $z\to z_0$ implicam em 
\[
\lim_{z\to z_0}
(H_1(x-x_0,y-y_0) +i H_2(x-x_0,y-y_0)) \frac{x-x_0}{z-z_0}
=
0
\]
e
\[
\lim_{z\to z_0}
(K_1(x-x_0,y-y_0) +i K_2(x-x_0,y-y_0)) \frac{x-x_0}{z-z_0}
=
0.
\]
Usando estes dois limites e a  igualdade \eqref{eq-aux1-reciproca-CR} podemos finalmente verificar que existe o limite
\[
f'(z_0)
=
\lim_{z\to z_0}\frac{f(z)-f(z_0)}{z-z_0}
=
\frac{\partial}{\partial x}u(x_0,y_0) +i \frac{\partial}{\partial x}v(x_0,y_0) .
\]
\end{proof}

\begin{exemplo}
Uma aplicação direta da Proposição \ref{prop-reciproca-CR} mostra que a função  
$f(z) =z\overline{z}$ só possui derivada na origem. De fato, neste caso temos 
$f(x+iy)= x+ 2+y^2$, logo $u(x,y)=x^2+y^2$ e $v(x,y)=0$ para todo $x+iy\in\mathbb{C}$. 
Logo
\[
\frac{\partial }{\partial x}u(x,y) = 2x,\quad 
\frac{\partial }{\partial y}u(x,y)= 2y,\quad 
\frac{\partial }{\partial x}v(x,y) = 0,\quad 
\frac{\partial }{\partial y}v(x,y) = 0.
\]

Da Proposição \ref{prop-cond-Cauchy-Riemann} segue que em todos os pontos onde $f$ 
tem derivada as Condições de Cauchy-Riemann devem ser satisfeitas. Analisando 
a derivads parciais acima verificamo que as condições de Cauchy-Riemann são verficadas apenas no ponto $z=0$.
Como  as derivadas parciais de $u$ e $v$ são contínuas podemos aplicar a  Proposição \ref{prop-reciproca-CR}
para garantir que existe $f'(0)=0$. Então $f$ possui derivada apenas em $z=0$.
\end{exemplo}



Podemos reformular, heuristicamente, as condições de Cauchy-Riemann em termos das variáveis $z$ e $\overline{z}$.
Para fazer isto usamos inicialmente as seguintes relações
\begin{align}\label{eq-CR-z-zbarra}
x = \frac{z+\overline{z}}{2}
\qquad\text{e}\qquad 
y=\frac{z-\overline{z}}{2i}.
\end{align}
Depois representamos uma função complexa, em coordenadas, em termos dessas variáveis como
mostrado abaixo
\begin{align*}
f(x+iy) = u(x,y)+iv(x,y) 
= 
u\left( \frac{z+\overline{z}}{2},\frac{z-\overline{z}}{2i} \right) 
+
v\left( \frac{z+\overline{z}}{2},\frac{z-\overline{z}}{2i} \right).
\end{align*}
Em seguida apelamos para uma suposta  Regra da Cadeia para concluir que 
\begin{align}\label{eq-aux1-CR-z-zbarra}
\frac{\partial f}{\partial \overline{z}}
=
\frac{\partial u}{\partial x}
\frac{\partial x}{\partial \overline{z}}
+
\frac{\partial u}{\partial y}
\frac{\partial y}{\partial \overline{z}}
+i
\left(
\frac{\partial v}{\partial x}
\frac{\partial x}{\partial \overline{z}}
+
\frac{\partial v}{\partial y}
\frac{\partial y}{\partial \overline{z}}
\right)
\end{align}
Pensando em $z$ e $\overline{z}$ nas igualdades \eqref{eq-CR-z-zbarra} como variáveis, temos que
\[
\frac{\partial x}{\partial \overline{z}} = \frac{\partial }{\partial \overline{z}}\left( \frac{z+\overline{z}}{2} \right) = \frac{1}{2}
\quad\text{e}\quad
\frac{\partial y}{\partial \overline{z}} = \frac{\partial }{\partial \overline{z}}\left( \frac{z-\overline{z}}{2i} \right)=-\frac{1}{2i}
\]
Usando estas duas relações em \eqref{eq-aux1-CR-z-zbarra} ficamos com
\[
\frac{\partial f}{\partial \overline{z}}
=
\frac{1}{2}
\frac{\partial u}{\partial x}
-
\frac{1}{2i}
\frac{\partial u}{\partial y}
+i
\left(
\frac{1}{2}
\frac{\partial v}{\partial x}
-
\frac{1}{2i}
\frac{\partial v}{\partial y}
\right)
=
\frac{1}{2} \left( \frac{\partial u}{\partial x}-\frac{\partial v}{\partial y} \right)
+
\frac{i}{2} \left( \frac{\partial u}{\partial y}+\frac{\partial v}{\partial x} \right).
\]
Assim é imediato verificar que as Condições de Cauchy-Riemann são válidas se, e somente se,
\[
\frac{\partial f}{\partial \overline{z}} = 0
\]
o que é o mesmo que dizer (no caso em que as todas derivadas parciais de $u$ e $v$ são contínuas) 
que se $f$ tem derivada complexa se, e somente se, ela não depende 
da ``variável'' $\overline{z}$.

\bigskip 

Podemos refazer a discussão acima de maneira completamente rigorosa, considerando a linguagem de 
operadores lineares agindo em determinados espaços vetoriais complexos 
de funções como, por exemplo, agindo em  
\[
C^1(A,\mathbb{C}) 
=
\left\{ 
	f:A\to\mathbb{C}:
		\begin{array}{c}
		f(x+iy)= u(x,y)+iv(x,y) 
		\ \text{e}
		\\[0.2cm]
		\displaystyle\frac{\partial u}{\partial x}, \frac{\partial u}{\partial y} , \frac{\partial v}{\partial x} \ \text{e} \ 
		\frac{\partial v}{\partial y} \ \text{são contínuas em}\ A
		\end{array}
\right\}.  
\]
Ao leitor não acostumado com este tipo de linguagem, observamos que $C^1(A,\mathbb{C}) $ 
é um conjunto, onde cada um dos seus elementos é uma função $f:A\to\mathbb{C}$
como definida acima.

Vamos definir então operadores lineares, conhecidos também como ``operadores de derivação'',
como sendo aplicações lineares que pegam uma função $f:A\to\mathbb{C}$ e levam em outra 
função. No nosso caso eles são:
\[
\frac{\partial }{\partial z}: C^1(A,\mathbb{C}) \to C(A,\mathbb{C})
\qquad\text{e}\qquad 
\frac{\partial }{\partial \overline{z}}: C^1(A,\mathbb{C}) \to (C(A),\mathbb{C}),
\]
onde $C(A,\mathbb{C})$ é o espaço vetorial sobre $\mathbb{C}$ das funções complexas 
contínuas definidas em $A$ e tomando valores em  $\mathbb{C}$. 
O primeiro deste operadores leva uma função $f\in C^1(A,\mathbb{C})$ 
em uma outra função $\frac{\partial}{\partial z}f$
\[
C^1(A,\mathbb{C}) \ni f\quad \longmapsto\quad \frac{\partial}{\partial z}f \in C(A,\mathbb{C})
\]
que é definida para cada $x+iy\in A$ pela seguinte expressão 
\begin{align*}
\frac{\partial}{\partial z}f(x+iy) 
&\equiv
\frac{1}{2}
\left( 
\frac{\partial}{\partial x} +\frac{1}{i}\frac{\partial  }{\partial y}\right)(u(x,y)+iv(x,y)) 
\\
&\equiv 
\frac{1}{2}
	\left( 
		\frac{\partial}{\partial x}u(x,y)
		+i\frac{\partial}{\partial x}v(x,y) 
		+\frac{1}{i}\frac{\partial  }{\partial y}u(x,y) 
		+\frac{\partial  }{\partial y}v(x,y) 
	\right) 
\\
&=
\frac{1}{2}
\left( 
\frac{\partial}{\partial x}u(x,y) +\frac{\partial  }{\partial y}v(x,y) 
+i
	\left( 
		\frac{\partial}{\partial x}v(x,y) 
		-\frac{\partial  }{\partial y}u(x,y) 
	\right)
\right),
\end{align*}
onde $\partial/\partial x$ e $\partial/\partial y$, como de costume, denotam as derivadas parciais
de funções de $\mathbb{R}^2$ para $\mathbb{R}$,
com respeito a $x$ e $y$, respectivamente. 

O segundo operador  é definido de maneira análoga e para ele temos a seguinte expressão
\begin{align*}
	\frac{\partial}{\partial \overline{z}}f(x+iy) 
	&\equiv
	\frac{1}{2}
	\left( 
	\frac{\partial}{\partial x} -\frac{1}{i}\frac{\partial  }{\partial y}\right)(u(x,y)+iv(x,y)) 
	\\
	&=
	\frac{1}{2}
	\left( 
	\frac{\partial}{\partial x}u(x,y) -\frac{\partial  }{\partial y}v(x,y) 
	+i
	\left( 
	\frac{\partial}{\partial x}v(x,y) 
	+\frac{\partial  }{\partial y}u(x,y) 
	\right)
	\right).
\end{align*}



De forma mais resumida podemos pensar nestes operadores formalmente como 
\[
\frac{\partial}{\partial z}  \equiv  \frac{1}{2}\left( \frac{\partial}{\partial x}+ \frac{1}{i}\frac{\partial}{\partial y} \right)
\qquad\text{e}\qquad
\frac{\partial}{\partial \overline{z}}  \equiv  \frac{1}{2}\left( \frac{\partial}{\partial x}-\frac{1}{i}\frac{\partial  }{\partial y} \right),
\]
respectivamente.

\begin{proposicao}[Versão Funcional das Condições de Cauchy-Riemann]
\label{prop-functional-CR}
Sejam $A\subset \mathbb{C}$ um conjunto aberto e $f\in C^1(A,\mathbb{C})$. 
Então $f$ é derivável no sentido complexo no ponto $z=x+iy\in A$
se, e somente se, 
\[
\frac{\partial}{\partial \overline{z}}f(x+iy) = 0.
\]
\end{proposicao}
\begin{proof}
Vamos supor inicialmente que $f$ é derivável no sentido complexo em $z\in A$.
Com visto acima, para toda $f\in C^1(A,\mathbb{C})$ temos
\begin{align*}
	\frac{\partial}{\partial \overline{z}}f(x+iy) 
=
	\frac{1}{2}
	\left( 
	\frac{\partial}{\partial x}u(x,y) -\frac{\partial  }{\partial y}v(x,y) 
	+i
	\left( 
	\frac{\partial}{\partial x}v(x,y) 
	+\frac{\partial  }{\partial y}u(x,y) 
	\right)
	\right).
\end{align*}
Já que $f$ é derivável em $z\in A$ segue da Proposição \ref{prop-cond-Cauchy-Riemann}
que as Condições de Cauchy-Riemann são válidas e portanto a parte real e imaginárias da igualdade
acima são nulas, ou seja 
\[
	\frac{\partial}{\partial \overline{z}}f(x+iy)  = 0.
\]

Reciprocamente, se $\frac{\partial}{\partial \overline{z}}f(x+iy) = 0$, então as Condições de Cauchy-Riemann
são satisfeitas. Como estamos assumindo que $f\in C^1(A,\mathbb{C})$ sabemos que as derivadas parciais
de $u$ e $v$ são contínuas em qualquer ponto $z\in A$. Assim podemos aplicar a Proposição \ref{prop-reciproca-CR}
para garantir que $f$ tem derivada no sentido complexo em $z$.
\end{proof}

\bigskip 


De maneira análoga para todo $r=2,3,4,\ldots$ podemos definir o espaço $C^r(A,\mathbb{C})$ 
como sendo o espaço das funções complexas definidas em $A$ cujas as funções coordenadas
$u$ e $v$ têm derivadas parciais de todas as ordens entre 1 e $r$, contínuas em $A$. Note que 
se $r\geqslant 2$ então $C^r(A,\mathbb{C})$ é um subespaço de $C^1(A,\mathbb{R})$.
Assim podemos considerar que nossos operadores lineares $\partial /\partial z$ e $\partial/\partial\overline{z}$
como operadores lineares definidos em $C^r(A,\mathbb{C})$. Para $r\geqslant 2$ faz sentido aplicar 
este operadores sucessivamente (no máximo duas vezes). 
Por exemplo, para $f\in C^2(A,\mathbb{R})$ está bem definida a seguinte 
função
\[
\frac{\partial}{\partial z} \frac{\partial}{\partial \overline{z}} f(x+iy).
\] 

Pelo Teorema de Clairaut-Schwarz (visto no Cálculo 2) 
sabemos que se as funções $u:A\subset\mathbb{R}^2\to\mathbb{R}$ e $v:A\subset\mathbb{R}^2\to\mathbb{R}$ 
são de classe $C^2$ em um aberto $A\subset\mathbb{R}^2$
então para qualquer ponto $(x,y)\in A$ temos
\[
\frac{\partial^2}{\partial x\partial y}u(x,y) = \frac{\partial^2}{\partial y\partial x}u(x,y)
\qquad\text{e}\qquad 
\frac{\partial^2}{\partial x\partial y}v(x,y) = \frac{\partial^2}{\partial y\partial x}v(x,y).
\]
Desta maneira, se consideramos o operador $\frac{\partial}{\partial z} \frac{\partial}{\partial \overline{z}}$ 
agindo em $C^2(A,\mathbb{R})$ podemos usar o  Teorema de Clairaut-Schwarz  para verificar que:
\begin{align*}
\frac{\partial}{\partial z} \frac{\partial}{\partial \overline{z}}
&=
\frac{1}{2}\left( \frac{\partial}{\partial x}+ \frac{1}{i}\frac{\partial}{\partial y} \right)
\frac{1}{2}\left( \frac{\partial}{\partial x}-\frac{1}{i}\frac{\partial  }{\partial y} \right)
\\
&=
\frac{1}{4}
\left( 
	\frac{\partial}{\partial x} \frac{\partial}{\partial x} 
	-
	\frac{1}{i}\frac{\partial}{\partial x}\frac{\partial  }{\partial y}
	+
	\frac{1}{i}\frac{\partial}{\partial y}\frac{\partial}{\partial x}
	-
	\frac{1}{i^2}\frac{\partial}{\partial y} \frac{\partial  }{\partial y}
\right)
\\
&=
\frac{1}{4}
\left( 
\frac{\partial}{\partial x} \frac{\partial}{\partial x} 
-
\frac{1}{i}\frac{\partial}{\partial x}\frac{\partial  }{\partial y}
+
\frac{1}{i}\frac{\partial}{\partial x}\frac{\partial}{\partial y}
+
\frac{\partial}{\partial y} \frac{\partial  }{\partial y}
\right)
\\
&=
\frac{1}{4}\left( \frac{\partial^2}{\partial x\partial x} + \frac{\partial^2}{\partial y\partial y}   \right)
\\
&=
\frac{1}{4} \Delta,
\end{align*}
onde $\Delta$ é o operador Laplaciano. Este fatos demostram que para toda $f\in C^2(A,\mathbb{C})$
com $f=u+iv$  temos
\begin{align}\label{eq-aux-CR-harmonica}
\frac{\partial}{\partial z} \frac{\partial}{\partial \overline{z}} f(x+iy)
=
\frac{1}{4}(\Delta u(x,y)+ i \Delta v(x,y)).
\end{align}

Lembramos que uma função $u:A\subset\mathbb{R}^2\to\mathbb{R}$ é dita uma {\bf função harmônica} em $A$,
\index{Função!harmônica}
se para todo $(x,y)\in A$ temos que 
\[
\Delta u(x,y) = 0,\quad \text{equivalentemente}\quad  \frac{\partial^2}{\partial x\partial x}u(x,y) + \frac{\partial^2}{\partial y\partial y}u(x,y)=0. 
\]

As funções harmônicas são muito presentes em várias das aplicações mais interessantes do Cálculo Diferencial e Integral
em Física, Engenharias, Biologia, Economia e em várias outras áreas; inclusive na Matemática pura.
Nosso próximo resultado estabelece uma relação forte entre funções harmônicas e funções deriváveis no sentido complexo.
\begin{lema}\label{lema-holomorfas-harmonicas}
Se uma função complexa $f:A\to\mathbb{C}$ definida em um aberto $A\subset\mathbb{C}$ e dada por
$f(z)=u(x,y)+iv(x,y)$  tem derivada, no sentido complexo, em todos os pontos de $A$ 
e as derivadas parciais de primeira e segunda ordem de $u$ e $v$ são contínuas em $A$, 
então a partes real e imaginárias de $f$ são funções harmônicas em $A$, isto é,  $u$ e $v$ são funções harmônicas em $A$.
\end{lema}

\begin{proof}
Já que $f$ tem derivada complexa em todo ponto de $A$ segue da  Proposição \ref{prop-functional-CR} 	
que 
\[
\frac{\partial}{\partial \overline{z}}f=0
\quad \Longrightarrow\quad 
\frac{\partial }{\partial z}\frac{\partial}{\partial \overline{z}}f=0.
\]
Deste fato e da identidade \eqref{eq-aux-CR-harmonica} concluímos que 
\[
\frac{1}{4}(\Delta u(x,y)+ i \Delta v(x,y))
=
\frac{\partial}{\partial z} \frac{\partial}{\partial \overline{z}} f(x+iy)
=
0
\]
e consequentemente que ambas $u$ e $v$ são funções harmônicas em $A$.
\end{proof}


\section{Funções Holomorfas}

Podemos finalmente definir o objeto central 
destas notas que é o conceito de função holomorfa.
Estas funções são conhecidas também na 
literatura (mais antiga) como funções regulares
ou diferenciáveis no sentido complexo. 
A última nomenclatura é bastante mais natural, 
dada que estas funções serão definidas em termos de limites de quocientes de Newton,
exatamente como ocorre no caso de funções da reta na reta. Mas apesar da semelhança,
uma função holomorfa de uma variável complexa irá satisfazer propriedades muito mais fortes
do que  uma função diferenciável da reta na reta. Por exemplo, iremos mostrar que uma função 
holomorfa é infinitamente derivável no sentido complexo! Este fato contrasta absurdamente
com o que pode ocorrer para funções reais. Pois uma função de $\mathbb{R}$ em $\mathbb{R}$  
pode ter uma derivada e não possuir uma segunda derivada. É simples imaginar diversos 
exemplos de funções reais satisfazendo esta propriedade. Na verdade, um fato muito mais 
impressionante é válido: uma função holomorfa é sempre analítica, no sentido que ela admite 
uma expansão em série de potências convergente em bolas em volta de cada um dos pontos de seu 
domínio (séries de potências complexas serão discutidas nas seções seguintes). Por 
esta razão, as funções holomorfas, muitas vezes, são chamadas de funções analíticas.
Esta propriedade é outra diferença marcante entre as funções que admitem derivadas no sentido complexo e
real. Já que existem diversos exemplos de funções deriváveis no sentido real possuindo 
infinitas derivadas e que não admitem representações em séries de potências em todos os 
pontos de seu domínio.



\begin{definicao}[Função Holomorfa]
\label{def-func-holomorfa}
\index{Função!holomorfa}
Seja $A\subset\mathbb{C}$ aberto e $f:A\to\mathbb{C}$ uma função. Dizemos que 
$f$ é holomorfa em $A$ se $f'(z)$ existe para todo $z\in A$. 
\end{definicao}

Note que a Proposição \ref{prop-reciproca-CR} fornece uma poderosa ferramenta
para verificar que se uma função complexa é holomorfa em um determinado aberto $A$.
Além do mais, vale a pena mencionar que a Proposição \ref{prop-functional-CR}
caracteriza as funções holomorfas como aquelas que não dependem da ``variável'' $\overline{z}$.

Na verdade, existe um resultado muito mais forte do que os mencionados acima, 
devido a Loomann e Menchoff, 
fornecendo condições suficientes para que uma função 
complexa $f:A\to\mathbb{C}$ seja holomorfa. Este resultado é conhecido hoje 
em dia como Teorema de Loomann-Menchof e ele afirma que se $f$ é uma função contínua 
e existem as derivadas parciais de $u$ e $v$ e estas por sua vez satisfazem as condições de 
Cauchy-Riemann em todos os pontos de $A$ então $f$ é holomorfa.
Este teorema foi enunciado inicialmente pelo famoso matemático Paul Montel em \cite{Montel13} e uma primeira ``prova'' foi dada por 
Herman Loomann em \cite{Loomann23}. Escrevemos  prova entre aspas, porque havia um problema no argumento de Loomann, 
e este foi posteriormente reparado por Menchoff em um livro \cite{Menchoff36} em 1936 editado por Paul Montel. 
Há hoje dia versões mais simples destas provas até mesmo
em livros didáticos como o de Narasimha-Nievergelt \cite{MR1803086}. 
Neste livro os autores provam este teorema no Capítulo 1, em uma seção que ocupa 6 páginas. 
Aos leitores realmente interessados nesta versão recomendamos começar a leitura desta prova
lendo o artigo de Gray e Morris de 1978, publicado na famosa American Mathematical Monthly \cite{MR470179}.
Neste artigo os autores abordam os aspectos históricos deste teorema, 
e também comentam sobre os pontos obscuros de vários trabalhos que abordaram este problema.

\bigskip


Um exemplo muito importante de função holomorfa é uma função polinomial de grau $n$,
isto é, $p:\mathbb{C}\to\mathbb{C}$ dada por 
\[
p(z)= a_0+a_1z+a_2z^2+\ldots +a_nz^n,
\]
onde $a_0,a_1,\ldots,a_n$ são números complexos fixados chamados de coeficientes de $p$.
Para verificar que esta função polinomial é de fato holomorfa em $\mathbb{C}$ basta mostrar
que a função $f:\mathbb{C}\to\mathbb{C}$ dada por $f(z)=z^m$ é holomorfa para qualquer $m\in \mathbb{N}$
e em seguida, aplicar sucessivamente a Proposição \ref{prop-derivada}.
Para isto observe que a seguinte identidade algébrica é satisfeita
\[
\frac{z^m-z_0^m}{z-z_0}
=
z^{m-1}+z^{m-2}z_0+z^{m-3}z_0^2+\ldots+z^{2}z_0^{m-2}+zz_0^{m-1}.
\] 
Já que o lado direito da igualdade acima é uma soma de funções contínuas em todo $\mathbb{C}$ 
podemos tomar o limite, quando $z\to z_0$, na igualdade acima e portanto temos que $f$ é derivável e além do mais
\begin{align*}
f'(z_0) 
&= 
\lim_{z\to z_0} \frac{z^m-z_0^m}{z-z_0}
\\
&=
\lim_{z\to z_0} 
z^{m-1}+z^{m-2}z_0+z^{m-3}z_0^2+\ldots+z^{2}z_0^{m-2}+zz_0^{m-1}
\\
&=
mz_{0}^{m-1}.
\end{align*}

Daí segue que qualquer que seja $z\in \mathbb{C}$ temos $f'(z)=mz^{m-1}$ logo
\[
p'(z) = a_1+2a_2z+3a_3z^2+\ldots +na_nz^{n-1}.
\]

Podemos usar este fato e novamente a Proposição \ref{prop-derivada} para verificar que 
uma função racional é holomorfa no conjunto onde seu denominador não se anula.
Mais precisamente, sejam $p:\mathbb{C}\to\mathbb{C}$ e $q:\mathbb{C}\to\mathbb{C}$ 
funções polinomiais. Seja $A$ o conjunto de todos os pontos do plano complexo 
exceto os zeros do polinômio $q$, isto é, 
$A=\mathbb{C}\setminus \{z\in \mathbb{C}: q(z)=0\}$. Vamos ver mais a frente
que o conjunto de zeros de qualquer polinômio de uma variável complexa com coeficientes
complexos é finito. Usando este fato podemos verificar que $A$ é um aberto do plano complexo. Agora 
considere a função racional $f:A\to\mathbb{C}$ definida por 
\[
f(z) =\frac{p(z)}{q(z)}.
\]
Pelo exemplo anterior e pela Proposição \ref{prop-derivada} temos que $f$ é holomorfa em $A$.

Observamos também que a regra da cadeia fornece uma ferramenta para construir novas funções 
holomorfas a partir de composições de funções holomorfas já conhecidas.


\begin{definicao}[Função Inteira]
\label{def-func-inteira}
\index{Função!inteira}
Uma função complexa $f:\mathbb{C}\to\mathbb{C}$ é dita uma função inteira se $f$ é holomorfa em todo $\mathbb{C}$.	
\end{definicao}


\begin{definicao}[Ponto Singular]
\label{def-singularidade}
\index{Ponto!singular}\index{Singularidade}
Um ponto singular de uma função complexa $f$ é um ponto $z_0$ do plano complexo 
tal que existe um disco $D(z_0,r)$ no qual $f$ é holomorfa, exceto em $z_0$. 	
\end{definicao}


Por exemplo, as funções constantes e polinômios são funções inteiras.
Já a função $f:\mathbb{C}\setminus\{0\}\to \mathbb{C}$ dada por $f(z)=1/z$ 
tem um ponto singular (ou uma singularidade isolada) na origem. Um exemplo de uma função 
possuindo dois pontos singulares é dado pela função racional
\[
f(z) = \frac{2}{(z+1)^2(z-2i)^3},
\]
Neste exemplo $z=-1$ e $z=2i$ são pontos singulares, ou singularidades isoladas da função $f$.

Por outro lado, a função $f(z)=\overline{z}$ não possui pontos singulares, pois esta função 
não é derivável em ponto algum.


\section{A Exponencial Complexa}

Nesta seção vamos apresentar a função exponencial complexa bem como várias de suas propriedades
elementares. Inicialmente vamos adotar uma notação especial para esta função $\exp(z)$.
Após estabelecer várias semelhanças desta com a função exponencial real $x\longmapsto e^x$,
com quem o leitor já está bem acostumado, passamos a usar, quando for conveniente 
a notação $e^z$. 


\begin{definicao}[Função Exponencial]
\label{def-func-exp}
\index{Função!exponencial complexa}
A função exponencial complexa será denotada por 
$\exp:\mathbb{C}\to\mathbb{C}$ e definida para cada ponto
$z=x+iy\in\mathbb{C}$ pela expressão 
\[
\exp(z) = e^{x}(\cos y+i\sen y).
\]	
\end{definicao} 

Para facilitar vamos nos referir a função exponencial complexa $z\longmapsto \exp(z)$ 
simplesmente por função exponencial. Descrita em coordenadas a função exponencial,
calculada em $z=x+iy$,
tem a seguinte forma $\exp(z)=u(x,y)+iv(x,y)$, onde
\[
u(x,y)= e^{x}\cos y   \qquad{e}\qquad v(x,y)=e^x\sen y. 
\]

Note que sua descrição é coordenadas é suficientemente simples 
de forma que podemos aplicar imediatamente os
resultados da seção anterior como, por exemplo, as Condições de 
Cauchy-Riemann e a Proposição \ref{prop-reciproca-CR} para
concluir que $z\longmapsto \exp(z)$ define uma função inteira. De fato,
já que para todo $x+iy\in\mathbb{C}$
\begin{align*}
&\frac{\partial}{\partial x}u(x,y) = e^{x}\cos y,
\qquad 
\frac{\partial}{\partial y}u(x,y) = -e^{x}\sen y,
\\[0.4cm]
&\frac{\partial}{\partial x}v(x,y) = e^{x}\sen y,
\qquad 
\frac{\partial}{\partial y}v(x,y) = e^{x}\cos y,
\end{align*}
as Condições de Cauchy-Riemann são válidas em todos os pontos do plano complexo.
Além do mais, como as funções $u$ e $v$ possuem derivadas parciais contínuas segue 
da Proposição \ref{prop-reciproca-CR} que $z\longmapsto \exp(z)$ é derivável em 
cada $z\in\mathbb{C}$ e portanto inteira.
Além do mais a Proposição \ref{prop-reciproca-CR} também 
fornece uma fórmula para derivada complexa em termos das funções coordenadas que é a seguinte:
\begin{align*}
\exp'(z) 
&= 
\frac{\partial}{\partial x}u(x,y) +i \frac{\partial}{\partial x}v(x,y) 
\\[0.2cm]
&=
e^{x}\cos y+i e^{x}\cos y 
\\[0.2cm]
&= e^x(\cos y+i\sen y)
\\[0.2cm]
&=
\exp(z).
\end{align*}

Note também que se $z$ é real, isto é, $z=x+i0$ então $\exp(x)=e^{x}$.
Como no caso caso da exponencial real a imagem de qualquer ponto pela 
função exponencial complexa é sempre diferente de zero. Uma maneira de 
ver isto é verificando o módulo de $\exp(z)$ é não-nulo. De fato,
para qualquer que seja $z=x+iy\in\mathbb{C}$ temos 
\begin{align*}
|\exp(z)| 
&= 
|e^{x}(\cos y+i\sen y)| 
\\[0.2cm]
&= |e^x|\cdot|\cos y +i\sen y|
\\[0.2cm]
&=
|e^x| \sqrt{\cos^2y+\sen^2y}
\\[0.2cm]
&= e^{x} > 0.
\end{align*}


A exponencial complexa possui uma propriedade análoga à uma das propriedades
fundamentais da exponencial real que é a identidade $e^{x}e^{y}=e^{x+y}$, válida
para todo par $x,y\in\mathbb{R}$. 
Dados quaisquer números complexos $z_1=x_1+iy_1$ e $z_2=x_2+iy_2$ temos que 
\[
\exp(z_1)\cdot \exp(z_2) = \exp(z_1+z_2).
\] 
Esta identidade é consequência direta da definição da função exponencial complexa e 
das identidades trigonométricas fundamentais, como mostrado abaixo:
\begin{align*}
\exp(z_1)&\cdot \exp(z_2)
=
e^{x_1}(\cos y_1+i\sen y_1)
\, 
e^{x_2}(\cos y_2+i\sen y_2)
\\[0.2cm]
&=
e^{x_1}e^{x_2}\, (\cos y_1+i\sen y_1)(\cos y_2+i\sen y_2)
\\[0.2cm]
&=
e^{x_1+x_2}(\cos y_1\cos y_2-\sen y_1\sen y_2 +i [\sen y_1\cos y_2+\cos y_1\sen y_2] )
\\[0.2cm]
&=
e^{x_1+x_2}(\cos(y_1+y_2)+i\sen(y_1+y_2))
\\[0.2cm]
&=
\exp( (x_1+x_2)+i(y_1+y_2) )
\\[0.2cm]
&=
\exp(z_1+z_2).
\end{align*}

A identidade $e^{-x}= 1/e^{x}$ também possui uma contrapartida na exponencial
complexa. Para provar esta generalização só precisamos 
usar a definição da exponencial complexa, as propriedades algébricas básicas
dos números complexos e lembrar que o cosseno é uma função
par, $\cos(-x)=\cos(x)$, e o seno é uma função ímpar, $\sen(-y)=-\sen(y)$.
Em seguida, proceder como segue
\begin{align*}
\frac{1}{\exp(z)}
&=
\frac{1}{e^{x}(\cos y+i\sen y)}
\\[0.3cm]
&=
\frac{1}{e^{x}(\cos y+i\sen y)}\cdot \frac{\cos y-i\sen y}{\cos y-i \sen y}
\\[0.3cm]
&=
\frac{e^{-x}(\cos y-i\sen y)}{\cos^2 y+\sen^2 y}
\\[0.3cm]
&=
e^{-x}(\cos(-y)+i\sen(-y))
\\[0.3cm]
&=
\exp(-x-iy)
\\[0.3cm]
&=
\exp(-z).
\end{align*}

Como consequência das identidades acima temos para quaisquer $z_1,z_2\in\mathbb{C}$
que 
\begin{align}\label{eq-razao-exps}
\frac{\exp(z_1)}{\exp(z_2)} = \exp(z_1)\exp(-z_2)=\exp(z_1-z_2).
\end{align}


Usando as identidades estabelecidas 
acima temos para quaisquer $z\in\mathbb{C}$ e $n\in\mathbb{Z}$ que 
\[
\exp(z)^n = \exp(nz).
\]
Para verificar a validade deste fato basta proceder uma indução formal em $n$.

\bigskip 

Agora vamos mostrar algumas diferenças entre a exponencial real e complexa. 
Em $\mathbb{R}$ temos bem definida a função $x\longmapsto e^{\frac{1}{n}x}$
para qualquer $n\in\mathbb{Z}$. O número $e^{\frac{1}{n}x}$ é caracterizado 
como sendo o único número real tal que o produto dele por ele mesmo $n$ vezes é
igual $e^x$. Em outras palavras, $y=e^{\frac{1}{n}x}$ é a única solução real 
da equção $y^n=e^x$. Considerando que a única solução real 
desta equação seja definição de $e^{\frac{1}{n}x}$, podemos nos perguntar o 
que aconteceria se considerarmos sua generalização natural para números complexos.
Isto é, $w^n=\exp(z)$. Para facilitar, consideramos separadamente os casos 
em que $n$ é positivo e negativo. Para $n$ inteiro positivo  
esta equação é um caso particular da equação $w^n=w_0$,
que como vimos, possui exatamente $n$ soluções distintas se $w_0\neq 0$, 
que é o caso de $w_0=\exp(z)\neq 0$. 

Desta forma não é possível pensar na relação $z\longmapsto \exp(z)^{\frac{1}{n}}$
como uma função. Pois, para cada $z$ podemos associar $n$ distintas 
raízes $n$-ésimas de $\exp(z)$ uma para cada solução da equação $w^n = \exp(z)$. 
Na literatura é comum entretanto se referir a relação $z\longmapsto \exp(z)^{\frac{1}{n}}$
como uma {\textit função multiforme}. 

Usando a fórmula deduzida em \eqref{eq-sol-ZN=w}, juntamente com a definição 
da exponencial complexa, temos para cada 
$0\leqslant j\leqslant n-1$ que o número complexo
\begin{align*}
e^{\frac{x}{n}} 
	\left( 
		\cos\Big( \frac{y+2\pi\, j}{n} \Big)  
		+i
		\sen\Big( \frac{y+2\pi\, j}{n} \Big)  
	\right)
=
\exp\Big( \frac{z+2\pi\, j}{n}  \Big)
\end{align*}
é uma raíz $n$-ésima de $\exp(z)$. As vezes, abusamos da notação 
e escrevemos 
\[
\exp(z)^{\frac{1}{n}} = \exp\Big( \frac{z+2\pi\, j}{n}  \Big),
\qquad \forall\, j\in\{0,1,\ldots,n-1\}.
\]

Discussão semelhante é válida para potencias racionais da exponencial, isto é 
$\exp(z)^{\frac{m}{n}}$, onde $m\in\mathbb{Z}$ e $n\in \mathbb{Z}\setminus\{0\}$.
Como no caso das raízes $n$-ésimas da função exponencial complexa, a relação 
$z\longmapsto \exp(z)^{\frac{m}{n}}$ não irá, em geral, definir uma função e 
sim um função multiforme que vamos denotar por 
\[
\exp(z)^{\frac{m}{n}} = \exp\Big( \frac{m}{n}(z+2\pi\, j)  \Big),
\qquad \forall\, j\in\{0,1,\ldots,n-1\}.
\]


Esta duas novidades são consequências da periodicidade da função 
exponencial complexa, isto é, 
\[
\exp(z+2i\pi) = e^{x}(\cos(y+2\pi)+i\sen(u+2\pi))=e^{x}(\cos y+i\sen y) = \exp(z).
\]

Desta forma a função exponencial mapeia retas paralelas ao eixo imaginário em 
círculos centrados na origem e retas paralelas ao eixo real são enviadas em
semi-retas saindo da origem. 


\bigskip 
Usando novamente que as funções trigonométricas cosseno e seno são 
funções par e ímpar, respectivamente, podemos verificar que o conjugado 
da exponencial é simplesmente a exponencial do conjugado, mais precisamente 
para todo $z\in\mathbb{C}$ temos:
\begin{align*}
\overline{\exp(z)}
&=
\overline{e^{x}(\cos y+i\sen y)}
\\
&=
e^{x}(\cos y-i\sen y)
=
e^{x}(\cos(-y)+i\sen(-y))
=
\exp(\overline{z}).
\end{align*}



Por último, mencionamos mais uma importante propriedade que a
exponencial é sobrejetora em $\mathbb{C}^{*}$, isto é,
$\exp:\mathbb{C}\to\mathbb{C}^{*}$ é uma aplicação sobrejetora.
Para verificar que $\exp(\mathbb{C})=\mathbb{C}^{*}$, escolha
um ponto arbitrário $w=a+ib\in \mathbb{C}^{*}$.
Considere a equação 
\[
e^{x}(\cos y+i\sen y) = \exp(z) = w = a+ib
\]
Como $w\neq 0$ podemos representá-lo em coordenadas polares como segue
$w=|w|(\cos \theta+i\sen\theta)$. Logo
\[
e^{x}(\cos y+i\sen y) = |w|(\cos \theta+i\sen\theta)
\]
e portanto tomando $x=\ln |w|$ e $y=\theta +2k\pi$, onde $k\in \mathbb{Z}$
temos que $\exp(x+iy)=w$,
mostrando que aplicação exponencial complexa é sobrejetiva em $\mathbb{C}^{*}$.



Note que se $z\in\mathbb{C}$ é imaginário puro, isto é, $z=iy$ 
para algum $y\in\mathbb{C}$ temos que $\exp(z)=\exp(iy)=\cos y+i\sen y$.
Esta observa mostra que podemos escrever um número complexo em sua forma
polar da seguinte maneira $w=r\exp(i\theta)$. 

Pelo fato da função exponencial complexa $\exp$ poder 
ser vista como uma extensão da função exponencial
real $x\longmapsto e^{x}$ e ter muitas propriedades semelhantes 
vamos adotar também no caso complexo a notação $e^z$. 
Observe que as propriedades demostradas acima permitem verificar que, por exemplo,
que se $z=x+iy$ então $e^z=e^xe^y$.


\section{Funções Trigonométricas e Trigonométricas Hiperbólicas Complexas}

Definimos as funções seno e cosseno complexos como segue 
\index{Função!seno complexo}\index{Função!cosseno complexo}
\[
\cos z = \frac{1}{2}\big( e^{iz}+e^{-iz} \big)
\qquad\text{e}\qquad 
\sen z = \frac{1}{2i}\big( e^{iz}-e^{-iz} \big)
\]

Observe que similarmente ao caso real, as derivadas destas funções são dadas por
\[
\frac{d}{dz}\cos z = -\sen z 
\qquad\text{e}\qquad
\frac{d}{dz}\sen z = \cos z,
\quad \forall z\in \mathbb{C}.
\]
Em particular, estas funções são inteiras. 

As funções seno e cosseno complexas definidas acima, 
podem ser vistas como extensões das funções trigonométricas 
reais. Para ver isto vamos determinar suas representações em coordenadas
\begin{align*}
\cos z 
&= 
\frac{1}{2}\big( e^{iz}+e^{-iz} \big)
\\
&=
\frac{1}{2}\big( e^{-y+ix}+e^{y-ix} \big)
\\
&=
\frac{1}{2}\big( e^{-y}(\cos x+i\sen x)+ e^{y}(\cos(-x)+i\sen(-x) \big)
\\
&=
\frac{1}{2}\big( (e^{y}+e^{-y}) \cos x +i\, (e^{y}-e^{-y}) \sen x  \big)
\\
&=
\cosh y \cos x +i \senh y \sen x.
\end{align*}
Agora fica claro que se $z=x+i0$ então $\cos z = \cos x$.  

Analogamente, podemos mostrar que se $z=x+iy$ então 
\[
\sen z =  \cosh y \sen x -i \senh y \cos x.
\]

Observe que as funções seno e cosseno complexos \textbf{não} são funções limitadas
em todo plano complexo.

Outra maneira interessante de definir estas funções é por meio dos seguintes
problemas de valor inicial (PVI) complexos

\[
z\longmapsto \sen z
\qquad \text{é a única solução do PVI}
\qquad 
\begin{cases}
f''(z)=-f(z);
\\[0.2cm]
f(0)=0 \ \text{e}\ f'(0)=1.
\end{cases}
\]
e
\[
z\longmapsto \cos z
\qquad \text{é a única solução do PVI}
\qquad 
\begin{cases}
f''(z)=-f(z);
\\[0.2cm]
f(0)=1 \ \text{e}\ f'(0)=0.
\end{cases}
\]
Vamos voltar a esta abordagem mais a frente depois de estabelecer 
analiticidade e a representação integral destas funções. \\


As funções trigonométricas hiperbólicas são definidas como segue
\index{Função!seno hiperbólico complexo}
\index{Função!cosseno hiperbólico complexo}
\[
\cosh z= \frac{e^z+e^{-z}}{2}
\qquad\text{e}\qquad
\senh z = \frac{e^{z}-e^{-z}}{2}.
\]

Estas duas funções possuem generalizações muito mais naturais do que de
as funções trigonométricas. Do ponto de vista de suas relações em termos 
de derivadas temos 
\[
\frac{d}{dz}\cosh z = \senh z 
\qquad \text{e}\qquad 
\frac{d}{dz}\senh z = \cosh z.
\]
Analogamente estas funções são soluções dos seguintes problemas de valor
inicial 
\[
\begin{cases}
f''(z)=f(z);
\\[0.2cm]
f(0)=0 \ \text{e}\ f'(0)=1,
\end{cases}
\qquad \text{e}\qquad 
\begin{cases}
f''(z)=f(z);
\\[0.2cm]
f(0)=0 \ \text{e}\ f'(0)=1,
\end{cases}
\]
respectivamente.

Do ponto de vistas da equações diferencias ou dos PVI's temos que
estas funções trigonométricas e as trigonométricas hiperbólicas 
são descritas de maneira totalmente similares em $\mathbb{R}$ e $\mathbb{C}$.
As diferenças entre elas, como por exemplo, a não limitação das funções seno e
cosseno complexos são devidas apenas à natureza dos domínios 
onde estas funções estão definidas bem como da derivada complexa.

As funções tagente, secante, cossecante e suas análogas hiperbólicas são
definidas de maneira análoga ao caso real.  



