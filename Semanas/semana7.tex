% !TeX spellcheck = pt_BR
%\chapter[Semana 7]{}
\chapter[Teoria de Cauchy]{Teoria de Cauchy:\\ Integração no plano complexo}
\chaptermark{}


\hfill%
\begin{minipage}{12cm}
	\begin{flushright}
		\rightskip=0.5cm
		\textit{``At the basis of the distance concept lies, for example, 
		the concept of a convergent point sequence and their defined limits,
		and one can, choosing these ideas as those fundamental to the point 
		set theory, eliminate the notions of distance ... Thirdly, 
		we can associate with each point of the set certain parts of the space
		called neighborhoods, and these can again be made building stones of the 
		theory with the elimination of the distance concept. Here the view
		of a set is in consideration of the association between elements and
		subsets.''}
		\\[0.1cm]
		\rightskip=0.5cm
		---F. Hausdorff, 1949
	\end{flushright}
\end{minipage}



\section{A Integral Complexa}

\begin{lema}
Sejam $U \subset \mathbb{C}$ um domínio e $f:U \to \mathbb{C}$ uma função holomorfa. Se $f'(z) = 0$ para
todo ponto $z\in U$, então $f$ é uma função constante.
\end{lema}

\begin{definicao}
\label{def:integral-complexa}
A integral da função $f$ ao longo do caminho $\gamma$ é
o número complexo
\begin{equation*}
    \int_{\gamma} f(z) dz = \int_a^b f(\gamma(t))\gamma'(t) dt,
\end{equation*}
sendo $\gamma:[a,b]\to\mathbb{C}$ um caminho suave e
$f:U \to \mathbb{C}$ função contínua com $U$ domínio.
\end{definicao}

\begin{definicao}
Sejam $f$ e $U$ como na Definição \ref{def:integral-complexa} e
$\gamma = \gamma_1*\gamma_2*\cdots*\gamma_n$ um caminho suave por partes em $U$.
A integral de $f(z)$ ao longo de $\gamma$ é o número complexo
\begin{equation*}
    \int_{\gamma} f(z)dz = \sum_{i=1}^n\int_{\gamma_i} f(z)dz.
\end{equation*}
\end{definicao}



\section[Primitivas e o Teorema Fundamental do Cálculo]
{Primitivas e o Teorema Fundamental\\ do Cálculo}

\begin{definicao}
\label{def:primitiva-complexa}
Seja $f:U \to \mathbb{C}$ uma função contínua com $U\subset\mathbb{C}$ domínio. Uma função $F:U\to\mathbb{C}$
é chamada primitiva de $f$ se $F$ é holomorfa em $U$ e $F'(z) = f(z), \forall z\in U$.
\end{definicao}

\begin{teorema}
Sejam $U\subset\mathbb{C}$ domínio, $f:U\to\mathbb{C}$ função contínua,
$F$ uma primitiva de $f$ em $U$ e $\gamma$ um caminho suave por partes em $U$ unindo $z_0$ a $z_1$. Então
\begin{equation*}
    \int_{\gamma} f(z)dz = F(z_1) - F(z_0).
\end{equation*}
Em particular, se o caminho é fechado então
\begin{equation*}
    \int_{\gamma} f(z)dz = 0.
\end{equation*}
\end{teorema}

\begin{proposicao}
Seja $\displaystyle{ f(z) = \sum_{n=0}^{\infty} a_n(z-z_0)^n }$
definida por uma série de potências com raio de convergência $R>0$.
Então a função
\begin{equation*}
    F(z) = \sum_{n=0}^{\infty} \frac{a_n}{n+1}(z-z_0)^{n+1}
\end{equation*}
é uma primitiva de $f$ e a série que a define converge para $|z-z_0| < R$.
\end{proposicao}

\begin{lema}[Lema Técnico]
Sejam $U\subset\mathbb{C}$ um domínio, $f:U\to\mathbb{C}$ uma função contínua e
$\gamma(t), a\leq t\leq b$ um caminho suave por partes em $U$, de comprimento $l(\gamma)$.
Seja $K\geq 0$ um número real tal que $|f(\gamma(t))| \leq K$ para todo $a\leq t\leq b$. Então
\begin{equation*}
    \left| \int_{\gamma}f(z)dz \right| \leq Kl(\gamma).
\end{equation*}
\end{lema}


\section[Os teoremas de Cauchy]{Os teoremas de Cauchy}
\begin{center}
Em construção...
\end{center}