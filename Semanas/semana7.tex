% !TeX spellcheck = pt_BR
%\chapter[Semana 7]{}
\chapter[Teoria de Cauchy]{Teoria de Cauchy:\\ Integração no plano complexo}
\chaptermark{}


\hfill%
\begin{minipage}{12cm}
	\begin{flushright}
		\rightskip=0.5cm
		\textit{``At the basis of the distance concept lies, for example, 
		the concept of a convergent point sequence and their defined limits,
		and one can, choosing these ideas as those fundamental to the point 
		set theory, eliminate the notions of distance ... Thirdly, 
		we can associate with each point of the set certain parts of the space
		called neighborhoods, and these can again be made building stones of the 
		theory with the elimination of the distance concept. Here the view
		of a set is in consideration of the association between elements and
		subsets.''}
		\\[0.1cm]
		\rightskip=0.5cm
		---F. Hausdorff, 1949
	\end{flushright}
\end{minipage}



\section{A Integral Complexa}

\begin{lema}
Sejam $U \subset \mathbb{C}$ um domínio e $f:U \to \mathbb{C}$ uma função holomorfa. Se $f'(z) = 0$ para
todo ponto $z\in U$, então $f$ é uma função constante.
\end{lema}

\begin{definicao}
\label{def:integral-complexa}
A integral da função $f$ ao longo do caminho $\gamma$ é
o número complexo
\begin{equation*}
    \int_{\gamma} f(z) dz = \int_a^b f(\gamma(t))\gamma'(t) dt,
\end{equation*}
sendo $\gamma:[a,b]\to\mathbb{C}$ um caminho suave e
$f:U \to \mathbb{C}$ função contínua com $U$ domínio.
\end{definicao}

\begin{definicao}
Sejam $f$ e $U$ como na Definição \ref{def:integral-complexa} e
$\gamma = \gamma_1*\gamma_2*\cdots*\gamma_n$ um caminho suave por partes em $U$.
A integral de $f(z)$ ao longo de $\gamma$ é o número complexo
\begin{equation*}
    \int_{\gamma} f(z)dz = \sum_{i=1}^n\int_{\gamma_i} f(z)dz.
\end{equation*}
\end{definicao}



\section[Primitivas e o Teorema Fundamental do Cálculo]
{Primitivas e o Teorema Fundamental\\ do Cálculo}

\begin{definicao}
\label{def:primitiva-complexa}
Seja $f:U \to \mathbb{C}$ uma função contínua com $U\subset\mathbb{C}$ domínio. Uma função $F:U\to\mathbb{C}$
é chamada primitiva de $f$ se $F$ é holomorfa em $U$ e $F'(z) = f(z), \forall z\in U$.
\end{definicao}

\begin{teorema}
Sejam $U\subset\mathbb{C}$ domínio, $f:U\to\mathbb{C}$ função contínua,
$F$ uma primitiva de $f$ em $U$ e $\gamma$ um caminho suave por partes em $U$ unindo $z_0$ a $z_1$. Então
\begin{equation*}
    \int_{\gamma} f(z)dz = F(z_1) - F(z_0).
\end{equation*}
Em particular, se o caminho é fechado então
\begin{equation*}
    \int_{\gamma} f(z)dz = 0.
\end{equation*}
\end{teorema}

\begin{proposicao}
Seja $\displaystyle{ f(z) = \sum_{n=0}^{\infty} a_n(z-z_0)^n }$
definida por uma série de potências com raio de convergência $R>0$.
Então a função
\begin{equation*}
    F(z) = \sum_{n=0}^{\infty} \frac{a_n}{n+1}(z-z_0)^{n+1}
\end{equation*}
é uma primitiva de $f$ e a série que a define converge para $|z-z_0| < R$.
\end{proposicao}

\begin{lema}[Lema Técnico]
\label{lema-tecnico}
Sejam $U\subset\mathbb{C}$ um domínio, $f:U\to\mathbb{C}$ uma função contínua e
$\gamma(t), a\leq t\leq b$ um caminho suave por partes em $U$, de comprimento $l(\gamma)$.
Seja $K\geq 0$ um número real tal que $|f(\gamma(t))| \leq K$ para todo $a\leq t\leq b$. Então
\begin{equation*}
    \left| \int_{\gamma}f(z)dz \right| \leq Kl(\gamma).
\end{equation*}
\end{lema}

\begin{teorema}
Seja $f:U\to \mathbb{C}$ uma função contínua definida no domínio $U\subset\mathbb{C}$. As seguintes afirmações são equivalentes:
\begin{enumerate}[(i)]
    \item $f$ tem uma primitiva em $U$
    \item $\int_{\gamma} f(z) dz = 0$ para qualquer caminho $\gamma$ fechado e suave por partes em $U$.
    \item $\int_{\lambda} f(z) dz$ só depende dos pontos inicial e final de qualquer caminho $\lambda$ suave por partes em $U$.
\end{enumerate}
\end{teorema}


\section[Os teoremas de Cauchy]{Os teoremas de Cauchy}

\begin{teorema}
\label{teo:cauchy-goursat}
Sejam $U$ um domínio em $\mathbb{C}$ e $f: U \to \mathbb{C}$ uma função holomorfa. Suponha que $\Delta \subset U$ é uma triângulo que limita
uma região inteiramente contida em $U$. Então
\begin{equation*}
    \int_{\Delta} f(z) dz = 0.
\end{equation*}
\end{teorema}


\begin{definicao}
Seja $U\subset\mathbb{C}$ um domínio. Dizemos que $U$ é estrelado se existe um ponto $z_0\in U$
tal que dado qualquer ponto $z\in U$, o segmento de reta $\overline{z_0z}$ está inteiramente contido em $U$. 
O ponto $z_0$ é chamado um centro do domínio $U$.
\end{definicao}


\begin{corolario}
Sejam $U\subset\mathbb{C}$ um domínio estrelado e $f:U\to\mathbb{C}$ uma
função holomorfa. Então $f$ admite uma primitiva em $U$.
\end{corolario}


\begin{corolario}[Teorema de Cauchy-Goursat bis.]
Sejam $U\subset\mathbb{C}$ um domínio estrelado e $f:U\to\mathbb{C}$ uma função holomorfa.
Se $\gamma$ é um caminho fechado suave por partes em $U$, então
\begin{equation*}
    \int_{\gamma} f(z) dz = 0.
\end{equation*}
\end{corolario}


\begin{teorema}[Fórmula Integral de Cauchy]
\label{teo:form-integral-cauchy}
Seja $f:U \to \mathbb{C}$ uma função holomorfa definida no domínio $U\subset\mathbb{C}$.
Sejam $\overline{D}(z_0, r_0)$ um disco fechado inteiramente contido em $U$ e $\Gamma$ sua fronteira, orientada compativelmente. 
Se $z$ é um ponto qualquer no interior de $\overline{D}(z_0, r_0)$ então
\begin{equation*}
    f(z) = \frac{1}{2\pi i}\int_{\Gamma} \frac{f(w)}{w-z}dw.
\end{equation*}
\end{teorema}


\begin{corolario}
Seja $f:U\to\mathbb{C}$ holomorfa com $U$ domínio. Então $f$ tem derivadas de todas as ordens em todos os pontos de $U$ e
\begin{equation*}
    f^{(n)}(z) = \frac{n!}{2\pi i}\int_{\gamma} \frac{f(w)}{(w-z)^{n+1}}dw, \forall z\in U,
\end{equation*}
onde $\gamma$ é qualquer circunferência centrada em $z$, percorrida no sentido anti-horário e limitando um disco fechado contido em $U$.
\end{corolario}


\begin{corolario}[Estimativas de Cauchy]
Seja $f$ uma função holomorfa definida no disco $D(z_0, R)$ e suponha que
$|f|\leq K$ em $D(z_0, R)$. Então
\begin{equation*}
    |f^{(n)}(z_0)| \leq \frac{n!K}{R^n}.
\end{equation*}
\end{corolario}


\begin{corolario}[Teorema de Liouville]
Seja $f$ função inteira. Se existe $K\geq 0$ tal que $|f(z)|\leq K$ então $f$ é uma função constante.
\end{corolario}


\begin{lema}
Seja $D(a,r), r>0$, um disco e $f: D(a,r)\to\mathbb{C}$ uma função holomorfa. 
Se a imagem de $f$ está contida no interior de uma circunferência $|w| - \alpha$, então $f$ é uma função constante.
\end{lema}


\begin{corolario}[Princípio do Módulo Máximo]
Sejam $U$ um domínio em $\mathbb{C}$ e $f:U\to\mathbb{C}$ uma função holomorfa.
Suponha que existe um ponto $a\in U$ tal que $|f(a)| \geq |f(z)|, \forall z\in U$. Então $f$ é uma função constante.
\end{corolario}


\begin{teorema}
Sejam $f:U\to\mathbb{C}$ uma função holomorfa com $U$ domínio e $z_0\in U$ qualquer.
Então
\begin{equation*}
    f(z) = \sum_{n=0}^{\infty} \frac{f^{(n)}(z_0)}{n!}(z-z_0)^n,
\end{equation*}
ou seja, $f$ é dada por sua série de Taylor de centro em $z_0$ e, portanto, é uma função analítica.
Ademais, essa série converge em qualquer disco (aberto) $D(z_0, r) \subset U$, isto é, o raio de convergência $R$ da série acima é a menor entre as distâncias de $z_0$ aos pontos da fronteira de $U$.
\end{teorema}


\begin{teorema}[Teorema de Cauchy]
\label{teo-cauchy}
Sejam $U\subset\mathbb{C}$ um domínio e $f:U\to\mathbb{C}$ uma função holomorfa. Seja $V\subset U$ um subconjunto fechado e limitado, cuja fronteira $\partial V$ consiste de um número finito de curvas de Jordan
suaves por partes, $\partial V = \gamma_1\cup\cdots\gamma_n$, e tal que
$V\setminus\partial V$ é domínio. Então
\begin{equation*}
    \int_{\partial V} f(z) dz = 0.
\end{equation*}
\end{teorema}

\begin{teorema}[Fórmula Integral de Cauchy bis.]
Seja $f:U\to\mathbb{C}$ uma função holomorfa definida no domínio $U$. Seja $V$ uma região fechada e limitada inteiramente contida em $U$, 
cuja fronteira $\partial V$ é uma curva de Jordan suave por partes, orientada no sentido anti-horário, sendo $V\setminus\partial V$ um domínio.
Se $z_0$ é um ponto qualquer no interior de $V$, então
\begin{equation*}
    f(z_0) = \frac{1}{2\pi i}\int_{\partial V} \frac{f(w)}{w-z_0} dw.
\end{equation*}
\end{teorema}


\begin{teorema}[Teorema de Morera]
\label{teo-morera}
Sejam $U\subset\mathbb{C}$ domínio e $f:U\to\mathbb{C}$ uma função contínua.
Se $\int_{\Delta} f(z) dz = 0$ para todo caminho triangular $\Delta\subset U$, então $f$ é holomorfa em $U$.
\end{teorema}


\section[Singularidades, resíduos e o Teorema de Rouché]{Singularidades, resíduos e o Teorema de Rouché}

\begin{teorema}[Teorema de Laurent]
\label{teo-laurent}
Seja $f$ uma função holomorfa...
\end{teorema}