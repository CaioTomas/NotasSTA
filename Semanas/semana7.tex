% !TeX spellcheck = pt_BR
%\chapter[Semana 7]{}
\chapter[Teoria de Cauchy]{Teoria de Cauchy:\\ Integração no plano complexo}
\chaptermark{}


\hfill%
\begin{minipage}{12cm}
	\begin{flushright}
		\rightskip=0.5cm
		\textit{``At the basis of the distance concept lies, for example, 
		the concept of a convergent point sequence and their defined limits,
		and one can, choosing these ideas as those fundamental to the point 
		set theory, eliminate the notions of distance ... Thirdly, 
		we can associate with each point of the set certain parts of the space
		called neighborhoods, and these can again be made building stones of the 
		theory with the elimination of the distance concept. Here the view
		of a set is in consideration of the association between elements and
		subsets.''}
		\\[0.1cm]
		\rightskip=0.5cm
		---F. Hausdorff, 1949
	\end{flushright}
\end{minipage}



\section{A Integral Complexa}

\begin{lema}
Sejam $U \subset \mathbb{C}$ um domínio
\end{lema}




\section[Primitivas e o Teorema Fundamental do Cálculo]
{Primitivas e o Teorema Fundamental\\ do Cálculo}
\begin{center}
Em construção...
\end{center}


\section[Os teoremas de Cauchy]{Os teoremas de Cauchy}
\begin{center}
Em construção...
\end{center}