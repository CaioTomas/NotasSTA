\chapter[Introdução]{Introdução}
\chaptermark{}


\section{Contexto Histórico}

    Este livro é um estudo do artigo de 8 páginas ``On the number of primes less than a given magnitude''\footnote{O título em Alemão é Ueber die Ahzahl der Primzahlen unter einer gegebenen Gr\"osse.} de Bernhard Riemann que marcou época e dos subsequentes desenvolvimentos da teoria que este artigo inaugura. O primeiro capítulo é uma análise do artigo em si e os 11 capítulos restantes são voltados para o trabalho que foi feito, desde 1859, sobre as questões que Riemann deixou sem resposta.

    A teoria para a qual o artigo de Riemann é uma contribuição tem seu início no Teorema de Euler, provado em 1737, da divergência da soma dos recíprocos dos números primos
    %
    \begin{equation}
    \label{Soma-Recip-primos}
    \frac{1}{2} + \frac{1}{3} + \frac{1}{5} + \frac{1}{7} + \frac{1}{11} + \frac{1}{13} + \frac{1}{17} + \cdots.
    \end{equation}
    %
    Este teorema vai além do clássico 
    Teorema de Euclides \cite{MR1932864}, sobre a infinitude 
    dos números primos, e mostra que os números primos 
    são densos no conjunto dos números inteiros - 
    mais densos que os quadrados, por exemplo, 
    no sentido de que a soma dos recíprocos dos quadrados converge.
    
    Euler vai além de simplesmente enunciar a divergência de
    \eqref{Soma-Recip-primos} e diz que, como 
    $1 + \frac{1}{2} + \frac{1}{3} + \frac{1}{4} + \cdots$ 
    diverge como o logaritmo e 
    (\ref{Soma-Recip-primos}) diverge como o logaritmo de
    \footnote
    {Isto é verdade por conta da fórmula de Euler $\sum(1/n) = \prod (1-p^{-1})^{-1}$, então $\log (\sum(1/n)) = - \log(\prod(1-p^{-1})^{-1}) = -\log(\prod(1-p^{-1})) = -\sum (p^{-1} + \frac{1}{2}p^{-2} + \frac{1}{3}p^{-3} + \cdots ) = \sum(1/p) + \text{ convergente}.$}
    $1 + \frac{1}{2} + \frac{1}{3} + \frac{1}{4} + \cdots$, a série \eqref{Soma-Recip-primos} deve divergir como o $\log$ do $\log$, que Euler escreve como 
    %
    \begin{equation}
    \label{Soma-Recip-primos-log}
    \frac{1}{2} + \frac{1}{3} + \frac{1}{5} + \frac{1}{7} + \frac{1}{11} + \frac{1}{13} + \frac{1}{17} + \cdots = \log(\log \infty).
    \end{equation}
    %
    Não é claro o que esta equação significava para Euler - isto se ele a entendia como algo além de um simples mnemônico. Entretanto uma interpretação óbvia dela poderia ser
    %
    \begin{equation}
    \label{Soma-Recip-primos-x}
    \sum_{p<x}\frac{1}{p} \approx \log(\log x) \ \ \ \ \ \ (x \to \infty),
    \end{equation}
    %
    onde o lado esquerdo denota a soma de $1/p$ sobre todos os primos $p$ menores que $x$ e o símbolo $\approx$ significa que o erro relativo é arbitrariamente pequeno para $x$ suficientemente grande ou, o que é a mesma coisa, que a razão entre os dois lados se aproxima de $1$ conforme $x \to \infty$. Agora,
    %
    $$\log(\log x) = \int_{1}^{\log x} \frac{du}{u} = \int_{e}^{x} \frac{dv}{v\log v},$$
    %
    então \eqref{Soma-Recip-primos-x} significa que a integral de $1/v$ relativa à métrica $dv/\log v$ diverge da mesma maneira que a integral de $1/v$ relativa à \textcolor{red}{métrica discreta} que dá peso 1 para os primos e peso 0 para todos os outros pontos. Neste sentido, \eqref{Soma-Recip-primos-x} tem a interpretação de que a densidade dos primos é, essencialmente, $1/\log v$. No entanto, não há evidências de que Euler tenha pensado na densidade dos números primos e seus métodos eram inadequados para provar a formulação \eqref{Soma-Recip-primos-x} do que ele enuncia em \eqref{Soma-Recip-primos-log}.
    
    Gauss afirma, em uma carta escrita em 1849, que ele havia observado por volta de 1792 ou 1793 que a densidade dos primos aparenta ser, em média, $1/\log x$ e diz que cada nova tabela de primos publicada em anos subsequentes vinha a confirmar sua crença da precisão desta aproximação. No entanto, ele não menciona a fórmula de Euler \eqref{Soma-Recip-primos-log} e não dá nenhuma base analítica para esta aproximação, que é apresentada somente com base em observações empíricas. Ele mostra a Tabela \ref{Tabela-Gauss}.
    %
    \begin{table}[H]
        \centering
        \begin{tabular}{cccc}
             $x$& número de primos $< x$ & $\displaystyle \int \frac{dn}{\log n}$ &  Diferença
             \\[0.45cm]
             \hline
             \\[-0.3cm]
             500.000   & 41.556  & 41.606,4  & 50,4 \\[0.1cm]
             1.000.000 & 78.501  & 78.627,5  & 126,5 \\[0.1cm]
             1.500.000 & 114.112 & 114.263,1 & 151,1 \\[0.1cm]
             2.000.000 & 148.883 & 149.054,8 & 171,8 \\[0.1cm]
             2.500.000 & 183.016 & 183.245,0 & 229,0 \\[0.1cm]
             3.000.000 & 216.745 & 216.970,6 & 225,6
        \end{tabular}
        \caption{}
        \label{Tabela-Gauss}
    \end{table}
    %
    
    Gauss não diz exatamente o que significa o símbolo $\int (dn/\log n)$, mas os dados na Tabela \ref{Tabela-Lehmer}, tirada de D. N. Lehmer \cite{Lehmer13}, indica que $n$ é uma variável contínua integrada de $2$ a $x$, ou seja, $\int_{2}^{x}(dt/\log t)$. Observe que a contagem de primos de Lehmer
    %
    \footnote{Lehmer insiste em contar 1 como número primo. Para adequar ao consenso comum, sua contagem foi reduzida em uma unidade na Tabela \ref{Tabela-Lehmer}}
    %
    , que podemos seguramente considerar precisa, difere da de Gauss e que a diferença corrobora a estimativa de Gauss para os valores grandes de $x$.
    
    %
    \begin{table}[H]
        \centering
        \begin{tabular}{cccc}
             $x$& número de primos $< x$ & $\displaystyle \int_{2}^{x} \frac{dt}{\log t}$ &  Diferença\\[0.45cm]
             \hline\\[-0.3cm]
             500.000   & 41.538  & 41.606  & 68  \\[0.1cm]
             1.000.000 & 78.498  & 78.628  & 130 \\[0.1cm]
             1.500.000 & 114.155 & 114.263 & 108 \\[0.1cm]
             2.000.000 & 148.933 & 149.055 & 122 \\[0.1cm]
             2.500.000 & 183.072 & 183.245 & 173 \\[0.1cm]
             3.000.000 & 216.816 & 216.971 & 155
        \end{tabular}
        \caption{}
        \label{Tabela-Lehmer}
    \end{table}
    %
    
    Por volta de 1800, Legendre publicou em sua obra \textit{Theorie des nombres} \cite{legendre1830} uma fórmula empírica para o número de primos menores que um dado valor, que sugeria o mesmo resultado, isto é, que a densidade dos primos é $1/\log x$. Mesmo com a tentativa de Legendre de provar sua fórmula, o seu argumento se reduz a nada mais que a afirmação de que, se assumimos que a densidade dos primos tem a forma
    %
    $$\frac{1}{A_1x^{m_1} + A_1x^{m_2} + \cdots},$$
    %
    onde $m_1 > m_2> \cdots$, então $m_1$ não pode ser positivo (porque a soma \eqref{Soma-Recip-primos} convergiria neste caso) mas sim ``infinitamente pequeno'' e 
    em consequência disto a densidade 
    deveria ser da forma
    %
    $$\frac{1}{A \log x + B}.$$
    %
    Em seguida, Legendre determina as constantes $A$ e $B$ de forma empírica, usando cálculos numéricos. 
    Ressaltamos que a fórmula de 
    Legendre era bem conhecida no mundo matemático e foi mencionada por matemáticos como Abel \cite{Abel-Holmboe}, Dirichlet \cite{Dirichlet38} 
    and Chebyschev \cite{Chebyshev52} no período de 1800 a 1850.
    
    Os primeiros resultados significativos além dos de Euler foram obtidos por Chebyshev por volta de 1850. Ele provou que o erro relativo na aproximação
    %
    \begin{equation}
        \label{pi(x)-Li(x)}
        \pi(x) \approx \int_{2}^{x} \frac{dt}{\log t},
    \end{equation}
    %
    onde $\pi(x)$ denota o número de primos menores que $x$, é menor que 11\% para $x$ suficientemente grande; ou seja, ele provou
    %
    \footnote{
    Chebyshev não enunciou seu resultado desta forma. Esta forma pode ser obtida das suas estimativas para o número de primos entre $l$ e $L$ \textcolor{red}{(veja Chebyshev [C3, seção 6])} fixando $l$ e tomando $L \to \infty$ e usando $\int_{2}^{L}(dt/\log t) \approx L/\log L$.}
    %
    que
    %
    \begin{equation*}
        (0,89)\int_{2}^{x} \frac{dt}{\log t} < \pi(x) < (1,11)\int_{2}^{x} \frac{dt}{\log t}
    \end{equation*}
    %
    para todo $x$ suficientemente grande. Além disso, ele provou que nenhuma aproximação do tipo Legendre
    %
    $$ \pi(x) \approx \frac{x}{A\log x + B} $$
    %
    pode ser melhor que a aproximação \eqref{pi(x)-Li(x)} e, 
    se a razão entre $\pi(x)$ e 
    $\int_{2}^{x} (dt/\log t$ se aproxima de um limite conforme $x \to \infty$, 
    então esse limite deve ser 1. Fica claro que Chebyshev estava tentando 
    provar que o erro relativo na aproximação \eqref{pi(x)-Li(x)} 
    se aproxima de zero quando $x \to \infty$, mas foi apenas 
    quase 50 anos depois que este teorema, conhecido como 
    ``Teorema do número primo'' foi provado. 
    Mesmo o trabalho de Chebyshev sendo publicado na França muito antes 
    do artigo de Riemann, ele não se refere a Chebyshev em seu artigo. 
    No entanto, Riemann se refere a Dirichlet que 
    {\red se correspondia com} Chebyshev 
    (veja o relatório de Chebyshev sobre sua viagem à Europa Oriental 
    \cite{Chebyshev47} volume 5). 
    Dirichlet teria provavelmente informado Riemann sobre o trabalho de Chebyshev. 
    Os trabalhos não publicados de Riemann contém muitas fórmulas de Chebyshev, 
    o que indica que ele estudou o trabalho de Chebyshev, 
    e contém pelo menos uma referência direta a Chebyshev {\red(veja a figura ???)}.
    
    A contribuição real do artigo de Riemann de 1859 não está em 
    seus resultados, mas, sim, em seus métodos. 
    O resultado principal é uma fórmula para $\pi(x)$ como uma soma de 
    uma série infinita da qual $\int_{2}^{x}(dt/\log t)$ é, de longe, o maior termo. 
    No entanto, a prova de Riemann deste fato é inadequada; 
    em particular, não é, de forma alguma, claro, pelos argumentos de Riemann, 
    que a série infinita para $\pi(x)$ convirja, muito menos que seu maior 
    termo seja realmente assintoticamente dominante para valores grandes de $x$. Por outro lado, os métodos de Riemann, 
    que incluem: o estudo da função $\zeta(s)$, como uma função de uma variável complexa; 
    o estudo dos zeros complexos de $\zeta(s)$;  \textcolor{red}{a inversão de Fourier}; 
    \textcolor{red}{a inversão de Möbius} e a representação da função $\pi(x)$ por ``fórmulas explícitas'' como uma série 
    infinita, foram importantes contribuições no desenvolvimento subsequente da teoria.
    
    Nos primeiros 30 anos após a publicação do artigo de Riemann, não houve praticamente nenhum avanço
    %
    \footnote{ Uma grande exceção foi o Teorema de Mertens 
    \cite{MR1579612} de 1874 que afirma que \eqref{Soma-Recip-primos-x} 
    é verdadeira no sentido de que a diferença dos dois lados se 
    aproxima de um limite quando $x \to \infty$. Este limite é a 
    constante de Euler adicionado de $\sum_{p}[\log(1-p^{-1}) + p^{-1}]$. 
    Possivelmente, um enunciado mais natural para o Teorema de Mertens é 
    %
    $$
    \lim_{x \to \infty} 
    \log \left( x \prod_{p<x}(1-p^{-1}) \right) 
    = 
    e^{-\gamma},
    $$
    %
    onde $\gamma$ é a constante de Euler. 
    Veja Hardy e Wright \cite{MR0067125}}
    %
    na área. É como se o mundo matemático precisasse deste tempo 
    todo para processar as ideias de Riemann. Então, em um período de 10 anos, 
    Hadamard, Von Mangoldt e de la Vallée Poussin conseguiram provar a 
    fórmula principal de Riemann para $\pi(x)$, 
    o Teorema dos números primos (conforme a expressão \eqref{pi(x)-Li(x)}) e vários outros 
    teoremas relacionados. As ideias de Riemann foram cruciais em 
    todas as demonstrações. Desde esse período, não houve falta de 
    novos problemas em Teoria Analítica dos Números e o progresso 
    se manteve estável neste campo, boa parte deste progresso é 
    inspirado pelas ideias de Riemann. 
    
    Nenhuma discussão do contexto histórico do artigo de Riemann estaria 
    completa se não mencionasse a hipótese de Riemann. Ao longo do artigo, 
    Riemann diz que considera ``muito provável'' que todos os zeros de 
    $\zeta(s)$ tem parte real igual a $1/2$, mas que foi incapaz de 
    provar a veracidade desta afirmação, que é conhecida, hoje em dia, 
    como a ``hipótese de Riemann''. A experiência dos sucessores de Riemann 
    com esta hipótese tem sido a mesma de Riemann - eles também consideram 
    que ela provavelmente é verdadeira mas não foram capazes de prová-la. 
    Hilbert incluiu o problema de provar a hipótese de Riemann em sua 
    lista \cite{hilbert1902problemes} dos problemas não resolvidos 
    mais importantes do mundo matemático em 1900. 
    Resolver este problema foi o foco de muitos dos melhores 
    matemáticos do Século XX. Este é, sem dúvidas, o 
    problema mais celebrado da matemática e continua a atrair 
    a atenção dos melhores matemáticos, não apenas porque 
    continua sem solução por tanto tempo, mas também por 
    que aparenta ser vulnerável e também porque há uma grande expectativa que para resolver este problema novas técnicas de grande potencial matemático
    tenha que ser desenvolvidas.
    
    
    
    \section{A fórmula do produto de Euler}
    
    
    Riemann toma, de início, a fórmula
    %
    \begin{equation}
        \label{Euler-prod}
        \sum_{n} \frac{1}{n^s} = \prod_{p} \frac{1}{\left( 1 - \frac{1}{p^s}\right)}
    \end{equation}
    %
    devida a Euler. Aqui $n$ percorre todos os inteiros positivos $(n=1,2,3, \dots)$ e $p$, 
    todos os primos $(p=2,3,5,7,11,\dots$. 
    Esta fórmula, conhecida como ``fórmula do produto de Euler'', resulta de expandir todos os fatores à direita
    %
    $$\frac{1}{\left( 1 - \frac{1}{p^s}\right)} = 1 + \frac{1}{p^s} + \frac{1}{(p^2)^s} + \frac{1}{(p^3)^s} + \cdots $$
    %
    e observar que seu produto é uma soma de fatores da forma
    %
    $$\frac{1}{(p_1^{n_1} p_2^{n_2} \cdots p_r^{n_r})^s},$$
    %
    onde $p_1, \dots, p_r$ são primos distintos e $n_1, \dots, n_r$ são números naturais, e, então, usar o Teorema Fundamental da Aritmética (todo inteiro pode, essencialmente, ser escrito de uma única maneira como um produto de números primos) para concluir que esta soma é, de fato, $\sum_{n}(1/n^s)$. Euler usou esta fórmula principalmente como uma identidade formal e para valores inteiros de $s$ (veja, por exemplo, Euler \cite{MR0016336}).
    
    Dirichlet também baseou seu trabalho
    %
    \footnote{A maior contribuição de Dirichlet para a teoria foi sua demonstração de que, se $m$ coprimo com $n$, então há infinitos números primos $p$ satisfzendo $p \equiv m \ (\mod n)$. Ele também estava interessado em questões relacionadas à densidade da distribuição dos primos, mas ele não teve um sucesso significativo com estas questões. 
    }
    %
    neste campo na fórmula do produto de Euler. Como Dirichlet foi um dos professores de Riemann e ele se refere a Dirichlet no primeiro parágrafo de seu artigo, o fato de Riemann ter usado a fórmula do produto foi provavelmente influenciado por Dirichlet. Diferentemente de Euler, Dirichlet usou a fórmula \eqref{Euler-prod} com $s$ sendo uma variável real e provou
    %
    \footnote{Dirichlet \cite{Dirichlet38}. Como os termos $p^{-s}$ são todos positivos, não há nada difícil relacionado a esta prova - é essencialmente uma reordenação de séries absolutamente convergentes - mas tem a importante efeito de transformar \eqref{Euler-prod} de uma identidade formal válida  para diversos valores de $s$ para uma expressão analítica verdadeira para todo número real $s>1$.
    }
    %
    rigorosamente que esta equação é válida sempre que $s>1$.
    
    É natural esperar que Riemann, como um dos fundadores da teoria de funções de uma variável complexa, considerasse $s$ como uma variável complexa. É fácil mostrar que ambos os lados de \eqref{Euler-prod} converge para $s$ complexo no semiplano $\Re(s)>1$, mas Riemann vai muito além e mostra que, mesmo os dois lados da fórmula divergindo para outros valores de $s$, a função que eles definem tem significado para todo $s$ complexo, exceto por um polo em $s=1$. Esta extensão do domínio da função requer alguns fatos sobre a função fatorial que veremos na próxima seção.
    
    
    \section{A função fatorial}
    
    
    Euler estendeu a função fatorial $n! = n(n-1)(n-2) \cdots 3 \cdot 2 \cdot 1$ dos números naturais $n$ para todo número real maior que $-1$ observando que 
    %
    \footnote{Euler escreveu a integral em termos de $y = e^{-x}$ como $\int_{0}^{1}(\log 1/y)^n \, dy$ (veja Euler \textcolor{red}{[E3]}
    }
    %
    \begin{equation}
        \label{factorial-Gamma}
        n! = \int_{0}^{\infty}e^{-x}x^n \, dx \qquad (n = 1, 2, 3, \dots)
    \end{equation}
    %
    (integração por partes) e observando, também, que que a integral à direita converge para valores não inteiros de $n$ desde que $n>-1$. Gauss \cite{gauss1813circa} introduziu a notação
    %
    \begin{equation}
        \label{factorial-Gamma-pi}
        \Pi(s) = \int_{0}^{\infty}e^{-x}x^s \, dx \qquad (s>-1)
    \end{equation}
    %
    para a integral de Euler no lado direito de \eqref{factorial-Gamma}. Então $\Pi(s)$ é definida para todo número real $s$ maior que -1, na verdade, para todo $s$ complexo no semiplano $\Re(s)>-1$. Além disso, $\Pi(s) = s!$ sempre que $s$ é um número natural. Existe uma outra representação de $\Pi(s)$ que era conhecida
    %
    \footnote{Infelizmente, Legendre adotou $\Gamma(s)$ para $\Pi(s-1)$. As razões de Legendre para considerar $(n-1)!$ em vez de $n!$ são desconhecidas (talvez ele sentiu que fosse mais natural o primeiro polo ser em $s=0$ em vez de $s=-1$). De qualquer modo, esta notação prevaleceu na França e, por volta do fim do século 19, no resto do mundo também. A notação de Gauss me parece muito mais natural e o fato de Riemann tê-la usado me dá a oportunidade de trazê-la de volta.
    }
    %
    por Euler
    %
    \begin{equation}
        \label{Pi(s)-lim-N}
        \Pi(s) = \lim_{N \to \infty} \frac{1 \cdot 2 \cdots N}{(s+1)(s+2)\cdots (s+N)}(N+1)^s.
    \end{equation}
    %
    Esta fórmula é válida para todos os $s$ para os quais \eqref{factorial-Gamma-pi} define $\Pi(s)$, ou seja, para todo $s$ no semiplano $\Re(s)>-1$. Por outro lado, não é difícil mostrar [use a fórmula \eqref{Pi(s)-prod} abaixo] que o limite \eqref{Pi(s)-lim-N} existe para todo valor de $s$, real ou complexo, desde que o denominador não se anule, isto é, que $s$ não seja um inteiro negativo. Em suma, a fórmula \eqref{Pi(s)-lim-N} estende a definição de $\Pi(s)$ para todos os valores de $s$ exceto para $s = -1, -2, -3, \dots$.
    
    As fórmulas \eqref{factorial-Gamma-pi} e \eqref{Pi(s)-lim-N} coincidem para $\Re(s)> -1$. Também usaremos, sem demonstrar, os seguintes fatos:
    %
    \begin{align}
        \Pi(s) &= \prod_{n=1}^{\infty}\frac{n^{1-s}(n+1)^s}{s+n} =  \prod_{n=1}^{\infty}\left(1 + \frac{s}{n}\right)^{-1} \left(1 + \frac{1}{n}\right)^{s} \label{Pi(s)-identidade} \\[0.3cm]
        %
        \Pi(s) &= s\Pi(s-1) \label{Pi(s)-identidade-2} \\[0.3cm]
        %
        \frac{\pi s}{\Pi(s)\Pi(-s)} &= \sin \pi s  \label{Pi(s)-identidade-3} \\[0.3cm]
        %
        \Pi(s) &= 2^s\Pi\left(\frac{s}{2}\right)\Pi\left(\frac{s-1}{2}\right)\pi^{-1/2} \label{Pi(s)-identidade-4}
    \end{align}
    %
    Para a demonstração destes resultados o leitor é indicado a procurar qualquer livro que lida com a função fatorial ou a ``função $\Gamma$'', por exemplo, Edwards \cite{MR1319337}. 
    A identidade \eqref{Pi(s)-identidade} é uma simples reformulação 
    de \eqref{Pi(s)-lim-N}. Usando ela, pode-se provar que $\Pi(s)$ é uma 
    função analítica da variável complexa $s$ que tem polos simples 
    em $s = -1, -2, -3, \dots$ e não se anula. 
    A identidade \eqref{Pi(s)-identidade-2} é chamada de 
    ``equação funcional da função fatorial''; 
    juntamente com $\Pi(0)=1$ (de \eqref{Pi(s)-identidade}), 
    obtemos $\Pi(n) = n!$ diretamente. A fórmula \eqref{Pi(s)-identidade-3} é, 
    essencialmente, a fórmula do produto para o seno; quando $s = 1/2$, 
    usamos \eqref{Pi(s)-identidade-2} para concluir que 
    $\Pi(\frac{-1}{2}) = \Pi^{1/2}$. 
    A identidade \eqref{Pi(s)-identidade-4} é conhecida como a relação 
    de Legendre. É o caso $n=2$ de outra relação mais geral
    %
    \begin{equation*}
        \frac{\Pi(s)}{n^s \Pi\left( \frac{s-1}{2}\right) \cdots \Pi\left( \frac{s-n + 1}{n}\right)} 
        = \left[ \frac{2\pi n}{(2\pi)^n}\right]^{1/2}
    \end{equation*}
    %
    que não será necessária.
    
    
    \section{A função $\zeta(s)$}
    
    
    Riemann não fala sobre a "continuação analítica" da função $\sum n^{-s}$ para além do semiplano $\Re(s)>1$, mas fala em encontrar uma fórmula para ela que "seja válida para todo $s$". Isto indica que ele enxergava este problema em termos mais próximos do problema da extensão da função fatorial pela fórmula \eqref{Pi(s)-lim-N} da seção anterior do que da extensão analítica da maneira que é vista atualmente. Enxergar a continuação analítica em termos de cadeias de discos e séries de potências convergentes em cada disco vem de Weierstrass e é quase a antítese da filosofia básica de Riemann, que deve-se lidar com funções analíticas de forma global e não localmente em termos de séries de potências. 
    
    Riemann obtém sua fórmula para $\sum n^{-s}$ que "se mantém válida para todo $s$" da forma seguinte. Ele substitui $nx$ no lugar de $x$ na integral de Euler para $\Pi(s-1)$ e obtém 
    %
    \begin{equation*}
        \int_{0}^{\infty}e^{-nx}x^{s-1} \, dx = \frac{\Pi(s-1)}{n^s}
    \end{equation*}
    %
    $(s > 0, n = 1, 2, \dots)$. Soma sobre $n$ e usa $\sum_{n=1}^{\infty} r^{-n} = (r-1)^{-1}$ para obter 
    %
    \footnote{
    Esta fórmula com $s=2n$ ocorre em um artigo \textcolor{red}{[A1]} de Abel que foi incluído na edição de 1839 do Abel's collected works. Parece bastante provável que Riemann estivesse apar deste fato. Uma formula muito similar
    %
    $$\int_0^{\infty} (e^x-1)^{-1}e^{-x}x^p \, dx = \Pi(p)\sum_{n=2}^{\infty} n^{-1-p} $$
    %
    é o ponto de partida de um artigo de Chebyshev \textcolor{red}{[C2]} de 1848.
    }
    %
    \begin{equation}
        \label{int-zeta-Pi}
        \int_0^{\infty} \frac{x^{s-1}}{e^{x} - 1} \, dx = \Pi(s-1)\sum_{n=1}^{\infty}\frac{1}{n^s}
    \end{equation}
    %
    $(s > 1)$. A convergência da integral imprópria à esquerda e a validade da troca da soma com a integração não são difíceis de demonstrar. 
    
    A seguir, ele considera a integral de contorno 
    %
    \begin{equation*}
        \int_{+\infty}^{+\infty} \frac{(-x)^s}{e^x - 1} \, \frac{dx}{x}.
    \end{equation*}
    %
    Os limites de integração buscam indicar que o caminho de integração vem de $+\infty$, vai para a esquerda no eixo real, circula a origem uma vez com orientação positiva (anti-horária) e retorna, pelos reais positivos, a $+\infty$. $(-x)^s$ é definido como $\exp(s\log(-x))$, onde $\log(-x)$ coincide com a definição usual de $\log z$ para $z$ fora do eixo real negativo, ou seja, escolhendo o ramo do logaritmo que exclui os reais não positivos. Então $(-x)^s$ não está definido para $x$ real positivo e, a rigor, o caminho de integração escolhido deve estar um pouco acima do eixo real conforme vem de $+\infty$ para $0$ e um pouco abaixo do eixo real quando retorna de 0 para $+\infty$. Quando esta integral é escrita na forma 
    %
    \begin{equation*}
         \int_{+\infty}^{+\delta} \frac{(-x)^s}{e^x - 1} \, \frac{dx}{x} +  \int_{|x| = \delta} \frac{(-x)^s}{e^x - 1} \, \frac{dx}{x} + \int_{\delta}^{+\infty} \frac{(-x)^s}{e^x - 1} \, \frac{dx}{x}
    \end{equation*}
    %
    o termo do meio é $2\pi i$ multiplicado pela valor médio de $(-x)^s(e^x - 1)^{-1}$ no círculo $|x| = \delta$ [pois, neste círculo, $i d\theta = (dx/x)$]. Então este termo vai a zero conforme $\delta \to 0$ desde que $s>1$ [pois $x(e^x -1)$ é indeterminado em $x = 0$]. Os outros dois termos podem, então, ser combinados para obtermos 
    %
    \begin{align*}
        \int_{+\infty}^{+\infty} \frac{(-x)^s}{e^x - 1} \, \frac{dx}{x}
        %
        &= \lim_{\delta \to 0} \left\{ \int_{+\infty}^{+\delta} \frac{\exp[s(\log(x)-i\pi)]}{(e^x - 1)x} \, dx + \int_{\delta}^{+\infty} \frac{\exp[s(\log(x)-i\pi)]}{(e^x - 1)x} \, dx \right\} \\
        %
        &= (e^{i\pi s} - e^{-i\pi s})\int_{0}^{+\infty} \frac{x^{s-1} }{e^x - 1} \, dx.
    \end{align*}
    %
    Usando a fórmula \eqref{int-zeta-Pi}, temos
    %
    \begin{equation*}
        \int_{+\infty}^{+\infty} \frac{(-x)^s}{e^x - 1} \, \frac{dx}{x} = 2 i \sen(\pi s)\Pi(s-1)\sum_{n=1}^{\infty}\frac{1}{n^s}.
    \end{equation*}
    %
    Multiplicando ambos os lados por $\Pi(-s)s/2\pi i s$ e usando aidentidade \eqref{Pi(s)-identidade-3}, obtemos
    %
    \begin{equation}
        \label{extensao-zeta-1}
        \frac{\Pi(-s)}{2\pi i} \int_{+\infty}^{+\infty} \frac{(-x)^s}{e^x - 1} \, \frac{dx}{x} = \sum_{n=1}^{\infty}\frac{1}{n^s}.
    \end{equation}
    %
    Em outras palavras, se $\zeta(s)$ é definida pela expressão
    %
    \footnote{
    Esta expressão é enunciada de forma incorreta pelos editores dos trabalhos de Riemann nas notas. Eles colocam o fator $\pi$ no lado errado da equação deles.
    }
    %
    \begin{equation}
        \label{extensao-zeta-2}
        \zeta(s) = \frac{\Pi(-s)}{2\pi i} \int_{+\infty}^{+\infty} \frac{(-x)^s}{e^x - 1} \, \frac{dx}{x},
    \end{equation}
    %
    então, para valores reais de $s$ maiores que 1, $\zeta(s)$ é igual à função de Drichlet
    %
    \begin{equation}
        \label{zeta-def-drich}
        \zeta(s) = \sum_{n=1}^{\infty}\frac{1}{n^s}
    \end{equation}
    %
    
    No entanto, a expressão \eqref{extensao-zeta-2} para $\zeta(s)$ "se mantém válida para todo $s$". De fato, como a integral em \eqref{extensao-zeta-2} converge para todos os valores de $s$, real ou complexo (pois $e^x$ cresce muito mais rápido que $x^s$ conforme $s \to \infty$), e a função que ela define é analítica (porque a convergência é uniforme em compactos), a função $\zeta(s)$ de \eqref{extensao-zeta-2} está definida e é analítica em todos os pontos, exceto, possivelmente, pelos pontos $s = 1, 2, 3, \dots$, onde $\Pi(-s)$ tem polos. Mas a expressão \eqref{zeta-def-dric} mostra que $\zeta(s)$ não tem polos em $s = 2,3,4, \dots$ (então a integral em \eqref{extensao-zeta-2} deve ter zeros que cancelam os polos de $\Pi(-s)$ nesses pontos, este fato segue imediatamente do Teorema de Cauchy) e, em $s = 1$, \eqref{zeta-def-dric} mostra que $\lim_{s \to 1} \zeta(s) = \infty$, então $\zeta(s)$ tem um polo simples em $s=1$ (pois o polo de $\Pi(-s)$ é simples). Portanto, \textit{a expressão \eqref{extensao-zeta-2} define uma função $\zeta(s)$ que é analítica em todos os pontos do plano complexo exceto por um polo em $s = 1$}. Esta função coincide com $\sum n^{-s}$ para $s>1$ real e, por continuação, em todo o semiplano $\Re(s) > 1$.
    
    A função $\zeta(s)$ é conhecida como a função zeta de Riemann.
    
    
    \section{Valores de $\zeta(s)$}
    
    
    A função $x(e^x - 1)^{-1}$ é analítica próximo de $x = 0$, logo, pode ser expressa em uma série de potências
    %
    \begin{equation}
        \label{xexp-serie-pot}
        \frac{x}{e^x - 1} = \sum_{n=0}^{\infty} \frac{B_n x^n}{n!}
    \end{equation}
    %
    válida perto de zero (na verdade, é válida no disco $|x| < 2\pi$ e estende a função para as singularidades $x = \pm 2\pi i$ de $x(e^x - 1)^{-1}$). Os coeficientes $B_n$ são, por definição, números de Bernoulli; são os primeiros são facilmente determinados:
    %
    \begin{align*}
        &B_0 = 1, \ \ \ \ &&B_1 = -\frac{1}{2}, \\
        &B_2 = \frac{1}{6}, \ \ \ \ &&B_3 = 0, \\
        &B_4 = -\frac{1}{30}, \ \ \ \ &&B_5 = 0, \\
        &B_6 = \frac{1}{42}, \ \ \ \ &&B_7 = 0, \\
        &B_8 = -\frac{1}{30}, \ \ \ \ &&B_9 = 0, 
    \end{align*}
    %
    Os números de Bernoulli com índice ímpar são todos nulos
    %
    \footnote{ Isto pode ser mostrado facilmente notando que $(-t)(e^{-t} - 1)^{-1} + (-t/2) = (-te^t + t - t)(1 - e^t)^{-1} - (t/2) = t(e^t - 1)^{-1} + (t/2)$, ou seja, $t(e^t - 1)^{-1} + (t/2)$ é uma função par. \textcolor{red}{Para provas alternativas veja as notas da seção 1.6 e a expressão (10) da seção 6.2}
    } 
    %
    após o primeiro e os de índice par podem ser determinados sucessivamente, mas não há uma fórmula simples para calculá-los. (Veja Euler \textcolor{red}{[E6]} para uma lista de valores de $(-1)^{n-1}B_{2n}$ até $B_{30}$).
    
    Quando $s = -n \ (n = 0, 1, 2, \dots)$, a expansão em \eqref{xexp-serie-pot} pode ser usada na definição de $\zeta(s)$ e obtém-se
    %
    \begin{align*}
        \zeta(-n) &= \frac{\Pi(n)}{2\pi i} \int_{+\infty}^{+\infty} \frac{(-x)^{-n}}{e^x - 1} \frac{dx}{x} \\
        %
        &= \frac{\Pi(n)}{2\pi i} \int_{|x| = \delta} \left( \sum_{m=0}^{\infty} \frac{B_m x^m}{m!} \right) \frac{(-x)^{-n}}{x} \frac{dx}{x} \\
        %
        &= \sum_{m=0}^{\infty} \Pi(n)\frac{B_m}{m!}(-1)^n \cdot \frac{1}{2\pi} \int_{0}^{2\pi}  x^{m - n - 1} \, d\theta \\
        %
        &= n!\frac{B_{n+1}}{(n+1)!}(-1)^n = (-1)^n \frac{B_{n+1}}{n+1}.
    \end{align*}
    %
    Riemann não mostra esta fórmula para $\zeta(-n)$, mas enuncia os resultados particulares $\zeta(-2) = \zeta(-4) = \zeta(-6) =  \cdots = 0$. Eles estava, certamente, ciente, no entanto, não apenas dos valores 
    %
    \footnote{ 
    O editor dos trabalhos de Riemann escreve o valor incorreto $\zeta(0) = 1/2$.
    }
    %
    \begin{equation*}
        \zeta(0) = -1/2, \ \ \zeta(-1) = -1/12, \ \ \zeta(-3) = 1/120, 
    \end{equation*}
    %
    etc., obtidos da fórmula, mas também dos valores
    %
    \begin{equation*}
        \zeta(2) = \pi^2/6, \zeta(4) = \pi^4/9, \dots, 
    \end{equation*}
    %
    e, em geral, 
    %
    \begin{equation}
        \label{zeta(2n)}
        \zeta(2n) = \frac{(2\pi)^{2n}(-1)^{n+1}B_{2n}}{2 \cdot (2n)!}
    \end{equation}
    %
    que foi encontrada por Euler \textcolor{red}{[E6]}. Não há maneira fácil deduzir esta famosa fórmula de Euler a partir da fórmula integral de Riemann \eqref{extensao-zeta-2} e pode ter sido o problema de provar \eqref{zeta(2n)} que levou Riemann à descoberta
    %
    \footnote{ Na verdade, a equação funcional aparece primeiro em um trabalho de Euler \textcolor{red}{[E7]} numa forma ligeiramente diferente e é inteiramente possível que Riemann a encontrou lá. (Veja também Hardy \textcolor{red}{[H5]}) De qualquer modo, Euler tinha apenas uma prova empírica da equação funcional e Riemann, invertendo os papeis, deu a primeira prova rigorosa de um fato enunciado, mas não provado adequadamente, por outra pessoa.
    }
    %
    da equação funcional da função zeta que é o assunto da próxima seção.
    
    
    \section{Primeira prova da equação funcional}
    
    
    Para valores reais negativos de $s$, Riemann calcula a integral
    %
    \begin{equation}
        \label{zeta-Pi-int}
        \zeta(s) = \frac{\Pi(-s)}{2\pi i} \int_{\infty}^{\infty} \frac{(-x)^s}{e^x - 1} \frac{dx}{x}
    \end{equation}
    %
    da seguinte maneira. Denote por $D$ o domínio do plano complexo que consiste dos pontos a uma distância menor que $\e$ do eixo real positivo ou a uma distância menor que $\e$ das singularidades $x = \pm 2 \pi i n$ do integrando de \eqref{zeta-Pi-int}. Seja $\partial D$ o bordo de $D$ orientado da maneira usual. Então, ignorando por um instante que $D$ não é compacto, temos, pelo Teorema de Cauchy, 
    %
    \begin{equation}
        \label{Pi-int-zero}
        \frac{\Pi(-s)}{2\pi i} \int_{\partial D} \frac{(-x)^s}{e^x - 1} \frac{dx}{x} = 0.
    \end{equation}
    %
    Uma componente desta integral é integral em \eqref{zeta-Pi-int} com a orientação invertida e as outras partes são integrais nos círculos $|x \pm 2\pi i n| = \e$ orientadas no sentido horário. Logo, quando os círculos são orientados no sentido anti-horário, \eqref{Pi-int-zero} nos dá
    %
    \begin{equation}
        \label{Pi-int-zero-2}
        -\zeta(s) - \sum \frac{\Pi(-s)}{2\pi i} \int_{|x \pm 2\pi i n| = \e} \frac{(-x)^s}{e^x - 1} \frac{dx}{x} = 0.
    \end{equation}
    %
    As integrais sobre os círculos podem ser calculadas definindo $x = 2\pi i n + y$ com $|y| = \e$
    %
    \begin{align*}
        \frac{\Pi(-s)}{2\pi i} \int_{|y| = \e} \frac{(-2 \pi i n - y)^s}{e^{2 \pi i n + y} - 1} \frac{dy}{2\pi i n + y} 
        &=
        %
        -\frac{\Pi(-s)}{2\pi i} \int_{|y| = \e} (-2 \pi i n - y)^{s-1} \frac{y}{e^y - 1} \frac{dy}{y} \\
        %
        &= - \Pi(-s)(-2\pi i n)^{s-1},
    \end{align*}
    %
    onde usamos a fórmula integral de Cauchy. Somando sobre os inteiros $n$ não nulos e usando \eqref{Pi-int-zero-2}, obtemos
    %
    \begin{align*}
        \zeta(s) &= \sum_{n=1}^{\infty} \Pi(-s)[(-2\pi i n)^{s-1} + (2\pi i n)^{s-1}] \\
        %
        &= \Pi(-s)(2\pi)^{s-1}[(i)^{s-1} + (-i)^{s-1}] \sum_{n=1}^{\infty} n^{s-1}.
    \end{align*}
    %
    Finalmente, usando que
    \begin{align*}
        (i)^{s-1} + (-i)^{s-1} & = \frac{1}{i}(e^{s \log (i)} - e^{s \log (-i)}) \\
        %
        &= \frac{1}{i}(e^{s \pi i/2} - e^{-s \pi i/2}) = 2 \sen \frac{s\pi}{2},
    \end{align*}
    %
    obtém-se a desejada fórmula
    %
    \begin{equation}
        \label{zeta-eq-funcional}
        \zeta(s) = \Pi(-s) (2\pi)^{s-1} 2 \sen (s\pi/2)\zeta(1-s).
    \end{equation}
    %
    Esta relação entre $\zeta(s)$ e $\zeta(1-s)$ é conhecida como a \textit{equação funcional da função zeta}.
    
    Para mostrar rigorosamente que \eqref{zeta-eq-funcional} é válida para $s<0$, é suficiente modificar o argumento acima definindo $D_n$ como a intercessão de $D$ com o disco $|s| \leq (2n+1)\pi$ e tomar o limite $n \to \infty$; então a integral  \eqref{Pi-int-zero} se separa em duas partes; uma sendo a integral sobre o círculo $|s| = (2n + 1)\pi$ com os pontos que distam menos que $\e$ do eixo real positivo deletados e a outra sendo uma integral cujo limite, conforme $n \to \infty$, é o lado esquerdo de \eqref{Pi-int-zero-2}. A primeira destas duas partes vai a zero, pois o tamanho do caminho de integração é menor que $2\pi (2n+1)\pi$, o fator $(e^x - 1)^{-1}$ é limitado no círculo $|s| = (2n + 1)\pi$ e o módulo de $(x)^s/x$ neste círculo é $|x|^{s-1} \leq [(2n + 1)\pi]^{-\delta -1}$ para $s \leq - \delta < 0$. Portanto, a segunda parte, que é o oposto da primeira pelo Teorema de Cauchy, também vai a zero, o que implica \eqref{Pi-int-zero-2} e, portanto, \eqref{zeta-eq-funcional}.
    
    Isto completa a prova da equação funcional \eqref{zeta-eq-funcional} no caso $s < 0$. No entanto, ambos os lados de \eqref{zeta-eq-funcional} são funções analíticas de $s$, então isto basta para provar \eqref{zeta-eq-funcional} para todos os valores de $s$ (exceto por $s = 0, 1, 2, \dots $, onde 
    %
    \footnote{ Quando $s = 2n + 1$, o fato de que $\zeta(s)$ não tem polo em $2n + 1$ implica que $\zeta(-2n) = 0$, uma vez que $\Pi$ tem um polo em $-2n-1$ e $\sen (s\pi/2)$ não tem zero em $2n + 1$. Portanto, pela fórmula para $\zeta(-2n)$ da seção anterior, os números de Bernoulli $B_3, B_5, \dots$ são todos nulos.
    }
    %
    um ou mais dos termos de \eqref{zeta-eq-funcional} têm polos).
    
    Para $s = 1-2n$, com a equação funcional e a identidade
    %
    \begin{equation*}
        \zeta(-(2n-1)) = (-1)^{2n-1} \frac{B_{2n}}{2n}
    \end{equation*}
    %
    da seção anterior, obtemos
    %
    \begin{equation*}
        (-1)^{2n-1} \frac{B_{2n}}{2n} = \Pi(2n-1) (2\pi)^{-2n} 2 (-1)^n \zeta(2n),
    \end{equation*}
    %
    donde segue a famosa fórmula de Euler para $\zeta(2n)$ (equação \eqref{zeta(2n)}).
    
    Riemann usa duas das identidades básicas da função fatorial (\eqref{Pi(s)-identidade-3} e \eqref{Pi(s)-identidade-4}) para reescrever a equação funcional \eqref{zeta-eq-funcional} na forma
    %
    \begin{equation*}
        \zeta(s) = \pi^{-1/2} 2^{-s} \Pi\left( -\frac{s}{2} \right) \Pi\left( -\frac{s+1}{2} \right) 2^{s} \pi^{s-1} \frac{\pi s / 2}{\Pi\left( \frac{s}{2} \right)\Pi\left( -\frac{s}{2} \right)} \zeta(1-s)
    \end{equation*}
    %
    e, portanto, na forma
    %
    \begin{equation}
        \label{zeta-eq-funcional-2}
        \Pi\left( \frac{s}{2} - 1 \right) \pi^{-s/2} \zeta(s) = \Pi\left( \frac{1 - s}{2} - 1 \right) \pi^{-(1 - s)/2} \zeta(1-s).
    \end{equation}
    %
    Em palavras, a função à esquerda de \eqref{zeta-eq-funcional-2} é invariante pela substituição $s = 1-s$.
    
    Riemann considera, aparentemente, esta expressão simétrica \eqref{zeta-eq-funcional-2} como a forma natural de enunciar a equação funcional, pois ele dá 
    %
    \footnote{Como a segunda prova torna a primeira totalmente desnecessária, isto leva ao questionamento da razão pela qual Riemann incluiu a primeira. Talvez a primeira prova mostra o argumento pelo qual ele originalmente descobriu a equação funcional ou talvez ela exiba propriedades que eram importantes segundo seu entendimento da $\zeta$
    }
    %
    uma prova alternativa que exibe esta simetria de uma maneira mais satisfatória. Mostramos esta segunda prova na próxima seção. 
    
    \section{Segunda prova da equação funcional}
    
    Riemann observa inicialmente que, com a substituição $x = n^2 \pi x$ na integral de Euler para $\Pi(s/2 - 1)$, obtém-se
    %
    \begin{equation*}
        \frac{1}{n^s} \pi^{-s/2} \Pi \left( \frac{s}{2} - 1 \right) = \int_{0}^{\infty} e^{-n^2 \pi x} x^{s/2} \frac{dx}{x} \ \ \ (\Re(s) > 1).
    \end{equation*}
    %
    Então a soma em $n$ implica
    %
    \begin{equation}
        \label{zeta-eq-funcional-3}
         \Pi \left( \frac{s}{2} - 1 \right) \pi^{-s/2} \zeta(s) = \int_{0}^{\infty} \psi(x) x^{s/2} \frac{dx}{x} \ \ \ (\Re(s) > 1).
    \end{equation}
    %
    onde 
    %
    \footnote{\textcolor{red}{Esta função $\psi(x)$ não tem relação com a função $\psi(x)$ que aparece no capítulo 3}
    }
    %
    $\psi(x) = \sum_{n=1}^{\infty} \exp(-n^2 \pi x)$. A forma simétrica da equação funcional é a afirmação de que a função \eqref{zeta-eq-funcional-3} é invariante pela substituição $s = 1 - s$. Para provar diretamente que a integral no lado direito de \eqref{zeta-eq-funcional-3} é invariante por esta substituição, Riemann usa a \textit{equação funcional da função theta} numa forma tirada de um trabalho de Jacobi, isto é,
    %
    \footnote{Riemann refere à seção 65 do tratado "Fundmenta Nova Theoriae Functionum Ellipticarum" de Jacobi. Mesmo que a fórmula não seja dada explicitamente, Jacobi, em um outro trabalho \textcolor{red}{[J1]} mostra como a fórmula procurada segue de (6) da seção 65. Jacobi atribui a fórmula a Poisson. \textcolor{red}{Para uma prova da fórmula veja a seção 10.4}.
    }
    %
    %
    \begin{equation}
        \label{psi-eq-funcional}
        \frac{1 + 2\psi(x)}{1 + 2\psi(1/x)} = \frac{1}{\sqrt{x}}.
    \end{equation}
    %
    (Como $\psi(x)$ se aproxima de zero muito rápido conforma $x \to \infty$, isto mostra, em particular, que $\psi(x)$ se comporta como $\frac{1}{2}(x^{-1/2} - 1)$ para $x$ próximo de zero e, portanto, que a integral ao lado direito de \eqref{zeta-eq-funcional-3} é convergente para $s>1$. A partir deste ponto, a validade de \eqref{zeta-eq-funcional-3} para $s>1$ pode ser mostrada por um argumento elementar usando convergência absoluta para justificar a troca entre a soma e a integral). Usando \eqref{psi-eq-funcional}, Riemann reescreve a integral do lado direito de \eqref{zeta-eq-funcional-3} como
    %
    \begin{align*}
        \int_{0}^{\infty} \psi(x) x^{s/2} \frac{dx}{x} &=
        %
        \int_{1}^{\infty} \psi(x) x^{s/2} \frac{dx}{x} - \int_{\infty}^{1} \psi \left( \frac{1}{x} \right) x^{-s/2} \frac{dx}{x} 
        \\[0.3cm]
        %
        &= \int_{1}^{\infty} \psi(x) x^{s/2} \frac{dx}{x} + \int_{1}^{\infty} \left[ x^{1/2} \psi(x) + \frac{x^{1/2}}{2} - \frac{1}{2} \right] x^{-s/2} \frac{dx}{x} 
        \\[0.3cm]
        %
        &= \int_{1}^{\infty} \psi(x) [x^{s/2} + x^{(1 - s)/2}] \frac{dx}{x} + \frac{1}{2} \int_{1}^{\infty} [x^{-(s-1)/2} - x^{- s/2}] \frac{dx}{x}
    \end{align*}
    %
    Mas $\int_{1}^{\infty} x^{-a}(dx/x) = 1/a$ para $a > 0$, então a segunda integral é
    %
    \begin{equation*}
        \frac{1}{2}\left[\frac{1}{(s-1)/2} - \frac{1}{s/2} \right] = \frac{1}{s(s-1)}
    \end{equation*}
    %
    para $s>1$. Então, para $s>1$, vale
    %
    \begin{equation}
        \label{zeta-eq-funcional-4}
        \Pi \left( \frac{s}{2} - 1 \right) \pi^{-s/2} \zeta(s) = \int_{1}^{\infty} \psi(x) [x^{s/2} + x^{(1 - s)/2}] \frac{dx}{x} - \frac{1}{s(s-1)}.
    \end{equation}
    %
    Mas, como $\psi(x)$ decresce mais rápido que qualquer potência de $x$ conforme $x \to \infty$, a integral nesta expressão converge para todo
    %
    \footnote{Observe que isto nos dá uma nova expressão para $\zeta(s)$ que é "válida para todo $s$" exceto $s = 0, 1$, ou seja, é uma prova alternativa de que $\zeta(s)$ pode ser continuada analiticamente.}
    %
    $s$. Uma vez que ambos os lados são analíticos, a mesma expressão vale para todo $s$. O lado direito é claramente invariante pela substituição $s = 1-s$, então está provada a equação funcional.
    
    \section{A função $\xi(s)$}
    
    A função $\Pi((s/2) - 1) \pi^{-s/2} \zeta(s)$, que ocorre na forma simétrica da equação funcional, tem polos $s=0$ e $s=1$ (isto segue imediatamente de \eqref{zeta-eq-funcional-4}). Riemann multiplica ela por $s(s - 1)/2$ e define
    %
    \footnote{Na verdade, Riemann usa $\xi$ para denotar a função que, atualmente, é usualmente denotada por $\Xi$, isto é, a função $\Xi(t) = \xi( \frac{1}{2} + it$ ) com $\xi$ como definida acima. Eu sigo Landau, e quase todos os escritores subsequentes, em rejeitar a mudança de variáveis $s = \frac{1}{2} + it$ por ser confusa. Há razões para acreditar que o próprio Riemann se confundia com ela (Veja as observações sobre $\xi(0)$ na Seção \ref{sec-1-16}).}
    %
    \begin{equation}
        \label{xi(s)-1}
        \xi(s) = \Pi(s/2) (s - 1) \pi^{-s/2} \zeta(s).
    \end{equation}
    %
    $\xi(s)$ é uma função inteira - ou seja, uma função analítica de $s$ que está definida para todos os valores de $s$ - e a equação funcional da função zeta é equivalente a $\xi(s) = \xi(1-s)$.
    
    Em seguida, Riemann obtém a seguinte representação de $\xi(s)$. Usando \eqref{zeta-eq-funcional-4}, tem-se 
    %
    \begin{align*}
        \xi(s) &= \frac{1}{2} - \frac{s(1-s)}{2} \int_{1}^{\infty} \psi(s) (x^{s/2} + x^{(1-s)/2}) \, \frac{dx}{x} \\[0.3cm]
        %
        &= \frac{1}{2} - \frac{s(1-s)}{2} \int_{1}^{\infty} \frac{d}{dx} \left\{ \psi(x) \left[ \frac{x^{s/2}}{s/2} + \frac{x^{(1-s)/2}}{(1-s)/2} \right] \right\} \, dx \\[0.3cm]
        %
        &+ \frac{s(1-s)}{2} \int_{1}^{\infty} \psi'(x) \left[ \frac{x^{s/2}}{s/2} + \frac{x^{(1-s)/2}}{(1-s)/2} \right]  \, dx \\[0.3cm]
        %
        &= \frac{1}{2} - \frac{s(1-s)}{2} \psi(1) \left[ \frac{2}{s} + \frac{2}{1-s} \right] \\[0.3cm]
        %
        &+ \int_{1}^{\infty} \psi'(x) \left[ (1-s) x^{s/2} + s x^{(1-s)/2} \right]  \, dx \\[0.3cm]
        %
        &= \frac{1}{2} + \psi(1) + \int_{1}^{\infty} x^{3/2} \psi'(x) \left[(1-s) x^{(s-1)/2} + s x^{-(s/2) - 1} \right] \, dx \\[0.3cm]
        %
        &= \frac{1}{2} + \psi(1) + \int_{1}^{\infty} \frac{d}{dx} [x^{3/2} \psi'(x) (-2 x^{(s-1)/2} -2 x^{-(s/2)} )] \, dx \\[0.3cm]
        %
        &- \int_{1}^{\infty} \frac{d}{dx} [x^{3/2} \psi'(x)] [-2 x^{(s-1)/2} -2 x^{-(s/2)} ] \, dx \\[0.3cm]
        %
        &= \frac{1}{2} + \psi(1) + \psi'(1)[-2 -2] \\[0.3cm]
        %
        &+ \int_{1}^{\infty} \frac{d}{dx} [x^{3/2} \psi'(x)] (2 x^{(s-1)/2} -+ 2 x^{-(s/2)} ) \, dx.
    \end{align*}
    %
    Agora, a derivada de 
    %
    \begin{equation*}
        2 \psi(x) + 1 = x^{-1/2}[ 2\psi(1/x) + 1]
    \end{equation*}
    %
    nos dá
    %
    \begin{equation*}
        \frac{1}{2} + \psi(1) + 4\psi'(1) = 0
    \end{equation*}
    %
    e, usando este fato, obtemos a forma final da expressão de $\xi$
    %
    \begin{equation}
        \label{xi(s)-2}
        \xi(s) = 4\int_{1}^{\infty} \frac{d[x^{3/2}\psi'(x)]}{dx} x^{-1/4} \cosh \left[ \frac{1}{2} \left( s - \frac{1}{2} \right) \log x \right] \, dx,
    \end{equation}
    %
    ou, na forma em que Riemann escreve,
    %
    \begin{equation*}
        \label{xi(s)-2}
        \xi \left( \frac{1}{2} + it \right) = 4\int_{1}^{\infty} \frac{d[x^{3/2}\psi'(x)]}{dx} x^{-1/4} \cos \left( \frac{t}{2} \log x \right) \, dx.
    \end{equation*}
    %
    Se $\cosh \left[ \frac{1}{2} \left( s - \frac{1}{2} \right) \log x \right]$ é expandido em série de potências $\cosh y = \frac{1}{2}(e^y + e^{-y}) =  \sum y^{2n}/(2n)!$, a expressão \eqref{xi(s)-2} nos mostra que
    %
    \begin{equation}
        \label{xi(s)-3}
        \xi(s) = \sum_{n=0}^{\infty} a_{2n} \left(s - \frac{1}{2} \right)^{2n},
    \end{equation}
    %
    onde
    %
    \begin{equation*}
        a_{2n} = 4\int_{1}^{\infty} \frac{d[x^{3/2}\psi'(x)]}{dx} x^{-1/4} \frac{\left( \frac{1}{2} \log x \right)^{2n}}{(2n)!} \, dx.
    \end{equation*}
    %
    Riemann enuncia que esta representação em série de $\xi(s)$ é uma função par de $s - \frac{1}{2}$ que "converge muito rapidamente", mas ele não dá estimativas explícitas nem fala qual a importância desta série nas afirmações que faz em seguida. 
    
    Os dois parágrafos seguintes à expressão \eqref{xi(s)-2} para $\xi(s)$ compõem a parte mais difícil do artigo de Riemann. O objetivo é, essencialmente, provar que $\xi(s)$ pode ser expandida como um produtório infinito
    %
    \begin{equation}
        \label{xi(s)-4}
        \xi(s) = \xi(0) \prod_{\rho} \left( 1 - \frac{s}{\rho}\right),
    \end{equation}
    %
    onde $\rho$ varia
    %
    \footnote{As raízes múltiplas (se existirem) devem ser contadas com multiplicidade. O mesmo vale para quaisquer expressões (produtórios ou somas) que variam sobre raízes $\rho$.}
    %
    sobre as raízes da equação $\xi(\rho) = 0$. Qualquer polinômio $p(s)$ pode ser expandido como um produto finito $p(s) = p(0) \prod_{\rho}[1 - (s/\rho)]$ onde $\rho$ varia sobre as raízes da equação $p(\rho) = 0$ (exceto que, quando $p(0) = 0$, a fórmula é ligeiramente diferente), então a expressão \eqref{xi(s)-4} significa que $\xi(s)$ \textit{se comporta como um polinômio de grau infinito} (de forma parecida, Euler pensava em $\sen x$ como um "polinômio de grau infinito" quando ele conjecturou, e provou, a fórmula $\sen \pi x = \pi x \prod_{n=1}^{\infty} [1 - (x/n)^2])$. por outro lado, a afirmação de que a série \eqref{xi(s)-3} converge "muito rapidamente" também significa que $\xi(s)$ é como um polinômio de grau infinito - um número finito de termos da série dá uma aproximação muito boa em qualquer parte finita do plano. Então há alguma relação entre a série \eqref{xi(s)-3} e a fórmula do produto \eqref{xi(s)-4} - de fato, o rápido decrescimento dos coeficientes $a_n$ é uma condição necessária e suficiente para a validade da fórmula do produto, como provado por Hadamard em 1893 - mas os passos que Riemann seguiu para passar de uma para a outra são obscuros, para dizer o mínimo.
    
    A próxima Seção contém uma discussão sobre a distribuição das raízes $\rho$ de $\xi(\rho) = 0$ e a seção seguinte a ela retorna à discussão sobre a formula do produto para $\xi(s)$.
    
    
    \section{As ráizes $\rho$ de $\xi$}
    
    
    Para provar a convergência do produtório $\xi(s) = \xi(0) \prod_{\rho} [1 - (s/\rho)]$, Riemann precisava, claro, investigar a distribuição das raízes $\rho$ de $\xi(\rho) = 0$. Ele começa observando que a fórmula do produto de Euler 
    %
    \begin{equation*}
        \zeta(s) = \prod_{p} (1 - p^{-s})^{-1} \ \ \ \ (\Re(s) > 1) 
    \end{equation*}
    %
    mostra que $\zeta(s)$ não tem zeros no semiplano $\Re(s) > 1$ (pois um produto infinito convergente só pode ser zero se um dos seus fatores é zero). Como $\xi(s) = \Pi(s/2) (s-1) \pi^{-s/2} \zeta(s)$ e os fatores distintos de $\zeta(s)$ têm apenas o zero $s = 1$, segue que nenhuma raiz $\rho$ de $\xi(\rho) = 0$ está no semiplano $\Re(s) > 1$. Como $1 - \rho$ é uma raiz se, e somente se, $\rho$ é, isto implica que não há nenhuma raiz no semiplano $\Re(s) < 0$ e, portanto, que todas as raízes estão na faixa $0 \leq \Re(s) \leq 1$.
    
    Ele, então, procede e afirma que o número de raízes $\rho$ cuja parte imaginária está entre 0 e $T$ é, aproximadamente,
    %
    \begin{equation}
        \label{sec-1.9-(1)}
        \frac{T}{2 \pi} \log \frac{T}{2 \pi} - \frac{T}{2 \pi}
    \end{equation}
    %
    e que o erro relativo
    %
    \footnote{Titchmarsh, num deslize que não percebeu no intervalo de 21 anos entre as publicações de seus dois livros sobre a função zeta, não percebeu que Riemann queria dizer o erro \textit{relativo} e acreditava que Riemann havia cometido um erro neste ponto. Veja Titchmarsh \textcolor{red}{[T8, pág. 213]}.}
    %
    nesta aproximação é da ordem $1/T$. Sua "prova" desta afirmação é simplesmente dizer que o número de raízes nesta região é igual à integral de $\xi'(s) ds / \xi(s) 2\pi i$ ao redor do bordo do retângulo $\{0 \leq \Re(s) \leq 1, 0 \leq \Im(s) \leq T \}$ e que esta integral é igual a \eqref{sec-1.9-(1)} com erro relativo $T^{-1}$. Infelizmente, ele não deu dica alguma do método que usado para estimar esta integral. Ele próprio era um mestre em calcular e estimar integrais definidas \textcolor{red}{(veja, por exemplo, Seção 1.14 ou Seção 7.4)} e é bem possível que ele assumiu que seus leitores seriam capazes de obter suas próprias estimativas desta integral; se foi isso, ele estava errado. Apenas em 1905 é que von Mangoldt conseguiu provar que a estimativa de Riemann estava correta \textcolor{red}{(Veja Seção 6.7)}.
    
    A próxima afirmação de Riemann é ainda mais desconcertante. Ele diz que o número de raízes na reta $\Re(s) = \frac{1}{2}$ é aproximadamente \eqref{sec-1.9-(1)}. Ele não torna preciso o sentido em que esta aproximação é verdadeira, mas é geralmente assumido que ele quis dizer que o erro relativo da aproximação do número de zeros de $\xi(\frac{1}{2} + it)$ para $0 \leq t \leq T$ por \eqref{sec-1.9-(1)} se aproxima de zero quando $T \to \infty$. Ele não dá qualquer indicação de prova de maneira alguma e ninguém, desde Riemann, foi capaz de provar ou mostrar a falsidade desta afirmação. Foi provado em 1914 que $\xi(\frac{1}{2} + it)$ tem infinitas raízes reais \textcolor{red}{(Hardy [H3])}; em 1921, que o número de raízes reais entre 0 e $T$ é, pelo menos, $KT$ para alguma constante positiva $K$ e $T$ suficientemente grande \textcolor{red}{(Hardy e Littlewood [H6])}; em 1942, que este número é, na verdade, $KT \log T$ para algum $K$ positivo e $T$ grande \textcolor{red}{(Selberg [S1])} e, em 1914, que o número de raízes complexas $t$ de $\xi(\frac{1}{2} + i) = 0$ no conjunto $\{0 \leq \Re(t) \leq T, -\e \leq \Im(t) \leq \e \}$ é igual, para qualquer $\e > 0$, a \eqref{sec-1.9-(1)} com um erro relativo que se aproxima de zero conforma $T \to \infty$ \textcolor{red}{(Bohr e Landau, [B8])}. No entanto, estes resultados parciais ainda estão longe da aafirmação feita por Riemann. Nós podemos apenas conjecturar o que está por trás desta afirmação \textcolor{red}{(Veja Siegel [S4, pág. 67], TitchMarshBohr [T8 pág. 213-214] ou Seção 7.8)}, mas sabemos que ela levou Riemann a conjecturar algo ainda mais forte, que todas as raízes estão em $\Re(s) = \frac{1}{2}$.
    
    Esta é, claro, a famosa "Hipótese de Riemann". Ele diz que considera "muito provável" que todas as raízes estejam na reta $\Re(s) = \frac{1}{2}$, mas diz que não foi capaz de provar isto (o que parece implicar que ele sentia que tinha provas rigorosas das duas afirmações precedentes). Como não é necessário para seu objetivo principal, que é a prova de sua fórmula para o número de primos menores que uma certa quantidade dada, ele simplesmente deixa de lado este assunto - onde permanece desde então - e continua para a fórmula do produto para $\xi(s)$.

    
    \section{A representação em produto de $\xi(s)$}
    
    
    Um tema recorrente no trabalho de Riemann é a \textit{caracterização global de funções analíticas por suas singularidades}
    %
    \footnote{Veja, por exemplo, a Inalgural Dissertation, especialmente o artigo 20 \textcolor{red}{(Werke, pág. 37-39)} ou a parte 3 da introdução do artigo "Theorie de Abel'Schen Functionen", que tem o título "Determination of a fnction of a complex variable by boundary values and singularities" \textcolor{red}{[R1]}. Veja também a introdução de Riemann ao Paper XI do \textit{collected works}, onde ele escreve "... our method, wich is based on the determination of functions by means of their singuatities \textit{(Unstetigkeiten und Unendlichwerden)}... \textcolor{red}{[R1]}. Por fim, veja Ahlfors \textcolor{red}{[A3]}, a seção no fim entitulada "Riemann's point of view".}
    %
    . Como a função $\log \xi(s)$ tem singularidades logarítmicas nas raízes $\rho$ de $\xi(s)$ e nenhuma outra singularidade, ela tem as mesmas singularidades que a soma formal
    %
    \begin{equation}
        \label{sec-1-10-(1)}
        \sum_{\rho} \log \left( 1 - \frac{s}{\rho} \right).
    \end{equation}
    %
    Então, se esta soma converge e a função definida por ela é, em algum sentido tão bem comportada perto de $\infty$ como $\log \xi(s)$ é, então isto deve significar que a soma \eqref{sec-1-10-(1)} deve diferir de $\log \xi(s)$ por, no máximo, uma constante aditiva. Colocando $s = 0$, obtemos o valor $\log \xi(0)$ para esta constante e, portanto, tomando a exponencial, temos
    %
    \begin{equation}
        \label{sec-1-10-(2)}
        \xi(s) = \xi(0) \prod_{\rho} \left( 1 - \frac{s}{\rho} \right)
    \end{equation}
    %
    como desejado. Este é, essencialmente, o outline 
    da prova da fórmula \eqref{sec-1-10-(2)} feito por Riemann.
    
    Existem dois problemas associados com a soma \eqref{sec-1-10-(1)}. 
    O primeiro é a determinação das partes imaginárias dos logaritmos que ela contém. 
    Riemann passa por este ponto sem dar muitos comentários e, de fato, 
    este não é um problema muito sério. Para qualquer $s$ fixado, 
    a ambiguidade na parte imaginária de $\log [1 - (s / \rho)]$ 
    desaparece para $\rho$ suficientemente grande; então a soma 
    \eqref{sec-1-10-(1)} está definida exceto por um múltiplo 
    finito de $2 \pi i$, que se torna irrelevante quando se 
    exponencia \eqref{sec-1-10-(2)}. Em suma, pode-se ignorar as 
    partes imaginárias de uma vez só. As partes reais dos termos 
    de \eqref{sec-1-10-(1)} não trazem qualquer ambiguidade e sua 
    soma é uma função harmônica que difere de $\Re(\log \xi(s))$ 
    por uma função harmônica sem singularidades e, se podemos 
    mostrar que esta função diferença é constante, isto 
    acarretará que sua harmônica conjugada também é constante. 
    
    O segundo problema associado com a soma 
    \eqref{sec-1-10-(1)} é sua convergência. 
    Ela é, na verdade, uma soma condicionalmente convergente e a 
    ordem dos termos da série deve ser especificada para que a 
    soma esteja bem determinada. De modo geral, a ordem natural 
    seria aquela de $|\rho|$ crescente ou, talvez, 
    de $|\rho - \frac{1}{2}|$ crescente. 
    Mas é suficiente simplesmente associar a cada 
    $\rho$ o seu par $\rho  \leftrightarrow 1 - \rho$
    %
    \begin{equation}
        \label{sec-1-10-(3)}
        \sum_{\Im(\rho) > 0} 
        \left[ \log \left( 1 - \frac{s}{\rho} \right) + \log \left( 1 - \frac{s}{1-\rho} \right) \right],
    \end{equation}
    %
    pois esta soma converge absolutamente. 
    A prova da convergência absoluta de \eqref{sec-1-10-(3)} é, 
    de maneira simplificada, como segue.
    
    Para provar a convergência absoluta de
    %
    \begin{equation*}
        \sum_{\Im(\rho) > 0} \left[ \log \left( 1 - \frac{s}{\rho} \right) \left( 1 - \frac{s}{1-\rho} \right) \right]
        %
        = \sum_{\Im(\rho) > 0} \log \left[ \left( 1 - \frac{s(1-s)}{\rho(1-\rho)} \right) \right]
    \end{equation*}
    %
    é suficiente provar a convergência absoluta de 
    %
    \begin{equation}
        \label{sec-1-10-(4)}
        \sum_{\Im(\rho) > 0} 
        \frac{1}{\rho(1-\rho)}
    \end{equation}
    %
    (em outras palavras, para provar a convergência absoluta de um 
    produto $\prod (1 + a_i)$, é suficiente provar a convergência 
    absoluta da soma $\sum a_i$). Mas as estimativas da distribuição das 
    raízes $\rho$ da seção anterior indicam que a densidade é, na prática,
    %
    \begin{equation*}
        d \left( \frac{T}{2\pi} \log \frac{T}{2\pi} - \frac{T}{2\pi} \right) = \frac{1}{2\pi} \log \frac{T}{2\pi} dT.
    \end{equation*}
    %
    Portanto, 
     %
    \begin{equation*}
        \sum_{\Im(\rho) > 0} \frac{1}{\rho(1-\rho)} \sim \int^{\infty} \frac{1}{T^2}\frac{1}{2\pi} \log \frac{T}{2\pi} \, dT < \infty
    \end{equation*}
    %
    ou, resumindo, os termos se comportam como $T^{-2}$ e sua densidade se comporta como $\log T$, então a soma converge. Como vai ser visto no \textcolor{red}{capítulo 2}, o passo mais difícil para transformar isto numa prova rigorosa da convergência absoluta de \eqref{sec-1-10-(3)} é mostrar que a densidade vertical das raízes $\rho$ é, em algum sentido, uma constante multiplicada por $\log(T/2\pi)$. Riemann enuncia este fato sem demonstrá-lo.
    
    Riemann continua e diz que a função definida em \eqref{sec-1-10-(3)} cresce apenas tão rápido quanto $s \log s$ para $s$ suficientemente grande; então, como ela difere de $\log \xi(s)$ apenas por uma função par de $s - \frac{1}{2}$ (e também porque $\log \xi(s)$ cresce como $s \log s$ para $s$ grande), esta diferença deve ser constante, já que não pode ter termos da forma $(s - \frac{1}{2})^2$, $(s - \frac{1}{2})^4$, $\dots$. Vai ser mostrado no \textcolor{red}{capítulo 2} que os passos neste argumento podem ser completados mais ou menos como Riemann indica, mas devemos admitir que o outline da prova dado por Riemann é tão breve que quase o torna inútil para construir uma prova de \eqref{sec-1-10-(2)}.
    
    A primeira prova da representação em produto \eqref{sec-1-10-(2)} de $\xi(s)$ foi publicada por Hadamard \textcolor{red}{[H1]} em 1893.
    
    
    \section{A conexão entre $\zeta(s)$ e os números primos}
    
    
    A essência da relação de $\zeta(s)$ com os números primos está na fórmla do produto de Euler
    %
    \begin{equation}
        \label{sec-1-11-(1)}
        \zeta(s) = \prod_{p} \frac{1}{1 - p^{-s}} \ \ \ \ (\Re(s)>1)
    \end{equation}
    %
    na qual o produto à direita é sobre todos os números primos. Tomando o $\log$ de ambos os lados e usando a série $\log(1-x) = -x - \frac{1}{2}x^2 - \frac{1}{3}x^3 - \cdots$ coloca esta expressão na forma
    %
    \begin{equation*}
        \log \zeta(s) = \sum_{p} \left[ \sum_n (1/n)p^{-ns} \right] \ \ \ \ (\Re(s)>1).
    \end{equation*}
    %
    Como a soma dupla à direita é absolutamente convergente, a ordem em que se soma não importa, então esta expressão pode ser escrita de forma mais simples como
    %
    \begin{equation}
        \label{sec-1-11-(2)}
        \log \zeta(s) = \sum_{p} \sum_n (1/n)p^{-ns} \ \ \ \ (\Re(s)>1).
    \end{equation}
    %
    Vai ser conveniente no que vem a seguir escrever esta soma como uma integral de Stieltjes
    %
    \begin{equation}
        \label{sec-1-11-(3)}
        \log \zeta(s) = \int_{0}^{\infty} x^{-s} \, dJ(x) \ \ \ \ (\Re(s)>1)
    \end{equation}
    %
    onde $J(x)$ é
    %
    \footnote{Riemann denota esta função por $f(x)$ e a maioria dos outros autores denota ela por $\Pi(x)$. Como $f(x)$ é, atualmente, usado para denotar uma função genérica e, neste livro, $\Pi(x)$ denota a função fatorial, eu tomei a liberdade de introduzir uma nova notação $J(x)$ para esta função.}
    %
    a função que começa em 0 para $x = 0$ e aumenta de um salto de uma unidade nos números primos, de um salto de $\frac{1}{2}$ nos quadrados de primos $p^2$, de um salto de $\frac{1}{3}$ nos cubos de primos $p^3$, e assim por diante. Como é usual na teoria de integrais de Stieltjes, o valorde $J(x)$ em cada salto é definido como a média entre seu valor antigo e seu novo valor. Então $J(x)$ é nula para $0 \leq x < 2$, é $\frac{1}{2}$ para $x = 2$, é 1 para $2 < x < 3$, é $1 + \frac{1}{2}$ para $x = 3$, é 2 para $3 < x < 4$, é $2 + \frac{1}{4}$ para $x = 4$, é $2 + \frac{1}{2}$ para $4 < x < 5$, é 3 para $x = 5$, é $3 + \frac{1}{2}$ para $5 < x < 7$ e assim por diante. Uma fórmula para $J(x)$ é 
    %
    \begin{equation*}
        J(x) = \frac{1}{2} \left[ \sum_{p^n < x} \frac{1}{n} + \sum_{p^n \leq x} \frac{1}{n} \right].
    \end{equation*}
    %
    
    Riemann, claro, não tinha o vocabulário da integração de Stieltjes disponível para ele. Ele enunciou \eqref{sec-1-11-(3)} numa forma ligeiramente diferente
    %
    \begin{equation}
        \label{sec-1-11-(4)}
        \log \zeta(s) = s\int_{0}^{\infty} J(x)x^{-s-1} \, dx \ \ \ \ (\Re(s)>1)
    \end{equation}
    %
    que pode ser obtida de \eqref{sec-1-11-(3)} usando integração por partes. (Conforme $x \downarrow 0$, temos $x^{-s}J(x) = 0$, pois $J(x) \equiv 0$ para $x<2$. Por outro lado, $J(x) < x$ para todo $x$, então $x^{-s}J(x) \to 0$ quando $x \to \infty$ para $\Re(s) > 1$). A integral em \eqref{sec-1-11-(4)} pode ser considerada uma integral de Riemann usual e a expressão em si pode ser obtida sem o uso da integração de Stieltjes se substituímos
    %
    \begin{equation*}
        \label{sec-1-11-(4)}
        p^{-ns} = s\int_{p^n}^{\infty} x^{-s-1} \, dx \ \ \ \ (\Re(s)>1)
    \end{equation*}
    %
    em \eqref{sec-1-11-(2)}, que é a forma como Riemann obtém \eqref{sec-1-11-(4)}.
    
    As expressões \eqref{sec-1-11-(2)} - \eqref{sec-1-11-(4)} devem ser pensadas como variações pequenas da fórmula do produto de Euler \eqref{sec-1-11-(1)} que é a ideia básica que conecta $\zeta(s)$ e os números primos.
    
    
    \section{Inversão de Fourier}
    
    
    Riemann foi um mestre da análise de Fourier e seu trabalho no desenvolvimento desta teoria deve certamente ser contado entre suas grandes contribuições para a matemática. Portanto, não é surpreendente que ele tenha aplicado inversão de fourier à expressão
    %
    \begin{equation}
        \label{sec-1-12-(1)} 
        \frac{\log \zeta(s)}{s} = \int_{0}^{\infty} J(x) x^{-s-1} \, dx \ \ \ \ (\Re(s) > 1)
    \end{equation}
    %
    para concluir que 
    %
    \begin{equation}
        \label{sec-1-12-(2)} 
        J(x) = \frac{1}{2\pi i} \int_{a - i \infty}^{a + i\infty} \log \zeta(s) x^{s} \frac{ds}{s} \ \ \ \ (a > 1).
    \end{equation}
    %
    Então, usando uma fórmula alternativa para $\log \zeta(s)$, ele obtém uma fórmula alternativa para $J(x)$ que é o resultado principal do artigo.
    
    (A integral imprópria \eqref{sec-1-12-(2)} é apenas condicionalmente convergente, então uma ordem para a soma deve ser especificada. Aqui, a integra em \eqref{sec-1-12-(2)} é o limite, quando $T \to \infty$, da integral sobre o segmento vertical de $a - iT$ até $a + iT$. De forma mais geral, integrais e séries condicionalmente convergentes são muito comuns em análise de Fourier e é sempre entendido que tais integrais e séries são somadas em sua "ordem natural". Por exemplo
    %
    \begin{align*}
        &\sum_{n=-\infty}^{\infty} c_n e^{inx} &&\text{significa} &&&\lim_{N \to \infty} \sum_{n=-N}^{N} c_n e^{inx} \\
        %
        &\int_{-\infty}^{\infty} f(y)e^{iyx} \, dy &&\text{significa} &&&\lim_{T \to \infty} \int_{-T}^{T} f(y)e^{iyx} \, dy.
    \end{align*}
    %
    Isto é análogo à convenção de que funções descontínuas como $J(x)$ assumem o valor intermediário $J(x) = \frac{1}{2}[J(x-\e) + J(x+\e)]$ em qualquer salto $x$, que integrais divergentes como $\text{Li}(x)$ (veja Seção \ref{sec-1-14} abaixo) são tomadas como o valor principal de Cauchy, e que o produto $\prod [1 - (s/\rho)]$ é ordenada de modo a parear $\rho$ e $1 - \rho$ ou, como mais à frente, ordenada por $|\lim (\rho)|$.)
    
    Ao obter \eqref{sec-1-12-(2)} a partir de \eqref{sec-1-12-(1)}, Riemann faz uso do "Teorema de Fourier" que, para ele, é
    %
    \footnote{Veja \textcolor{red}{Riemann [R2,pág. 86]}}
    %
    a fórmula da inversão de Fourier
    %
    \begin{equation}
        \label{sec-1-12-(3)} 
        \phi(x) = \frac{1}{2\pi} \int_{-\infty}^{\infty} \left[ \int_{-\infty}^{\infty} \phi(\lambda) e^{i(x-\lambda) \mu} \, d\lambda \right] \, d\mu.
    \end{equation}
    %
    Enunciado de outra forma, o "Teorema de Fourier" diz que, para escrever uma dada função em termos de uma superposição de exponenciais
    %
    \begin{equation*}
        \phi(x) =  \int_{-\infty}^{\infty}  \Phi(\mu) e^{i\mu x} \, d\mu,
    \end{equation*}
    %
    é necessário e suficiente (sob certas condições) que os coeficientes $\Phi(\mu)$ sejam definidos por
    %
    \begin{equation*}
        \Phi(x) =  \frac{1}{2\pi} \int_{-\infty}^{\infty}  \phi(\lambda) e^{-i\lambda \mu} \, d\lambda.
    \end{equation*}
    %
    O enunciado do Teorema de Fourier traz à tona a analogia com as séries de Fourier
    %
    \begin{equation*}
        f(x) = \sum_{-\infty}^{\infty} a_n e^{inx}  \Longleftrightarrow a_n = \frac{1}{2\pi} \int_{0}^{2\pi}  f(\lambda) e^{-in\lambda} \, d\lambda
    \end{equation*}
    %
    e, de fato, o Teorema \eqref{sec-1-12-(3)} para integrais de Fourier segue de uma passagem ao limite no teorema para séries de Fourier.
    
    Para obter \eqref{sec-1-12-(2)} a partir de \eqref{sec-1-12-(1)}, seja $s = a + i\mu$, onde $a>1$ é uma constante e $\mu$ é uma variável, sejam $\lambda = \log x$ e $\phi(x) = 2 \pi J(e^x) e^{-ax}$. Então \eqref{sec-1-12-(1)} se torna
    %
    \begin{align*}
        \frac{\log \zeta(a + i\mu)}{a + i\mu} &= \int_{-\infty}^{\infty} J(e^{\lambda}) e^{-(a + i\mu)\lambda} \, d\lambda \\
        %
        &= \frac{1}{2\pi} \int_{-\infty}^{\infty} \phi(\lambda) e^{-i\mu\lambda} \, d\lambda \ \ \ \ (a > 1).
    \end{align*}
    %
    Quando esta função é colocada no lugar de $\Phi(\mu)$, o Teorema de Fourier implica que
    %
    \begin{align*}
        2\pi J(e^x) e^{-ax} &= \int_{-\infty}^{\infty} \frac{\log \zeta(a + i\mu)}{a + i\mu} e^{i \mu x} \, d\mu, \\
        %
        J(y) &= \frac{1}{2\pi} \int_{-\infty}^{\infty} \frac{\log \zeta(a + i\mu)}{a + i\mu} y^{a + i\mu} \, d\mu,
    \end{align*}
    %
    dos quais \eqref{sec-1-12-(2)} segue imediatamente.
    
    Riemann ignora completamente a questão da aplicabilidade do Teorema de Fourier à função $J(e^x)e^{-ax}$ e simplesmente enuncia que \eqref{sec-1-12-(2)} vale de forma "completamente geral". No entanto, $J(e^x)e^{-ax}$ é uma função muito bem comportada - ela tem saltos simples e bem comportados, é identicamente nula para $x < 0$, e vai a zero mais rápido que $e^{-(1-a)x}$, conforme $x \to \infty$ - e os teoremas mais simples sobre integrais de Fourier são suficientes para mostrar rigorosamente a afirmação de Riemann de que \eqref{sec-1-12-(2)} vale de forma completamente geral.
    
    
    
    
    
    
    
    
    
    \section{A} \label{sec-1-14}
    \section{Os termos remanescentes} \label{sec-1-16}
    
    
    
    
    
    
    
    
    
    
    
    
    
    
    
    
%regaça 
%\o/













