Introdução do texto

Colocar a tradução do Cap1 do Edwards

    % \begin{center}
    %     \textbf{Capítulo 1: O artigo de Riemann}
    % \end{center}
    \section*{Contexto Histórico}

    Este livro é um estudo do artigo de 8 páginas ``On the number of primes less than a given magnitude''\footnote{The german title is Ueber die Ahzahl der Primzahlen unter einer gegebenen Grösse.} de Bernhard Riemann que marcou época e dos subsequentes desenvolvimentos da teoria que este artigo inaugura. O primeiro capítulo é uma análise do artigo em si e os 11 capítulos restantes são voltados para o trabalho que foi feito, desde 1859, sobre as questões que Riemann deixou sem resposta.

    A teoria para a qual o artigo de Riemann é uma contribuição tem seu início no Teorema de Euler, provado em 1737, da divergência da soma dos recíprocos dos números primos
    %
    \begin{equation}
    \label{Soma-Recip-primos}
    \frac{1}{2} + \frac{1}{3} + \frac{1}{5} + \frac{1}{7} + \frac{1}{11} + \frac{1}{13} + \frac{1}{17} + \cdots.
    \end{equation}
    %
    Este teorema vai além do clássico \textcolor{red}{(antigo)} Teorema de Euclides, sobre a infinitude dos números primos, e mostra que os números primos são densos no conjunto dos números inteiros - mais densos que os quadrados, por exemplo, no sentido de que a soma dos recíprocos dos quadrados converge.
    
    Euler vai além de simplesmente enunciar a divergência de \eqref{Soma-Recip-primos} e diz que, como $1 + \frac{1}{2} + \frac{1}{3} + \frac{1}{4} + \cdots$ diverge como o logaritmo e (\ref{Soma-Recip-primos}) diverge como o logaritmo de
    \footnote
    {Isto é verdade por conta da fórmula de Euler $\sum(1/n) = \prod (1-p^{-1})^{-1}$, então $\log (\sum(1/n)) = - \log(\prod(1-p^{-1})^{-1}) = -\log(\prod(1-p^{-1})) = -\sum (p^{-1} + \frac{1}{2}p^{-2} + \frac{1}{3}p^{-3} + \cdots ) = \sum(1/p) + \text{ convergente}.$}
    $1 + \frac{1}{2} + \frac{1}{3} + \frac{1}{4} + \cdots$, a série \eqref{Soma-Recip-primos} deve divergir como o $\log$ do $\log$, que Euler escreve como 
    %
    \begin{equation}
    \label{Soma-Recip-primos-log}
    \frac{1}{2} + \frac{1}{3} + \frac{1}{5} + \frac{1}{7} + \frac{1}{11} + \frac{1}{13} + \frac{1}{17} + \cdots = \log(\log \infty).
    \end{equation}
    %
    Não é claro o que esta equação significava para Euler - isto se a entendia como algo além de algo \textcolor{red}{mnemônico} - mas uma interpretação óbvia dela poderia ser
    %
    \begin{equation}
    \label{Soma-Recip-primos-x}
    \sum_{p<x}\frac{1}{p} \approx \log(\log x) \ \ \ \ \ \ (x \to \infty),
    \end{equation}
    %
    onde o lado esquerdo denota a soma de $1/p$ sobre todos os primos $p$ menores que $x$ e o símbolo $\approx$ significa que o erro relativo é arbitrariamente pequeno para $x$ suficientemente grande ou, o que é a mesma coisa, que a razão entre os dois lados se aproxima de $1$ conforme $x \to \infty$. Agora,
    %
    $$\log(\log x) = \int_{1}^{\log x} \frac{du}{u} = \int_{e}^{x} \frac{dv}{v\log v},$$
    %
    então \eqref{Soma-Recip-primos-x} significa que a integral de $1/v$ relativa à métrica $dv/\log v$ diverge da mesma maneira que a integral de $1/v$ relativa à \textcolor{red}{métrica discreta} que dá peso 1 para os primos e peso 0 para todos os outros pontos. Neste sentido, \eqref{Soma-Recip-primos-x} tem a interpretação de que a densidade dos primos é, essencialmente, $1/\log v$. No entanto, não há evidências de que Euler tenha pensado na densidade dos números primos e seus métodos eram inadequados para provar a formulação \eqref{Soma-Recip-primos-x} do que ele enuncia em \eqref{Soma-Recip-primos-log}.
    
    Gauss afirma, em uma carta escrita em 1849, que ele havia observado por volta de 1792 ou 1793 que a densidade dos primos aparenta ser, em média, $1/\log x$ e diz que cada nova tabela de primos publicada em anos subsequentes vinha a confirmar sua crença da precisão desta aproximação. No entanto, ele não menciona a fórmula de Euler \eqref{Soma-Recip-primos-log} e não dá nenhuma base analítica para esta aproximação, que é apresentada somente com base em observações empíricas. Ele mostra a Tabela \ref{Tabela-Gauss}.
    %
    \begin{table}[H]
        \centering
        \begin{tabular}{cccc}
             $x$& número de primos $< x$ & $\displaystyle \int \frac{dn}{\log n}$ &  Diferença
             \\[0.45cm]
             \hline
             \\[-0.3cm]
             500.000   & 41.556  & 41.606,4  & 50,4 \\[0.1cm]
             1.000.000 & 78.501  & 78.627,5  & 126,5 \\[0.1cm]
             1.500.000 & 114.112 & 114.263,1 & 151,1 \\[0.1cm]
             2.000.000 & 148.883 & 149.054,8 & 171,8 \\[0.1cm]
             2.500.000 & 183.016 & 183.245,0 & 229,0 \\[0.1cm]
             3.000.000 & 216.745 & 216.970,6 & 225,6
        \end{tabular}
        \caption{}
        \label{Tabela-Gauss}
    \end{table}
    %
    
    Gauss não diz exatamente o que significa o símbolo $\int (dn/\log n)$, mas os dados na Tabela \ref{Tabela-Lehmer}, \textcolor{red}{tirada de D. N. Lehmer [L9]}, indica que $n$ é uma variável contínua integrada de $2$ a $x$, ou seja, $\int_{2}^{x}(dt/\log t)$. Observe que a contagem de primos de Lehmer
    %
    \footnote{Lehmer insiste em contar 1 como número primo. Para adequar ao consenso comum, sua contagem foi reduzida em uma unidade na Tabela \ref{Tabela-Lehmer}}
    %
    , que podemos seguramente considerar precisa, difere da de Gauss e que a diferença corrobora a estimativa de Gauss para os valores grandes de $x$.
    
    %
    \begin{table}[H]
        \centering
        \begin{tabular}{cccc}
             $x$& número de primos $< x$ & $\displaystyle \int_{2}^{x} \frac{dt}{\log t}$ &  Diferença\\[0.45cm]
             \hline\\[-0.3cm]
             500.000   & 41.538  & 41.606  & 68  \\[0.1cm]
             1.000.000 & 78.498  & 78.628  & 130 \\[0.1cm]
             1.500.000 & 114.155 & 114.263 & 108 \\[0.1cm]
             2.000.000 & 148.933 & 149.055 & 122 \\[0.1cm]
             2.500.000 & 183.072 & 183.245 & 173 \\[0.1cm]
             3.000.000 & 216.816 & 216.971 & 155
        \end{tabular}
        \caption{}
        \label{Tabela-Lehmer}
    \end{table}
    %
    
    Por volta de 1800, Legendre publicou em sua obra \textit{Theorie des nombres} \textcolor{red}{[L11]} uma fórmula empírica para o número de primos menores que um dado valor, \textcolor{red}{que acarretou mais ou menos} no mesmo resultado, isto é, que a densidade dos primos é $1/\log x$. Mesmo com a tentativa de Legendre de provar sua fórmula, o seu argumento se reduz a nada mais que a afirmação de que, se assumimos que a densidade dos primos tem a forma
    %
    $$\frac{1}{A_1x^{m_1} + A_1x^{m_2} + \cdots},$$
    %
    onde $m_1 > m_2> \cdots$, então $m_1$ não pode ser positivo (porque a soma \eqref{Soma-Recip-primos} convergiria neste caso), então $m_1$ deve ser "infinitamente pequeno" e a densidade deve ser da forma
    %
    $$\frac{1}{A \log x + B}.$$
    %
    Ele, então, determina $A$ e $B$ de forma empírica. A fórmula de Legendre era bem conhecida no mundo matemático e foi mencionada por matemáticos como Abel \textcolor{red}{[A2]}, Dirichlet \textcolor{red}{[D3]} and Chebyschev \textcolor{red}{[C2]} no período de 1800 a 1850.
    
    Os primeiros resultados significativos além dos de Euler foram obtidos por Chebyshev por volta de 1850. Ele provou que o erro relativo na aproximação
    %
    \begin{equation}
        \label{pi(x)-Li(x)}
        \pi(x) \approx \int_{2}^{x} \frac{dt}{\log t},
    \end{equation}
    %
    onde $\pi(x)$ denota o número de primos menores que $x$, é menor que 11\% para $x$ suficientemente grande; ou seja, ele provou
    %
    \footnote{
    Chebyshev não enunciou seu resultado desta forma. Esta forma pode ser obtida das suas estimativas para o número de primos entre $l$ e $L$ \textcolor{red}{(veja Chebyshev [C3, seção 6])} fixando $l$ e tomando $L \to \infty$ e usando $\int_{2}^{L}(dt/\log t) \approx L/\log L$.}
    %
    que
    %
    \begin{equation*}
        (0,89)\int_{2}^{x} \frac{dt}{\log t} < \pi(x) < (1,11)\int_{2}^{x} \frac{dt}{\log t}
    \end{equation*}
    %
    para todo $x$ suficientemente grande. Além disso, ele provou que nenhuma aproximação do tipo Legendre
    %
    $$ \pi(x) \approx \frac{x}{A\log x + B} $$
    %
    pode ser melhor que a aproximação \eqref{pi(x)-Li(x)} e, 
    se a razão entre $\pi(x)$ e 
    $\int_{2}^{x} (dt/\log t$ se aproxima de um limite conforme $x \to \infty$, 
    então esse limite deve ser 1. Fica claro que Chebyshev estava tentando 
    provar que o erro relativo na aproximação \eqref{pi(x)-Li(x)} 
    se aproxima de zero quando $x \to \infty$, mas foi apenas 
    quase 50 anos depois que este teorema, conhecido como 
    ``Teorema do número primo'' foi provado. 
    Mesmo o trabalho de Chebyshev sendo publicado na França muito antes 
    do artigo de Riemann, ele não se refere a Chebyshev em seu artigo. 
    No entanto, Riemann se refere a Dirichlet que 
    {\red se correspondia com} Chebyshev 
    (veja o relatório de Chebyshev sobre sua viagem à Europa Oriental 
    {\red [C5, Vol. 5...]}). 
    Dirichlet teria provavelmente informado Riemann sobre o trabalho de Chebyshev. 
    Os trabalhos não publicados de Riemann contém muitas fórmulas de Chebyshev, 
    o que indica que ele estudou o trabalho de Chebyshev, 
    e contém pelo menos uma referência direta a Chebyshev {\red(veja a figura ???)}.
    
    A contribuição real do artigo de Riemann de 1859 não está em 
    seus resultados, mas, sim, em seus métodos. 
    O resultado principal é uma fórmula para $\pi(x)$ como uma soma de 
    uma série infinita da qual $\int_{2}^{x}(dt/\log t)$ é, de longe, o maior termo. 
    No entanto, a prova de Riemann deste fato é inadequada; 
    em particular, não é, de forma alguma, claro, pelos argumentos de Riemann, 
    que a série infinita para $\pi(x)$ convirja, muito menos que seu maior 
    termo domine para valores grandes de $x$. Por outro lado, os métodos de Riemann, 
    que incluem o estudo da função $\zeta(s)$ como uma função de uma variável complexa, 
    o estudo dos zeros complexos de $\zeta(s)$, \textcolor{red}{inversão de Fourier}, 
    \textcolor{red}{inversão de Möbius} e a representação de funções para 
    funções do tipo $\pi(x)$ por "fórmulas explícitas" como sua série 
    infinita foram importantes contribuições no desenvolvimento subsequente da teoria.
    
    Nos primeiros 30 anos após a publicação do artigo de Riemann, não houve praticamente nenhum avanço
    %
    \footnote{ Uma grande exceção foi o Teorema de Mertens 
    \textcolor{red}{[M5]} de 1874 que afirma que \eqref{Soma-Recip-primos-x} 
    é verdadeira no sentido de que a diferença dos dois lados se 
    aproxima de um limite quando $x \to \infty$. Este limite é a 
    constante de Euler adicionado de $\sum_{p}[\log(1-p^{-1}) + p^{-1}]$. 
    Possivelmente, um enunciado mais natural para o Teorema de Mertens é 
    %
    $$\lim_{x \to \infty} \log \left( x \prod_{p<x}(1-p^{-1}) \right) = e^{-\gamma},$$
    %
    onde $\gamma$ é a constante de Euler. Veja Hardy e Wright \textcolor{red}{[H7]}}
    %
    na área. É como se o mundo matemático precisasse deste tempo 
    todo para processar as ideias de Riemann. Então, em um período de 10 anos, 
    Hadamard, Von Mangoldt e de la Vallée Poussin conseguiram provar a 
    fórmula principal de Riemann para $\pi(x)$, o 
    Teorema do número primo \eqref{pi(x)-Li(x)} e vários outros 
    teoremas relacionados. As ideias de Riemann foram cruciais em 
    todas as demonstrações. Desde esse período, não houve falta de 
    novos problemas em Teoria Analítica dos Números e o progresso 
    se manteve estável neste campo, boa parte deste progresso é 
    inspirado pelas ideias de Riemann. 
    
    Nenhuma discussão do contexto histórico do artigo de Riemann estaria 
    completa se não mencionasse a hipótese de Riemann. Ao longo do artigo, 
    Riemann diz que considera "muito provável" que todos os zeros de 
    $\zeta(s)$ tem parte real igual a $1/2$, mas que foi incapaz de 
    provar a veracidade desta afirmação, que é conhecida, hoje em dia, 
    como a "hipótese de Riemann". A experiência dos sucessores de Riemann 
    com esta hipótese tem sido a mesma de Riemann - eles também consideram 
    que ela provavelmente é verdadeira mas não foram capazes de prová-la. 
    Hilbert incluiu o problema de provar a hipótese de Riemann em sua 
    lista \textcolor{red}{[H9]} dos problemas não resolvidos 
    mais importantes do mundo matemático em 1900. 
    Resolver este problema foi o foco de muitos dos melhores 
    matemáticos do século 20. Este é, sem dúvidas, o 
    problema mais celebrado da matemática e continua a atrair 
    a atenção dos melhores matemáticos, não apenas porque 
    continua sem solução por tanto tempo, mas também por 
    que aparenta ser vulnerável e sua sua solução traria à 
    luz técnicas importantíssimas e com grande potencial matemático. 
    
    
    
    \section*{A fórmula do produto de Euler}
    
    
    Riemann toma, de início, a fórmula
    %
    \begin{equation}
        \label{Euler-prod}
        \sum_{n} \frac{1}{n^s} = \prod_{p} \frac{1}{\left( 1 - \frac{1}{p^s}\right)}
    \end{equation}
    %
    devida a Euler. Aqui $n$ percorre todos os inteiros positivos $(n=1,2,3, \dots)$ e $p$, 
    todos os primos $(p=2,3,5,7,11,\dots$. 
    Esta fórmula, conhecida como ``fórmula do produto de Euler'', resulta de expandir todos os fatores à direita
    %
    $$\frac{1}{\left( 1 - \frac{1}{p^s}\right)} = 1 + \frac{1}{p^s} + \frac{1}{(p^2)^s} + \frac{1}{(p^3)^s} + \cdots $$
    %
    e observar que seu produto é uma soma de fatores da forma
    %
    $$\frac{1}{(p_1^{n_1} p_2^{n_2} \cdots p_r^{n_r})^s},$$
    %
    onde $p_1, \dots, p_r$ são primos distintos e $n_1, \dots, n_r$ são números naturais, e, então, usar o Teorema Fundamental da Aritmética (todo inteiro pode, essencialmente, ser escrito de uma única maneira como um produto de números primos) para concluir que esta soma é, de fato, $\sum_{n}(1/n^s)$. Euler usou esta fórmula principalmente como uma identidade formal e para valores inteiros de $s$ (veja, por exemplo, Euler \textcolor{red}{[E5]}).
    
    Dirichlet também baseou seu trabalho
    %
    \footnote{A maior contribuição de Dirichlet para a teoria foi sua demonstração de que, se $m$ coprimo com $n$, então há infinitos números primos $p$ satisfzendo $p \equiv m \ (\mod n)$. Ele também estava interessado em questões relacionadas à densidade da distribuição dos primos, mas ele não teve um sucesso significativo com estas questões. 
    }
    %
    neste campo na fórmula do produto de Euler. Como Dirichlet foi um dos professores de Riemann e ele se refere a Dirichlet no primeiro parágrafo de seu artigo, o fato de Riemann ter usado a fórmula do produto foi provavelmente influenciado por Dirichlet. Diferentemente de Euler, Dirichlet usou a fórmula \eqref{Euler-prod} com $s$ sendo uma variável real e provou
    %
    \footnote{Dirichlet \textcolor{red}{[D3]}. Como os termos $p^{-s}$ são todos positivos, não há nada difícil relacionado a esta prova - é essencialmente uma reordenação de séries absolutamente convergentes - mas tem a importante efeito de transformar \eqref{Euler-prod} de uma identidade formal válida  para diversos valores de $s$ para uma expressão analítica verdadeira para todo número real $s>1$.
    }
    %
    rigorosamente que esta equação é válida sempre que $s>1$.
    
    É natural esperar que Riemann, como um dos fundadores da teoria de funções de uma variável complexa, considerasse $s$ como uma variável complexa. É fácil mostrar que ambos os lados de \eqref{Euler-prod} converge para $s$ complexo no semiplano $\Re(s)>1$, mas Riemann vai muito além e mostra que, mesmo os dois lados da fórmula divergindo para outros valores de $s$, a função que eles definem tem significado para todo $s$ complexo, exceto por um polo em $s=1$. Esta extensão do domínio da função requer alguns fatos sobre a função fatorial que veremos na próxima seção.
    
    
    \section{A função fatorial}
    
    
    Euler estendeu a função fatorial $n! = n(n-1)(n-2) \cdots 3 \cdot 2 \cdot 1$ dos números naturais $n$ para todo número real maior que $-1$ observando que 
    %
    \footnote{Euler escreveu a integral em termos de $y = e^{-x}$ como $\int_{0}^{1}(\log 1/y)^n \, dy$ (veja Euler \textcolor{red}{[E3]}
    }
    %
    \begin{equation}
        \label{factorial-Gamma}
        n! = \int_{0}^{\infty}e^{-x}x^n \, dx \qquad (n = 1, 2, 3, \dots)
    \end{equation}
    %
    (integração por partes) e observando, também, que que a integral à direita converge para valores não inteiros de $n$ desde que $n>-1$. Gauss \textcolor{red}{[G1]} introduziu a notação
    %
    \begin{equation}
        \label{factorial-Gamma-pi}
        \Pi(s) = \int_{0}^{\infty}e^{-x}x^s \, dx \qquad (s>-1)
    \end{equation}
    %
    para a integral de Euler no lado direito de \eqref{factorial-Gamma}. Então $\Pi(s)$ é definida para todo número real $s$ maior que -1, na verdade, para todo $s$ complexo no semiplano $\Re(s)>-1$. Além disso, $\Pi(s) = s!$ sempre que $s$ é um número natural. Existe uma outra representação de $\Pi(s)$ que era conhecida
    %
    \footnote{Infelizmente, Legendre adotou $\Gamma(s)$ para $\Pi(s-1)$. As razões de Legendre para considerar $(n-1)!$ em vez de $n!$ são desconhecidas (talvez ele sentiu que fosse mais natural o primeiro polo ser em $s=0$ em vez de $s=-1$). De qualquer modo, esta notação prevaleceu na França e, por volta do fim do século 19, no resto do mundo também. A notação de Gauss me parece muito mais natural e o fato de Riemann tê-la usado me dá a oportunidade de trazê-la de volta.
    }
    %
    por Euler
    %
    \begin{equation}
        \label{Pi(s)-lim-N}
        \Pi(s) = \lim_{N \to \infty} \frac{1 \cdot 2 \cdots N}{(s+1)(s+2)\cdots (s+N)}(N+1)^s.
    \end{equation}
    %
    Esta fórmula é válida para todos os $s$ para os quais \eqref{factorial-Gamma-pi} define $\Pi(s)$, ou seja, para todo $s$ no semiplano $\Re(s)>-1$. Por outro lado, não é difícil mostrar [use a fórmula \eqref{Pi(s)-prod} abaixo] que o limite \eqref{Pi(s)-lim-N} existe para todo valor de $s$, real ou complexo, desde que o denominador não se anule, isto é, que $s$ não seja um inteiro negativo. Em suma, a fórmula \eqref{Pi(s)-lim-N} estende a definição de $\Pi(s)$ para todos os valores de $s$ exceto para $s = -1, -2, -3, \dots$.
    
    As fórmulas \eqref{factorial-Gamma-pi} e \eqref{Pi(s)-lim-N} coincidem para $\Re(s)> -1$. Também usaremos, sem demonstrar, os seguintes fatos:
    %
    \begin{align}
        \label{Pi(s)-identidade}
        \Pi(s) &= \prod_{n=1}^{\infty}\frac{n^{1-s}(n+1)^s}{s+n} =  \prod_{n=1}^{\infty}\left(1 + \frac{s}{n}\right)^{-1} \left(1 + \frac{1}{n}\right)^{s}  \\
        %
        \Pi(s) &= s\Pi(s-1)\\
        %
        \frac{\pi s}{\Pi(s)\Pi(-s)} &= \sin \pi s\\
        %
        \Pi(s) &= 2^s\Pi\left(\frac{s}{2}\right)\Pi\left(\frac{s-1}{2}\right)\pi^{-1/2}
    \end{align}
    %
    Para a demonstração destes resultados o leitor é indicado a procurar qualquer livro que lida com a função fatorial ou a "função $\Gamma$", por exemplo, Edwards \textcolor{red}{[E1]}. A identidade \eqref{Pi(s)-identidade} é uma simples reformulação de \eqref{Pi(s)-lim-N}. Usando ela, pode-se provar que $\Pi(s)$ é uma função analítica da variável complexa $s$ que tem polos simples em $s = -1, -2, -3, \dots$ e não se anula. A identidade \eqref{Pi(s)-identidade-2} é chamada de "equação funcional da função fatorial"; juntamente com $\Pi(0)=1$ (de \eqref{Pi(s)-identidade}), obtemos $\Pi(n) = n!$ diretamente. A fórmula \eqref{Pi(s)-identidade-3} é, essencialmente, a fórmula do produto para o seno; quando $s = 1/2$, usamos \eqref{Pi(s)-identidade-2} para concluir que $\Pi(\frac{-1}{2}) = \Pi^{1/2}$. A identidade \eqref{Pi(s)-identidade-4} é conhecida como a relação de Legendre. É o caso $n=2$ de outra relação mais geral
    %
    \begin{equation*}
        \frac{\Pi(s)}{n^s \Pi\left( \frac{s-1}{2}\right) \cdots \Pi\left( \frac{s-n + 1}{n}\right)} 
        = \left[ \frac{2\pi n}{(2\pi)^n}\right]^{1/2}
    \end{equation*}
    %
    que não será necessária.
    
    
    \section{A função $\zeta(s)$}
    
%regaça 
%\o/













