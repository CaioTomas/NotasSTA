\appendix

\chapter{On the Number of Primes Less Than a Given Magnitude}

%\hfill%
%\begin{minipage}{10cm}
\begin{flushleft}
\rightskip=0.5cm
by BERNHARD RIEMANN\footnote{Translated from \textit{Uber die Anzahl der Primzahlen unter einer gegebenen Grösse} by H.M. Edwards.}
\end{flushleft}
%\end{minipage}

I believe I can best express my gratitude for the honor which the Academy 
has bestowed on me in naming me as one of its correspondents by immediately 
availing myself of the privilege this entails to communicate an investigation
of the frequency of prime numbers, a subject which because of the interest
shown in it by Gauss and Dirichlet over many years seems not wholly unworthy
of such a communication.

In this investigation I take as my starting point the observation of Euler
that the product
%
\[
\prod \frac{1}{1 - \dfrac{1}{p^s}} = \sum \frac{1}{n^s},
\]
%
where $p$ ranges over all prime numbers and $n$ over all whole numbers. The
function of a complex variable $s$ which these two expressions define when they
converge I denote by $\zeta(s)$. They converge only when the real part of $s$
is greater than $1$; however, it is easy yo find and expression of the function
always is valid. By applying the equation
%
\[
\int_0^{\infty} e^{-nx}x^{s-1} \, dx = \frac{\prod(s-1)}{n^s},
\]
%
one finds first
%
\[
\prod(s-1)\zeta(s) = \int_0^{\infty} \frac{x^{s-1}\, dx}{e^x - 1}.
\]
%
If one considers the integral
%
\[
\int \frac{(-x)^{s-1} \, dx}{e^x - 1}
\]
%
from $+\infty$ to $+\infty$ in the positive sense around the boundary
of a domain which contains the value 0 but no other singularity of the
integrand in its interior, then it is easily seen to be equal to
%
\[
(e^{-\pi si} - e^{\pi si})\int_0^{\infty} \frac{x^{s-1} \, dx}{e^x - 1},
\]
%
provided that in the many-valued function $(-x)^{s-1} = e^{(s-1)\log(-x)}$
the logarithm of $-x$ is determined in such a way that it is real for negative
values of $x$. Thus
%
\[
2\sin\pi s\prod(s-1)\zeta(s) 
= i \int_{\infty}^{\infty} \frac{(-x)^{s-1} \, dx}{e^x - 1}
\]
%
when the integral is defined as above.

This equation gives the value of the function $\zeta(s)$ for all complex
$s$ and shows that it is single-valued and finite for all values of $s$
other than 1, and also that it vanishes when $s$ is a negative even integer.

When the real part of $s$ is negative, the integral can be taken, instead
of in the positive sense around the boundary of the given domain, in the
negative sense around the complement of this domain because in that case
(when $\Re s < 0$) the integral over values with infinitely large modulus
is infinitely small. But inside this complementary domain the only singularities
of the integrand are at the integer multiples of $2\pi i$, and the integral 
is therefore equal to the sum of the integrals taken around these singularities
in the negative sense. Since the integral around the value $n2\pi i$ is
$(-n2\pi i)^{s-1}(-2\pi i)$, this gives
%
\[
2\sin\pi s\prod(s-1)\zeta(s) = (2\pi)^s\sum n^{s-1}[(-i)^{s-1} + i^{s-1}],
\]
%
and therefore a relation between $\zeta(s)$ and $\zeta(1-s)$ which, by making
use of known properties of the function $\prod$, can also be formulated as the
statement that
%
\[
\prod\left(\frac{s}{2} - 1\right)\pi^{-s/2}\zeta(s)
\]
%
remains unchanged when $s$ is replaced by $1-s$.

This property of the function motivated me to consider the integral 
$\prod ((s/2) - 1)$ instead of the integral $\prod(s-1)$ in the general
term of $\sum n^{-s}$, which leads to a very convenient expression of
the function $\zeta(s)$. In fact
%
\[
\frac{1}{n^s}\prod\left(\frac{s}{2} - 1\right)\pi^{-s/2}
= \int_0^{\infty} e^{-nn\pi x}x^{(s/2) - 1} \, dx;
\]
%
so when one sets
%
\[
\sum_{1}^{\infty} e^{-nn\pi x} = \psi(x),
\]
%
it follows that
%
\[
\prod\left(\frac{s}{2} - 1\right)\pi^{-s/2}\zeta(s)
= \int_0^{\infty} \psi(x)x^{(s/2) - 1} \, dx
\]
%
or, because
%
\begin{equation*}
\tag{Jacobi, Fund., p. 184}
2\psi(x) + 1 = x^{-1/2}\left[ 2\psi\left(\frac{1}{x}\right) + 1 \right],
\end{equation*}
%
that
%
\begin{align*}
    \prod\left(\frac{s}{2} - 1\right) 
    &= \int_1^{\infty} \psi(x)x^{(s/2) - 1} \, dx
    + \int_0^1 \psi\left(\frac{1}{x}\right) x^{(s-3)/2} \, dx \\
    &+ \frac{1}{2}\int_0^1 (x^{(s-3)/2} - x^{(s/2) - 1}) \, dx \\
    &= \frac{1}{s(s-1)} 
    + \int_1^{\infty} \psi(x)(x^{(s/2) - 1} + x^{-(1+s)/2}) \, dx.
\end{align*}
%
I now set $s = \frac{1}{2} + ti$ and
%
\[
\prod\left(\frac{s}{2}\right)(s-1)\pi^{-s/2}\zeta(s) = \xi(t)
\]
%
so that
%
\[
\xi(t) = \frac{1}{2} - 
\left(tt + \frac{1}{4}\right)\int_1^{\infty} \psi(x)x^{-3/4}
\cos\left(\frac{1}{2}t\log x\right) \, dx
\]
%
or also
%
\[
\xi(t) = 4\int_1^{\infty} \frac{d[x^{3/2}\psi'(x)]}{dx}
x^{-1/4}\cos\left(\frac{1}{2}t\log x\right) \, dx.
\]
%
This function is finite for all finite values of $t$ and can be developed
as a power series in $tt$ which converges very rapidly. Now since for values
of $s$ with real part greater than 1, $\log\zeta(s) = - \sum \log(1 - p^{-s})$
is finite and since the same is true of the other factors of $\xi(t)$, the
function $\xi(t)$ can vanish only when the imaginary part of $t$ lies
between $\frac{1}{2}i$ and $-\frac{1}{2}i$. The number of roots of $\xi(t) = 0$
whose real parts lie between 0 and $T$ is about
%
\[
= \frac{T}{2\pi}\log\frac{T}{2\pi} - \frac{T}{2\pi}
\]
%
because the integral $\int d\log\xi(t)$ taken in the positive sense around the
domain consisting of all values whose imaginary parts lie between $\frac{1}{2}i$
and $-\frac{1}{2}i$ and whose real parts lie between 0 and $T$ is
(up to a fraction of the order of magnitude of $1/T$) equal to
$[T\log(T/2\pi) - T]i$ and is, on the other hand, equal to the number
of roots within these bounds and it is very likely that all of the
roots are real. One would of course like to have a rigorous proof
of this, but I have put aside the search for such a proof after some fleeting
vain attempts because it is not necessary for the immediate objetive of my
investigation.

If one denotes by $\alpha$ the roots of the equation $\xi(\alpha) = 0$,
then one can express $\log\xi(t)$ as
%
\[
\sum\log\left(1 - \frac{tt}{\alpha\alpha}\right) + \log\xi(0)
\]
%
because, since the density of roots of size $t$ grows only like $\log(t/2\pi)$
as $t$ grows, this expression converges and for infinite $t$ is only infinite
like $t\log t$; thus it differs from $\log\xi(t)$ by a function of $tt$ which
is continuous and finite for finite $t$ and which, when divided by $tt$, is
infinitely small for infinite $t$. This difference is therefore a constant, 
the value of which can be determined by setting $t=0$.

With these preparatory facts, the number of primes less than $x$ can now be
determined.

Let $F(x)$, when $x$ is not exactly equal to a prime, be equal to this number,
but when $x$ is a prime let it be greater by $\frac{1}{2}$ so that for an $x$
where $F(x)$ jumps
%
\[
F(x) = \frac{F(x+0) + F(x-0)}{2}.
\]
%
If one sets
%
\[
p^{-s} = s\int_p^{\infty} x^{-s-1} \, dx, 
\qquad
p^{-2s} = s\int_{p^2}^{\infty} x^{-s-2} \, dx,
\qquad
\dots
\]
%
in the formula
%
\[
\log\zeta(s) = -\sum\log(1 - p^{-s}) 
             = \sum p^{-s} + \frac{1}{2}\sum p^{-2s} + \frac{1}{3}\sum p^{-3s}
             + \cdots,
\]
%
one finds
%
\[
\frac{\log\zeta(s)}{s} = \int_1^{\infty} f(x)x^{-s-1} \, dx
\]
%
when one denotes
%
\[
F(x) + \frac{1}{2}F(x^{1/2}) + \frac{1}{3}F(x^{1/3}) + \cdots
\]
%
by $f(x)$.

This equation is valid for every complex value $a+bi$ os $s$ provided $a>1$.
But when in such circumstances
%
\[
g(s) = \int_0^{\infty} h(x)x^{-s} d\log x
\]
%
is valid, the function $h$ can be expressed in terms of $g$ by means of 
Fourier's theorem. The equation splits when $h$ is real and when 
$g(a+bi) = g_1(b) + ig_2(b)$ into the equation
%
\begin{align*}
    g_1(b) &= \int_0^{\infty} h(x)x^{-a}\cos(b\log x) d\log x, \\
    ig_2(b) &= -i\int_0^{\infty} h(x)x^{-a}\sin(b\log x) d\log x.
\end{align*}
%
When both equations are multiplied by $[\cos(b\log y) + i\sin(b\log y)]\, db$
and integrated from $-\infty$ to $+\infty$, one finds in both cases that the
right side is $\pi h(y)y^{-\alpha}$ so that when they are added and multiplied
by $iy^{\alpha}$
%
\[
2\pi i h(y) = \int_{a - \infty y}^{a + \infty t} g(s) y^s \, ds,
\]
%
where the integration is to be carried out in such a way that the real part
of $s$ remains constant.

The integral represents, for a value of $y$ where the function $h(y)$ has a
jump, the middle value between the two values of $h$ on either side of the
jump. The function $f$ was defined in such a way that it too has this property,
so one has in full generality
%
\[
f(y) 
= \frac{1}{2\pi i}\int_{a-\infty i}^{a+\infty i} \frac{\log\zeta(s)}{s}y^s \, ds.
\]
%
For $\log\zeta$ when can now substitute the expression
%
\begin{align*}
    \frac{s}{2}\log\pi - \log(s-1) - \log\prod\left(\frac{s}{2}\right) \\
    + \sum_{\alpha} \log\left[1 + \frac{(s - \frac{1}{2})^2}{\alpha\alpha}\right]
    + \log\xi(0)
\end{align*}
%
found above; the integrals of the individual terms of this expression will not
converge, however, when they are taken to infinity, so it is advantageous to
reformulate the equation as
%
\[
f(x) = -\frac{1}{2\pi i}\frac{1}{\log x}
\int_{a-\infty i}^{a+\infty i} \frac{d\frac{\log\zeta(s)}{s}}{ds} x^s \, ds
\]
%
by integration by parts.

Since
%
\[
-\log\prod\left(\frac{s}{2}\right) 
= \lim\left[ \sum_{n=1}^{m} \log\left( 1 + \frac{s}{2n} \right) 
- \frac{s}{2}\log m \right]
\]
%
for $m=\infty$ and therefore,
%
\[
-\frac{d\frac{1}{2}\log\prod\left(\frac{s}{2}\right)}{ds}
= \sum_{1}^{\infty} \frac{d\frac{1}{2}\log\left(1 + \frac{s}{2n}\right)}{ds},
\]
%
all of the terms in the expression for $f(x)$ except for the term
%
\[
\frac{1}{2\pi i}\frac{1}{\log x} 
\int_{a-\infty i}^{a+\infty i} \frac{1}{ss}\log\xi(0)x^s \, ds = \log\xi(0)
\]
%
take the form
%
\[
\pm\frac{1}{2\pi i}\frac{1}{\log x}
\int_{a-\infty i}^{a+\infty i} 
\frac{d\left[\frac{1}{s}\log\left(1 - \frac{s}{\beta}\right)\right]}{ds}x^s \, ds.
\]
%
But
%
\[
\frac{d\left[\frac{1}{s}\log\left(1 - \frac{s}{\beta}\right)\right]}{d\beta} 
= \frac{1}{(\beta - s)\beta}
\]
%
and, when the real part of $s$ is greater than the real part of $\beta$,
%
\[
-\frac{1}{2\pi i}\int_{a-\infty i}^{a+\infty i}
\frac{x^s \, ds}{(\beta - s)\beta} 
= \frac{x^{\beta}}{\beta}
= \int_{\infty}^x x^{\beta - 1} \, dx
\]
%
or
%
\[
= \int_0^x x^{\beta - 1} \, dx
\]
%
depending on whether the real part ot $\beta$ is negative or positive.
Thus
%
\begin{align*}
    \pm\frac{1}{2\pi i}\frac{1}{\log x}
    \int_{a-\infty i}^{a+\infty i}
    \frac{d\left[\frac{1}{s}\log\left(1 - \frac{s}{\beta}\right)\right]}
    {ds}x^s \, ds \\
    = -\frac{1}{2\pi i}\int_{a-\infty i}^{a+\infty i}
    \frac{1}{s}\log\left( 1 - \frac{s}{\beta} \right)x^s \, ds \\
    = \int_{\infty}^x \frac{x^{\beta - 1}}{\log x} \, dx + \text{const}
\end{align*}
%
in the first case and
%
\[
= \int_0^x \frac{x^{\beta - 1}}{\log x} \, dx + \text{const}
\]
%
in the second case.

In the first case the constant of integration can be determined by taking
$\beta$ to be negative and infinite. In the second case the integral from
0 to $x$ takes on two values which differ by $2\pi i$ depending on whether
the path of integration is in the upper halfplane or in the lower halfplane;
if the path of integration is in the upper halfplane, the ingral will be
infinitely small when the coefficient of $i$ in $\beta$ is infinite and positive,
and if the path is in the lower halfplane, the integral will be infinitely
small when the coefficient of $i$ in $\log[1 - (s/\beta)]$ on the left side
in such a way that the constants of integration drop out.

By setting these values in the expression for $f(x)$ one finds
%
\begin{align*}
    f(x) = \text{Li}(x) - 
    \sum_{\alpha} 
    [\text{Li}(x^{(1/2) + \alpha i}) + \text{Li}(x^{(1/2) - \alpha i})] \\
    + \int_x^{\infty} \frac{1}{x^2 - 1}\frac{dx}{x\log x} + \log\xi(0),
\end{align*}
%
where the sum $\sum_{\alpha}$ is over all positive roots (or all roots with
positive real parts) of the equation $\xi(\alpha) = 0$, ordered according
to their size. It is possible, by means of a more exact discussion of the
function $\xi$, easily to show that with this ordering of the roots the sum
of the series
%
\[
\sum_{\alpha} 
[\text{Li}(x^{(1/2) + \alpha i}) + \text{Li}(x^{(1/2) - \alpha i})]\log x
\]
%
is the same as the limiting value of
%
\[
\frac{1}{2\pi i}\int_{a-bi}^{a+bi} 
\frac{d\frac{1}{s}\sum\log\left[1 + \frac{(s - \frac{1}{2})^2}{\alpha\alpha}\right]}{ds} x^s \, ds
\]
%
as $b$ grows without bound; by a different ordering, however, it can approach
any arbitrary real value.

From $f(x)$ one can find $F(x)$ by inverting
%
\[
f(x) = \sum \frac{1}{n}F(x^{1/n})
\]
%
to find
%
\[
F(x) = \sum (-1)^{\mu} \frac{1}{m} f(x^{1/m}),
\]
%
where $m$ ranges over all positive integers which are not divisible by any square
other than 1 and where $\mu$ denotes the number o prime factors of $m$.

If $\sum_{\alpha}$ is restricted to a finite number of terms, then the derivative
of the expression for $f(x)$ or, except for a part which decreases very rapidly
as $x$ increases,
%
\[
\frac{1}{\log x} - 2\sum_{\alpha}\frac{\cos(\alpha\log x)x^{-1/2}}{\log x}
\]
%
gives an approximate expression for the density of primes + half the density of
prime squares + $\frac{1}{3}$ the density of prime cubes, etc., of magnitude $x$.

Thus the known approximation $F(x) = \text{Li}(x)$ is correct only to an order
of magnitude of $x^{1/2}$ and gives a value which is somewhat too large, because
the nonperiodic terms in the expression of $F(x)$ are, except for quantities
which remain bounded as $x$ increases, 
%
\begin{align*}
\text{Li}(x) - \frac{1}{2}\text{Li}(x^{1/2}) - \frac{1}{3}\text{Li}(x^{1/3})
- \frac{1}{5}\text{Li}(x^{1/5}) \\ 
+ \frac{1}{6}\text{Li}(x^{1/6}) - \frac{1}{7}\text{Li}(x^{1/7}) + \cdots.
\end{align*}
%
In fact the comparison of $\text{Li}(x)$ with the number of primes less than
$x$ which was undertaken by Gauss and Goldschmidt and which was pursued up
to $x=$ three million shows that the number of primes is already less than
$\text{Li}(x)$ in the first hundred thousand and that the difference, with
minor fluctuations, increases gradually as $x$ increases. The thickening
and thinning of primes which is representend by the periodic terms in the
formula has also been observed in the counts of primes, without, however, any
possibility of establishing a law for it having been noticed. It would be
interesting in a future count to examine the influence of individual periodic
terms in the formula for the density of primes. More regular than the behavior
of $F(x)$ is the behavior of $f(x)$ which already in the first hundred is on
average very nearly equal to $\text{Li}(x) + \log\xi(0)$.