\chapter[Continuação analítica]{Continuação analítica}
\chaptermark{}

\hfill%
\begin{minipage}{10cm}
    \begin{flushright}
    \rightskip=0.5cm
        \textit{``The greatest strategy is doomed if it's implemented badly.''}
        \\[0.1cm]
    \rightskip=0.5cm
    --- Bernhard Riemann
    \end{flushright}
\end{minipage}

\section{O princípio da reflexão de Schwarz} %
\label{sec:Schwarz} %
    Neste capítulo, vamos tratar de continuações analíticas ao longo de caminhos.
    Para iniciar a discussão, começamos abordando um princípio simples que nos permite
    estender holomorficamente certas funções analíticas a domínios ``maiores'': 
    o princípio da reflexão de Schwarz.
    
    A demonstração desse teorema consiste de duas partes: precisamos definir a 
    (candidata a) extensão e checar que ela é holomorfa. Começamos com o segundo passo.
    
    Seja $\Omega\subseteq\C$ um aberto simétrico com respeito à reta real, ou seja,
    %
    \begin{equation*}
        z\in\Omega \iff \overline{z}\in\Omega.
    \end{equation*}
    %
    Denotemos por 
    $\Omega^+$ a porção de $\Omega$ no semi-plano superior e por 
    $\Omega^-$ a porção de $\Omega$ no semi-plano inferior.
    %
    \begin{center}
        \textbf{diagrama fig.11 Stein p.58}
    \end{center}
    %
    Por fim, denotemos por $I$ o conjunto que ``sobrou'' de $\Omega$: 
    $I = \Omega\cap\R$. É claro então que
    %
    \begin{equation*}
        \Omega = \Omega^+ \cup I \cup \Omega^-
    \end{equation*}
    %
    e os casos interessantes do teorema a seguir ocorrem quando $I\neq\varnothing$.
    
    \begin{teorema}[Princípio da simetria]
    \label{teo-principio-simetria}
        Se $f^+$ e $f^-$ são funções holomorfas em $\Omega^+$ e $\Omega^-$, respectivamente,
        que se estendem continuamente para $I$ e tais que $f^+(x) = f^-(x), \forall x\in I$,
        então a função $f$ definida em $\Omega$ por
        %
        \begin{equation*}
            f(z) =  \begin{cases}
                      f^+(z),          & z\in\Omega^+ \\
                      f^+(z) = f^-(z), & z\in I       \\
                      f^-(z),          & z\in\Omega^-
                    \end{cases}
        \end{equation*}
        %
        é holomorfa em todo $\Omega$.
    \end{teorema}
    
    Apresentamos aqui a demonstração dada em \cite{MR1976398}, pp. 58-59, 
    devido à sua simplicidade e clareza. Um argumento mais rigoroso pode ser 
    encontrado em \cite{MR503901}, pp. 213-214.
    
    \begin{proof}
        Note, antes de mais nada, que por definição temos $f$ contínua em $\Omega$.
        A nossa única dificuldade se encontra em mostrar a holomorficidade de $f$ em $I$.
        
        Seja $D$ um disco centrado em um ponto de $I$ e inteiramente contido em $\Omega$.
        Usando o Teorema de Morera, vamos mostrar que $f$ é holomorfa em $D$. Para tal,
        seja $T$ um triângulo em $D$. Se $T$ não intercepta $I$, então
        %
        \begin{equation*}
            \int_T f(z) \, dz = 0
        \end{equation*}
        %
        já que $f$ é holomorfa nos semi-discos superior e inferior. 
        Suponhamos, então, que $T$ intercepta $I$ e analisemos os três casos que surgem:
        \paragraph{Caso I:} $T$ tem apenas um vértice sobre $I$. Nesse caso, se 
        $Q_\varepsilon$ é o quadrilátero obtido levantando levemente o vértice sobre $I$, 
        temos
        %
        \begin{equation*}
            \int_{Q_\varepsilon} f(z) \,dz = 0
        \end{equation*}
        %
        pois esse quadrilátero está inteiramente contido no semi-disco superior. Daí,
        fazendo $\varepsilon\to 0$, segue por continuidade que
        %
        \begin{equation*}
            \int_T f(z) \, dz = 0.
        \end{equation*}
        
        \paragraph{Caso II:} $T$ tem um lado sobre $I$. Nesse caso, podemos supor sem perda
        de generalidade que o restante de $T$ está no semi-disco superior. Se $T_\varepsilon$
        é o triângulo obtido de $T$ levantando levemente a aresta que está sobre $I$, temos
        %
        \begin{equation*}
            \int_{T_\varepsilon} f(z) \, dz = 0
        \end{equation*}
        %
        pois esse novo triângulo está contido no semi-disco superior. Daí, fazendo 
        $\varepsilon\to 0$, segue por continuidade que
        %
        \begin{equation*}
            \int_T f(z) \, dz = 0.
        \end{equation*}
        
        \begin{center}
            \textbf{diagrama p.59 Stein}
        \end{center}
        
        \paragraph{Caso III:} $T$ tem duas arestas interceptando $I$. Nesse caso, o interior
        de $T$ intercepta $I$ e podemos reduzir esse caso aos anteriores se escrevermos $T$
        como união de triângulos que têm um lado ou um vértice sobre $I$.
        
        Portanto, pelo Teorema de Morera segue que $f$ é holomorfa em $D$, como desejado.
    \end{proof}
    
    Podemos agora enunciar o princípio da reflexão, usando a mesma notação acima.
    
    \begin{teorema}[Princípio da Reflexão de Schwarz]
    \label{teo-reflexao-schwarz}
        Suponha que $f$ é uma função holomorfa em $\Omega^+$ que se estende 
        continuamente para $I$ e tal que $f$ toma valores reais em $I$. 
        Então existe uma função $F$ holomorfa em todo o $\Omega$ tal que
        $F\big|_{\Omega^+} = f$.
    \end{teorema}
        
    \begin{proof}
        A ideia é definir $F$ em $\Omega^-$ por
        %
        \begin{equation*}
            F(z) = \overline{f(\overline{z})} .
        \end{equation*}
        %
        Para mostrar que $F$ é holomorfa em $\Omega^-$, notamos que se $z, z_0\in\Omega^-$
        então $\overline{z}, \overline{z_0}\in\Omega^+$ e, portanto, a expansão de $f$
        em série de potências em torno de $\overline{z_0}$ nos dá
        %
        \begin{equation*}
            f(\overline{z}) = \sum_{n=0}^{\infty} a_n (\overline{z} - \overline{z_0})^n,
        \end{equation*}
        %
        donde segue que
        %
        \begin{equation*}
            F(z) = \sum_{n=0}^{\infty} \overline{a_n} (z - z_0)^n,
        \end{equation*}
        %
        ou seja, $F$ é holomorfa em $\Omega^-$. Como $f$ toma valores reais em $I$, temos
        $\overline{f(x)} = f(x), \forall x\in I$ e, portanto, $F$ se estende continuamente
        a $I$. Invocando o princípio da simetria, terminamos a prova.
    \end{proof}
%

\section{O logaritmo de uma função}

Vamos investigar se é possível definir o logaritmo complexo para uma função holomorfa 
$f: G \to \C$, onde $G$ é um domínio contendo a origem. Caso tal construção fosse possível,
então $\log f(z)$ seria a primitiva de $f'(z)/f(z)$. O princípio do argumento nos mostra que
%
\begin{equation*}
    \int_\gamma \frac{f'(z)}{f(z)} dz = 2\pi i K,
\end{equation*}
%
onde $\gamma$ é uma curva fechada e $K$ é um inteiro. Isto significa que $\log f(z)$ muda 
de um tanto $2\pi i K$ cada vez que a variável $z$ percorre $\gamma$; como consequência disso,
a parte imaginária do logaritmo (o argumento) muda de um tanto $2 \pi K$. No entanto, uma
função holomorfa que admite uma primitiva é tal que sua integral ao longo de uma curva fechada
é sempre nula. Isto mostra que deveríamos ter
%
\begin{equation*}
    \int_\gamma \frac{f'(z)}{f(z)} \, dz = 0,
\end{equation*}
o que nos mostra que não podemos definir $\log f(z)$.

De uma forma mais precisa, definir o logaritmo para a função $f$ significa escolher um ramo 
do logaritmo em $G$, i.e., uma função $g: G \to \C$ tal que 
%
\begin{equation*}
    \exp(g(z)) = f(z), \forall z \in G.  
\end{equation*}
%
Já vimos que esta construção não será possível por conta do princípio do argumento, mas a
seguir faremos uma tentativa de definir tal função e veremos em detalhes por que não é 
possível fazê-la.

Vamos tratar de um caso específico em que $G$ é um domínio contendo a origem e 
$\gamma: [0,1] \to \C$ é uma circunferência com centro em $0$, $\gamma = r\exp{2 \pi i t}$, 
em que $r > 0$ é tal que os pontos de $\gamma$ estão em $G$. Vamos supor que $f$ tem apenas 
uma singularidade $z_0$ no interior da região delimitada por $\gamma$. Sendo $\gamma$ uma
aplicação contínua, temos que a curva que ela parametriza é um compacto de $\C$, pois é o
conjunto $\gamma([0,1])$ e $[0,1]$ é compacto em $\R$. Isso significa que podemos escolher um
número finito de bolas $B_i$ centradas nos pontos $\gamma(t_i)$, com $t_i \in [0,1]$ e 
$i \in \{1,\dots, n\}$. Vamos supor que $t_1 < t_2 < \cdots < t_n$ e que 
$\gamma(t_1) = \gamma(t_n)$.  Os raios das bolas $B_i$ podem ser escolhidos todos iguais a
algum $\varepsilon > 0$ tal que $z_0 \not \in B_i$ e $B_i \subseteq G$ para todo $i$.

A ideia é escolher, para cada $i$, um ramo $\mathcal{L}_i$ do logaritmo em $\C$ tal que 
$\log f(z)$ esteja bem definido na bola $B_i$. Essa escolha deve ser feita de tal modo que
$\mathcal{L}_i$ e $\mathcal{L}_{i+1}$ coincidam na interseção $B_i \cap B_{i+1}$ para todo $i$.
Dessa forma, usando o princípio da identidade, vamos definir $\log f(z)$ estendendo o primeiro
ramo analiticamente para os outros ramos em um subconjunto de $G$ que não contém $z_0$. 
É importante excluir $z_0$, pois $\log$ não está definido em $0$ em nenhum ramo.

Denote $a_i = \gamma(t_i)$. Vamos mostrar que podemos definir $\log f(z)$ em $B_i$. 
O que precisamos aqui é que seja possível escolher uma semirreta 
$L_\phi = \{s(\cos \phi, \sin \phi): s > 0\}$ que não intercepte $f(B_i)$; caso contrário, 
não poderíamos definir a imagem de todos os pontos de $f(B_i)$ por um mesmo ramo do logaritmo.
Nesta parte a discussão fica um pouco mais técnica, então vamos apenas enunciar o resultado 
que usaremos. 
%
\begin{teorema}
    Seja $f: G \to \C$ uma função holomorfa, onde $G$ é um domínio. 
    Se $U \subseteq G$ é um aberto tal que $\overline{U} \subseteq G$, então
    \begin{itemize}
        \item $f(\partial U) \subseteq \partial f(U)$;
        \item $f(U)$ é aberto.
    \end{itemize}
\end{teorema}
%
Faremos a hipótese adicional daqui em diante de que $\overline{B}_i \subseteq G$. 
Este teorema nos permite concluir que podemos, de fato, escolher algum $L_\phi$ 
bom o bastante para definir o ramo $\mathcal{L}_i$. Observe ainda que $0 \not \in f(B_i)$.

Passemos agora à construção de $\log f(z)$. Começamos com $B_1$ centrado em $\gamma(t_1)$
escolhendo um ramo $\mathcal{L}_1$ adequado, denotamos a semirreta $L_\phi$ correspondente por
$L_{\phi_1}$. Para o próximo ramo, $\mathcal{L}_2$, escolhemos $\phi_2$ tal que $L_{\phi_2}$
não intercepte a interseção $f(B_1) \cap f(B_2)$. Dessa forma, 
$\log_{\phi_1} f(z) = \log_{\phi_2} f(z)$ para todo $z$ nesta interseção. 
Neste ponto, conseguimos definir $\log f(z)$ em $B_1 \cup B_2$ da seguinte forma:
%
\begin{equation*}
    \log f(z) =
    \begin{cases}
        \log_{\phi_1} f(z), z \in B_1 \\
        \log_{\phi_2} f(z), z \in B_2.
    \end{cases}
\end{equation*}
%
Continuamos dessa maneira e fazemos agora o mesmo para a escolha de $L_{\phi_3}$, 
restringindo nossa atenção à interseção $f(B_2) \cap f(B_3)$. Por simplicidade, 
vamos escrever $l_i(z) =  \log_{\phi_i} f(z)$. Observe que, como $a_n = a_1$, 
definiremos um total de $n-1$ ramos do logaritmo. Quando chegarmos na última bola, 
$B_n \ (= B_1)$, temos uma condição de compatibilidade adicional a ser verificada. 
Devemos ter $l_{n-1}(z) = l_1(z)$ para todo $z$ em $B_{n-1} \cap B_1$. Vamos supor que isso
tenha sido feito com sucesso.

Denote por $\gamma_j$ a função $\gamma$ restita ao intervalo $[t_{j-1},t_j]$, ou seja, ela
parametriza o pedaço de $\gamma$ que vai de $a_{j-1}$ para $a_j$. Adicionamos daqui em diante 
a hipótese de que $a_i$ e $a_{i+1}$ são pontos de $B_i$ para todo $i$. Pelo teorema fundamental
do cálculo, temos
%
\begin{equation*}
    \int_{\gamma_j} = \frac{f'(z)}{f(z)}\, dz = l_j(a_j) - l_j(a_{j-1}).  
\end{equation*}
%
Integrando em $\gamma$ e usando o princípio do argumento, obtemos
\begin{align*}
    2 \pi i K &= \int_{\gamma} = \frac{f'(z)}{f(z)}\, dz \\
    &= \sum_{j=1}^{n-1}\int_{\gamma_j} = \frac{f'(z)}{f(z)}\, dz \\
    &= [l_1(a_2) - l_1(a_1)] + [l_2(a_3) - l_2(a_2)] + \cdots \\
    &+ [l_{n-2}(a_{n-1}) - l_{n-2}(a_{n-2})] + [l_{n-1}(a_{n}) - l_{n-1}(a_{n-1})] \\
    &= l_{n-1}(a_n) - l_1(a_1)\\
    &= 0,
\end{align*}
onde $K$ é a multiplicidade do zero $z_0$ de $f$. Observe que as condições de compatibilidade,
que são satisfeitas por construção ($l_i$ e $l_{i+1}$ são coincidentes na interseção 
$B_i \cap B_{i+1}$), fazem com que a soma acima tenha vários cancelamentos, restando apenas o
termo na penúltima igualdade. A soma então se anula por conta da condição de compatibilidade
adicional que supomos ser satisfeita ($l_{n-1}(z) = l_1(z)$ para todo $z$ em 
$B_{n-1} \cap B_1$). Lembre-se ainda que $a_1 = a_n$.

Por esta contradição, concluímos que não podemos definir $\log f(z)$ desta forma que fizemos.
Este é apenas um caso particular, mas mostra um pouco do por quê não é possível fazer tal
construção.

\section{Continuação analítica ao longo de caminhos}

% fazer a discussão introdutória (slides da Aula 1 e início dos slides da Aula 2)

    \begin{definicao}
        \label{def-elemento-funcao}
        Um elemento de função é um par $(f, G)$ onde $G$ é um domínio e $f$
        é uma função analítica em $G$. Dado um elemento de função $(f,G)$, 
        definimos o germe de $f$ em $a$, denotado por $[f]_a$, 
        como a coleção de todos os elementos de função $(g,D)$ tais que 
        $a\in D$ e $f(z) = g(z)$ para todo $z$ em uma vizinhança de $a$.
    \end{definicao}


    \begin{definicao}
    \label{def-continuacao-analitica}
        Seja $\gamma: [0,1] \to \C$ um caminho e suponha que para cada
        $t\in[0,1]$ existe um elemento de função $(f_t, D_t)$ tal que
        \begin{enumerate}[(a)]
            \item $\gamma(t) \in D_t$;
            \item para cada $t\in[0,1]$ existe $\delta > 0$ tal que se $|s-t| < \delta$
            então $\gamma(s) \in D_t$ e $[f_s]_{\gamma(s)} = [f_t]_{\gamma(s)}$.
        \end{enumerate}
        Nesse caso, dizemos que $(f_1, D_1)$ é a continuação analítica de $(f_0,D_0)$ ao 
        longo de $\gamma$ ou ainda que $(f_1, D_1)$ é obtido a partir de $(f_0,D_0)$ por
        continuação analítica ao longo de $\gamma$.
    \end{definicao}


    \begin{proposicao}
    \label{prop-unicidade-continuacao-analitica-caminho}
        Seja $\gamma: [0,1]\to\C$ um caminho de $a$ para $b$ e sejam
        \begin{align*}
            \left\{ (f_t, D_t): 0\leq t\leq 1 \right\}, \\
            \left\{ (g_t, B_t): 0\leq t\leq 1 \right\}
        \end{align*}
        continuações analíticas ao longo de $\gamma$ tais que $[f_0]_a = [g_0]_a$.
        Então $[f_1]_b = [g_1]_b$.
    \end{proposicao}

    \begin{proof}
        A ideia é mostrar que
        %
        \begin{equation*}
            T = \left\{ t\in [0,1] : [f_t]_{\gamma(t)}= [g_t]_{\gamma(t)} \right\}
        \end{equation*}
        %
        é aberto e fechado em $[0,1]$ (com a topologia induzida pela topologia usual da reta).
        Isso, juntamente aos fatos de que $T$ é não-vazio ($0\in T$) e $[0,1]$ é conexo, 
        nos permitirá concluir que $T = [0,1]$ e, em particular, $1\in T$.
        
        Para mostrar que $T$ é aberto, seja $t\in T$ não nulo e diferente de 1 
        (se $t=1$ terminamos). Por definição de continuação analítica, existe $\delta > 0$
        tal que se $|s-t|<\delta$ então $\gamma(s) \in D_t\cap B_t$ e
        %
        \begin{equation*}
            \begin{cases}
            [f_s]_{\gamma(s)} = [f_t]_{\gamma(s)} \\
            [g_s]_{\gamma(s)} = [g_t]_{\gamma(s)}.
            \end{cases}
        \end{equation*}
        %
        Seja $H\subseteq D_t\cap B_t$ tal que $\gamma(s), \gamma(t)\in H$. 
        %
        \begin{center}
            \textbf{diagrama slide 17 aula 2}
        \end{center}
        %
        Como $t\in T$,
        $f_t(z) = g_t(z), \forall z\in H$. Logo, $[f_t]_{\gamma(s)} = [g_t]_{\gamma(s)}$
        para todo $\gamma(s)\in D_t\cap D_s$. Segue então das duas igualdades acima que
        $[f_s]_{\gamma(s)} = [g_s]_{\gamma(s)}$ se $|s-t|<\delta$. Logo, 
        $(t-\delta, t+\delta)\subseteq T$ e, como $t$ é um ponto qualquer de $T$, temos
        $T$ aberto em $[0,1]$.
        
        Agora, para mostrar que $T$ é também fechado em $[0,1]$, tome $t$ um ponto 
        limite de $T$ e escolha $\delta>0$ tal que $\gamma(s) \in D_t\cap B_t$ e
        %
        \begin{equation*}
            \begin{cases}
                [f_s]_{\gamma(s)} = [f_t]_{\gamma(s)} \\
                [g_s]_{\gamma(s)} = [g_t]_{\gamma(s)}.
            \end{cases}
        \end{equation*}
        %
        sempre que $|s-t|<\delta$. Como $t$ é ponto limite de $T$, existe $s\in T$ com
        $|s-t|<\delta$. Seja então $G$ um domínio tal que 
        $\gamma((t-\delta, t+\delta))\subseteq G\subseteq D_t\cap B_t$.
        Em particular, temos $\gamma(s)\in G$ e, por definição de $T$, segue que
        $f_s(z) = g_s(z), \forall z\in G$. Usando novamente as duas igualdades anteriores, 
        podemos garantir que
        %
        \begin{equation*}
            \begin{cases}
                f_s(z) = f_t(z) \\
                g_s(z) = g_t(z)
            \end{cases}, \forall z\in G
        \end{equation*}
        %
        donde segue que $f_t(z) = g_t(z), \forall z\in G$. Ora, já que $G$ é aberto 
        (e portanto possui um ponto limite em $D_t\cap B_t$), segue que 
        $[f_t]_{\gamma(t)} = [g_t]_{\gamma(t)}$, ou seja, $t\in T$ e, consequentemente,
        $T$ contém todos os seus pontos limites, i.e., é fechado.
    \end{proof}


    \begin{observacao}
        Note que o caminho $\gamma$ na 
        Proposição \ref{prop-unicidade-continuacao-analitica-caminho} é o
        mesmo para $f$ e $g$! Isso será muito relevante a seguir devido ao 
        seguinte exemplo.
        
        Defina $\arg_0: \left\{ z\in\C : \Re(z) > 0 \right\} \to (-\pi/2, \pi/2)$,
        que é uma restrição do ramo principal do argumento a um domínio mais ``restrito''.
        %
        \begin{center}
            \textbf{diagrama slide 4 aula 3}
        \end{center}
        %
        Considere também a função analítica 
        $f: \left\{ z\in\C : \Re(z) > 0 \right\} \to\C$
        dada por $f(z) = \ln(z) + i\arg_0(z)$.
        %
        \begin{center}
            \textbf{diagrama slide 5 aula 3}
        \end{center}
        %
        Vamos definir na região do diagrama da esquerda uma continuação analítica
        $\left\{ (f_t, D_t), t\in [0,1] \right\}$ ao longo de $\gamma_1$, 
        com $-1\in D_1$, $f_1(-1) = i\pi/2$,
        $D_0 = \left\{ z\in\C : \Re(z) > 0 \right\}$ e $f_0 = f$.
        
        Na região mostrada no diagrama da direita, vamos definir uma continuação analítica
        $\left\{ (g_t, B_t), t\in [0,1] \right\}$ ao longo de $\gamma_2$,
        com $-1\in B_1$, $g_1(-1) = -i\pi/2$, 
        $B_0 = \left\{ z\in\C : \Re(z) > 0 \right\}$ e $g_0 = f$.
        
        Note que podemos usar essas continuações analíticas para construir duas
        \textbf{extensões analíticas} $F:U_1\to\C$ e $G:U_2\to\C$, 
        com $F(z) = f_t(z), z\in D_t$ e $G(z) = g_t(z), z\in B_t$, 
        sendo $U_1$ e $U_2$ as uniões do domínio original de $f$
        com as regiões azul e vermelha, respectivamente. Ora, mas note que
        $F$ e $G$, apesar de coincidirem no domínio original de $f$, são distintas!
        
        Esse fato parece, à primeira vista, contradizer o princípio da identidade. 
        Entretanto, isso não acontece porque $F$ e $G$ não estão definidas no mesmo
        domínio. À segunda vista, pode parecer que a conclusão em que chegamos contradiz
        a Proposição \ref{prop-unicidade-continuacao-analitica-caminho}. Mas isso também 
        não ocorre, porque $F$ e $G$ foram construídas por caminhos distintos!
    \end{observacao}


    \begin{definicao}
    \label{def-continuacao-analitica-germe}
        Se $\gamma: [0,1]\to\C$ é um caminho de $a$ para $b$ e 
        $\left\{ (f_t, D_t): 0\leq t\leq 1 \right\}$ é uma continuação analítica ao
        longo de $\gamma$, então o germe $[f_1]_b$ é a continuação analítica de
        $[f_0]_a$ ao longo de $\gamma$.
    \end{definicao}

Note que é preciso usar a Proposição \ref{prop-unicidade-continuacao-analitica-caminho}
para verificar que a definição acima está bem posta. Ademais, note que ela torna
as coisas mais práticas, pois nos permite simplificar a descrição dos $D_t$'s.

    \begin{observacao}
        Sejam $U_1, U_2\subseteq\C$ domínios não-vazios tais que $U_1\cap U_2$
        também é um domínio não-vazio.
        %
        \begin{center}
            \textbf{diagrama slide 9 aula 3}
        \end{center}
        %
        Sejam ainda $f:U_1\to\C$ e $g:U_2\to\C$ funções analíticas
        tais que $f(z) = g(z), \forall z\in U_1\cap U_2$. Nesse caso, dizemos que
        $g$ é uma continuação analítica de $f$ a $U_2$ e vice-versa: $f$ é 
        uma continuação analítica de $g$ a $U_1$.
        
        Essas duas aplicações podem ser usadas para construir uma extensão analítica
        $F: U_1\cup U_2\to\C$ dada por partes:
        %
        \begin{equation*}
            F(z) = 
            \begin{cases}
                f(z), z\in U_1; \\
                g(z), z\in U_2.
            \end{cases}
        \end{equation*}
        %
        Nesse caso, $(f,U_1)$ e $(g,U_2)$ são os elementos de função de $F$. 
        Essa é a motivação para a definição de \textit{elemento de função}.
        
        Vale reiterar que extensões analíticas por continuações analíticas distintas
        não precisam coincidir. No diagrama abaixo temos $F: U\cup V\to\C$
        e $G:U\cup W\to\C$ analíticas, com
        %
        \begin{align*}
            F(z) &= 
            \begin{cases}
                f(z), z\in U; \\
                g(z), z\in V.
            \end{cases}, \\
            G(z) &= 
            \begin{cases}
                f(z), z\in U; \\
                h(z), z\in W.
            \end{cases}
        \end{align*}
        %
        mas $F\neq G$.
        %
        \begin{center}
            \textbf{diagrama slide 11 aula 3}
        \end{center}
        %
    \end{observacao}

Damos sequência com duas definições que, à primeira vista, não parecem ter 
nada a ver com os termos por elas definidos.

    \begin{definicao}
    \label{def-funcao-analitica-completa}
        Se $(f,G)$ é um elemento de função, então a função analítica completa obtida
        a partir de $(f,G)$ é a coleção $\mathcal{F}$ de todos os germes $[g]_b$ 
        para os quais existem $a\in G$ e um caminho $\gamma$ de $a$ para $b$ 
        tais que $[g]_b$ é a continuação analítica de $[f]_a$ ao longo de $\gamma$.
    \end{definicao}


    \begin{definicao}
        Uma coleção de germes $\mathcal{F}$ é chamada função analítica completa se
        existe um elemento de função $(f,G)$ tal que $\mathcal{F}$ é a função analítica
        completa obtida de $(f,G)$.
    \end{definicao}

Note que esta definição nos permite ``esquecer'' do ponto $a$ mencionado na
Definição \ref{def-funcao-analitica-completa}. Podemos escolher qualquer ponto
de $G$, já que ele é um domínio.

A pergunta natural a se fazer é: ``por que a terminologia de 
\textbf{função analítica completa} é adequada quando nos referimos a uma
coleção de germes?'', e a resposta é surpreendente. Vamos mostrar que
podemos construir uma função a partir da coleção $\mathcal{F}$ de germes.

Inicialmente, consideremos a coleção
%
\begin{equation*}
    \mathcal{R} = \left\{ (z, [g]_z) : [g]_z \in \mathcal{F} \right\}.
\end{equation*}
%
Definimos então $\widetilde{\mathcal{F}}:\mathcal{R}\to\C$ por
%
\begin{equation*}
    \widetilde{\mathcal{F}}((z, [g]_z)) = g(z),
\end{equation*}
%
ou seja, escolhemos um ``representante'' do germe e calculamos sua imagem em $z$.

{\red Falar do teorema de uniformização...}

\section{Teorema de Monodromia}

Monodromia é o estudo de como objetos da Análise, Topologia Algébrica, Geometria
Algébrica e Geometria Diferencial se comportam à medida que 
``circundam uma singularidade''. O dá origem ao fenômeno de monodromia são certas
funções que gostaríamos de definir falham em ter um valor único quando ``circulam''
uma singularidade (e.g., o logaritmo).

Como vimos acima com o exemplo do logaritmo, continuações analíticas em 
caminhos distintos não precisam coincidir. O leitor familiar com teoria de
homotopia poderia pensar que uma condição suficiente para a
igualdade das continuações pode ser dada em termos dessa teoria. Entretanto,
como todas as curvas no plano (complexo) são homotópicas, o resultado teria
de ser fraseado em termos de homotopia em um domínio próprio de $\C$,
que é o conteúdo do teorema de monodromia.

Ainda assim, em domínios mais ``malucos'', a teoria de homotopia não é suficiente
e é preciso apelar para a homologia. Manteremos a discussão no nível mais ``baixo'',
sem utilizar a linguagem de homologia, haja vista que tais
ferramentas são deveras \textit{overkill} para os nossos propósitos.

Retornando à discussão sobre o teorema de monodromia: sabemos que se $(f,D)$ é
um elemento de função e $a\in D$, então $f$ tem uma expansão em série de potências
em torno de $a$. O primeiro passo para demonstrar o teorema de Monodromia será
estudar o comportamento do raio de convergência da continuação analítica ao longo
de um caminho.

    \begin{lema}
    \label{lema-raio-convergencia-continuo}
        Sejam $\gamma: [0,1]\to\C$ um caminho e 
        $\left\{ (f_t, D_t): 0\leq t\leq 1 \right\}$ uma continuação analítica ao longo de
        $\gamma$. Para $t\in[0,1]$, seja $R(t)$ o raio de convergência da expansão em 
        série de potências de $f_t$ em torno de $z=\gamma(t)$. Então ou $R(t) \equiv\infty$
        ou $R: [0,1]\to (0, \infty)$ é uma função contínua.
    \end{lema}

    \begin{proof}
        Suponha, inicialmente, que existe $0\leq t\leq 1$ tal que $R(t)\equiv\infty$.
        Vamos mostrar que
        %
        \begin{equation*}
            T = \left\{ t\in[0,1] : R(t)\equiv\infty \right\}
        \end{equation*}
        %
        é fechado e aberto, donde seguirá que $T = [0,1]$ por conexidade.
        
        Da hipótese de existência de um $t\in T$, podemos escrever
        %
        \begin{equation*}
            f_t(z) = \sum_{n=0}^\infty a_n(\gamma(t))(z - \gamma(t))^n, \forall z\in D_t
        \end{equation*}
        %
        e, ademais, a séria acima define uma função inteira (devido ao raio de convergência
        infinito). Como $D_t$ é conexo, podemos estender $f_t$ a uma função inteira $F_t$.
        
        Pelas condições de compatibilidade da Definição \ref{def-continuacao-analitica},
        existe $\delta > 0$ tal que se $|s-t|<\delta$ então 
        $[f_t]_{\gamma(s)} = [f_s]_{\gamma(s)}$. Portanto, se
        $(F_t,\C) \in [f_t]_{\gamma(s)}$ então $(F_t,\C) \in [f_s]_{\gamma(s)}$.
        Note que $(F_t,\C) \in [f_s]_{\gamma(s)}$ implica $F_t(z) = f_s(z)$, para todo
        $z\in D_s$. Portanto, 
        %
        \begin{align*}
            \sum_{n=0}^\infty \widetilde{a_n}(\gamma(s))(z-\gamma(s))^n 
            =
            F_t(z)
            =
            f_s(z)
            =
            \sum_{n=0}^\infty a_n(\gamma(s))(z-\gamma(s))^n,
        \end{align*}
        %
        onde a série à direita é a expansão em série de potências de $F_t$ em torno 
        de $\gamma(s)$. Pelo princípio da identidade, devemos ter 
        $a_n(\gamma(s)) = \widetilde{a_n(\gamma(s))}, \forall n$ e, como o raio 
        de convergência da série à direita é infinito (pois $F_t$ é inteira), segue
        que $R(s) = \infty$. Portanto, temos $T$ aberto.
        
        Para mostrar que $T$ é fechado, seja $(s_n)_{n\in\N}\subseteq T$ uma
        sequência em $T$ que converge para um $t$ fixado. Sabemos que existe 
        $\delta = \delta(t)$ tal que se $|s-t|<\delta$ então 
        $[f_t]_{\gamma(s)} = [f_s]_{\gamma(s)}$. Como $s_n\xrightarrow{n\to\infty} t$,
        existe $n_0\in\N$ tal que se $n\geq n_0$ então $|s_n - t|<\delta$.
        Logo, para um $n$ satisfazendo tal condição, temos
        $[f_t]_{\gamma(s_n)} = [f_{s_n}]_{\gamma(s_n)}$. Temos então
        %
        \begin{align*}
            \sum_{j=0}^\infty \widetilde{a_j}(\gamma(t))(z-\gamma(t))^j
            =
            f_t(z)
            \stackrel{z\in D_{s_n}}{=}
            f_{s_n}(z)
            =
            \sum_{j=0}^\infty a_n(\gamma(t))(z-\gamma(t))^n.
        \end{align*}
        %
        Ora, então $\widetilde{a_j}(\gamma(t)) = a_j(\gamma(t))$ e, por hipótese, temos
        %
        \begin{equation*}
            R(t) := \frac{1}{\displaystyle{\limsup_{j\to\infty}} \sqrt{ |a_j(\gamma(t))| } } 
            = \infty,
        \end{equation*}
        %
        de modo que $R(t) = \infty$ e, portanto, $T$ é fechado.
        
        Segue então que se existe $t\in [0,1]$ tal que $R(t) = \infty$, então
        $R(t) \equiv\infty$ em $[0,1]$.
        
        Resta analisar o caso $R(t) < \infty, \forall t\in[0,1]$. Por analiticidade, podemos
        supor também que $R(t) > 0$ em $[0,1]$.
        
        Fixe $t\in [0,1]$ e considere a expansão em série de potência
        %
        \begin{equation*}
            f_t(z) = \sum_{n=0}^{\infty} a_n(\gamma(t))(z-\gamma(t))^n.
        \end{equation*}
        %
        Seja $\delta_1$ tal que 
        %
        \begin{enumerate}[i)]
            \item se $|s-t|<\delta_1$ então $\gamma(s)\in D_t\cap D(\gamma(t), R(t));$
            %
            \begin{center}
                \textbf{diagrama slide 31 aula 3}
            \end{center}
            %
            \item $[f_t]_{\gamma(s)} = [f_s]_{\gamma(s)}$.
        \end{enumerate}
        %
        Fixe $s$ tal que $|s-t<\delta_1|$. Note que podemos pensar em $f_t$ como uma 
        função analítica em $D(\gamma(t), R(t))$ (basta estendê-la analiticamente caso
        $D(\gamma(t), R(t))$ tenha pontos fora de $D_t$). Note, ainda, que existe uma 
        vizinhança $W\subseteq\C$ de $\gamma(s)$ tal que 
        $f_t(z) = f_s(z), \forall z\in W$.
        %
        \begin{center}
            \textbf{diagrama slide 32 aula 3}
        \end{center}
        %
        A menos de estender analiticamente, podemos pensar em $f_s$ como uma função analítica
        em $D_s\cup D(\gamma(t), R(t))$. Suponha que
        %
        \begin{equation*}
            f_s(z) = \sum_{n=0}^\infty a_n(\gamma(s))(z-\gamma(s))^n
        \end{equation*}
        %
        seja a expansão em série de potências de $f_s$ em torno de $\gamma(s)$. Temos então
        %
        \begin{equation*}
            R(t) - |\gamma(t) - \gamma(s)| \leq R(s).
        \end{equation*}
        %
        Por outro lado, podemos também pensar em $f_t$ como sendo analítica no conjunto
        $D_t\cup D(\gamma(s), R(s))$ e expandir
        %
        \begin{equation*}
            f_t(z) = \sum_{n=0}^\infty a_n(\gamma(t))(z-\gamma(t))^n,
        \end{equation*}
        %
        obtendo
        %
        \begin{equation*}
            R(s) - |\gamma(t) - \gamma(s)| \leq R(t).
        \end{equation*}
        %
        Juntando as duas desigualdade, obtemos
        %
        \begin{equation*}
            -|\gamma(t) - \gamma(s)| \leq R(t) - R(s) \leq |\gamma(t) - \gamma(s)| 
            \iff
            |R(t) - R(s)| \leq |\gamma(t) - \gamma(s)|.
        \end{equation*}
        %
        A última desigualdade implica, então, que $t\mapsto R(t)$ define uma função contínua,
        encerrando a demonstração.
    \end{proof}

É interessante observar que o final da demonstração anterior nos permite dizer
mais do que ``$R(t)$ é uma função contínua em $[0,1]$''. De fato, definindo o que
vem a ser um 
\href{https://en.wikipedia.org/wiki/Modulus_of_continuity}{módulo de continuidade},
notamos que $R(t)$ tem o mesmo módulo de continuidade que $\gamma$; por exemplo, se
$\gamma$ é Lipschitz, então $R(t)$ também o é 
(e isso é facilmente verificado com a última desigualdade da demonstração).
Dito de outro modo, $R(t)$ herda o módulo de continuidade de $\gamma$.

    \begin{lema}
    \label{lema-ext-epsilon-proximas}
        Seja $\gamma: [0,1] \to \C$ um caminho do ponto $a$ para o ponto $b$ e seja
        $\{(f_t, D_t): 0 \leq t \leq 1\}$ uma continuação analítica ao longo de $\gamma$. 
        Existe um $\varepsilon > 0$ tal que, se $\sigma :[0,1] \to \C$ é um caminho 
        de $a$ para $b$ satisfazendo $|\gamma(t) - \sigma(t)| < \varepsilon$ para todo $t$ 
        e $\{(g_t, B_t): 0 \leq t \leq 1\}$ é uma continuação analítica ao longo de $\sigma$ 
        com $[f_0]_a = [g_0]_a$, então $[f_1]_b = [g_1]_b$.
    \end{lema}

    \begin{proof}
        Novamente, aqui, $R(t)$ é o raio de convergência da série de potências da função 
        $f_t$ em torno de $\gamma(t)$. Pelo Lema anterior, temos que $R(t) \equiv \infty$ 
        ou $0 < R(t) < \infty$, onde $R(t) > 0$ devido ao fato de que cada função $f_t$ é
        analítica em $D_t$. 
        
        Se $R(t) \equiv \infty$, então qualquer $\varepsilon > 0$ escolhido é suficiente 
        para que o Lema seja válido. Como o raio de convergência é infinito para cada $t$,
        podemos estender a função $g_t$ analiticamente para a função definida pela série de
        potências da expansão de $f_t$ usando o princípio da identidade.
        
        Suponha agora que $0 < R(t) < \infty$. No restante da demonstração, vamos supor que 
        $D_t$ é o disco de centro $\gamma(t)$ e raio $R(t)$. Isto pode ser feito, pois, caso
        $D_t$ não fosse um disco, então, ou $D_t \subseteq D(\gamma(t),R(t))$ ou
        $D(\gamma(t),R(t)) \subseteq D_t$ e poderíamos sempre estender $f_t$ usando o 
        princípio da identidade. Por motivos inteiramente análogos, podemos supor também
        que $B_t$ são discos centrados em $\beta(t)$. Essas suposições simplificarão 
        os argumentos.
        %
        \begin{center}
            \textbf{diagrama slide 5 Aula 4}
        \end{center}
        %
        Escolha um $\varepsilon$ satisfazendo
        %
        \begin{equation*}
        0 < \varepsilon < \frac{1}{2}\min\{R(t): t \in [0,1]\}.
        \end{equation*}
        %
        Note que $R$ alcança o mínimo pois é uma função contínua no compacto $[0,1]$. 
        Dito de outro modo, existe $t_* \in [0,1]$ tal que $R(t_*) = \min\{R(t): t \in [0,1]\}$.
        
        Para cada $t \in [0,1]$, temos $|\gamma(t) - \sigma(t)| < \varepsilon < R(t)$, então
        $\sigma(t) \in B_t \cap D_t$ para todo $t$. Queremos mostrar que $[f_1]_b = [g_1]_b$, 
        o que pode ser feito mostrando que $f_1(z) = g_1(z)$ para todo $z \in B_1 \cap D_1$. 
        Como é difícil mostrar isso para o valor particular $t=1$ diretamente, vamos passar 
        a uma estratégia que já adotamos antes: mostrar que um determinado conjunto é aberto
        e fechado relativo a $[0,1]$.
        
        Defina
        %
        \begin{equation*}
        T = \{t \in [0,1]: f_t(z) = g_t(z) \forall z \in B_t \cap D_t\}.
        \end{equation*}
        %
        $T$ não é vazio por conta da hipótese $[f_0]_a = [g_0]_a$. Vamos mostrar que $T$ é
        aberto e fechado no conexo $[0,1]$, ou seja, $T = [0,1]$. Seja $t \in T$; 
        existe um $\delta > 0$ tal que
        %
        \begin{align}
        \label{Eq5}
            &|\gamma(s) - \gamma(t)| < \varepsilon, \ \ \ \ \ [f_s]_{\gamma(s)} =
            [f_t]_{\gamma(s)} \\
            &|\sigma(s) - \sigma(t)| < \varepsilon, \ \ \ \ \ [g_s]_{\gamma(s)} =
            [g_t]_{\gamma(s)} \\
            &\sigma(s) \in B_t
        \end{align}
        %
        sempre que $|s-t|<\delta$. Aqui, usamos a continuidade da funções $\gamma$ e $\sigma$ 
        e a condição de compatibilidade da definição de continuação analítica ao longo de um
        caminho.
        
        Seja $G = B_s \cap B_t \cap D_s \cap D_t$, como ilustrado abaixo. 
        %
        \begin{center}
            \textbf{diagrama slide 8 Aula 4}
        \end{center}
        %
        Vamos mostrar que 
        $\sigma(s) \in G$, de modo que $G \neq \emptyset$. Pela definição de continuação 
        ao longo de um caminho, temos $\sigma(s) \in B_t \cap B_s$. Da hipótese do Lema,
        %
        \begin{equation*}
        |\sigma(s) - \gamma(s)| < \varepsilon < R(s),
        \end{equation*}
        %
        donde segue que $\sigma(s) \in D_s$. Usando a desigualdade triangular e as condições em \ref{Eq5}, temos que
        %
        \begin{align*}
        |\sigma(s) - \gamma(t)| &= |\sigma(s) - \gamma(s) + \gamma(s) - \gamma(t)| \\
        &\leq |\sigma(s) - \gamma(s)| + |\gamma(s) - \gamma(t)| \\
        &< 2\varepsilon < R(t).
        \end{align*}
        %
        Isto mostra que $\sigma(s) \in D_t$ e, portanto, $\sigma(s) \in G$.
        
        Como $t \in T$, temos que $f_t(z) = g_t(z)$ para todo $z$ em $G$. 
        As condições em \ref{Eq5} implicam que 
        %
        \begin{align*}
            \begin{cases}
                f_s(z) = f_t(z) \\
                g_s(z) = g_t(z)
            \end{cases}
        \end{align*}
        %
        sempre que $z \in G$. Segue que $f_s(z) = g_s(z)$ para todo $z \in G$. 
        Pelo princípio da identidade, podemos concluir que 
        $f_s(z) = g_s(z) \forall z \in D_s \cap B_s$, ou seja, $s \in T$ sempre que 
        $|s-t|< \delta$, o que mostra que $T$ é aberto.
        
        Vamos mostrar agora que $T$ é também fechado. Lembre-se que um conjunto é fechado 
        se, e somente se, contém todos os seus pontos de acumulação. Sejam $t$ um ponto de
        acumulação de $T$ e $\{s_n\}$ uma sequência de pontos em $T$ com $s_n \to t$. 
        Novamente, existe um $\delta > 0$ tal que valem as condições em \ref{Eq5} sempre que
        $|s-t|<\delta$; existe também um $N \in \N$ tal que
        %
        \begin{equation*}
        n \geq N \implies |s_n - t| < \delta.
        \end{equation*}
        %
        Aplicando um raciocínio totalmente análogo ao que fizemos antes, mostra-se que
        $\sigma(s_n) \in G = B_s \cap B_t \cap D_s \cap D_t$. Como consequência das 
        condições de compatibilidade análogas àquelas listadas em \ref{Eq5} 
        (apenas trocando $s \leftrightarrow s_n)$, temos
        %
        \begin{align*}
            \begin{cases}
                f_{s_n}(z) = f_t(z) \\
                g_{s_n}(z) = g_t(z)
            \end{cases}
        \end{align*}
        %
        sempre que $z \in G$. Agora, $s_n \in T$, então
        %
        \begin{equation*}
            f_t(z) = f_{s_n}(z) = g_{s_n}(z) = g_t(z), \forall z\in G
        \end{equation*}
        %
        Aplicando o princípio da identidade, concluímos que $t \in T$. Logo, $T$ é fechado. 
        Isto demonstra o Lema.
    \end{proof}

    \begin{definicao}
    \label{def-continuacao-irrestrita}
        Sejam $(f,D)$ um elemento de função e $G\subseteq\C$ domínio tal que 
        $D\subseteq G$. Vamos dizer que $f$ admite uma continuação analítica irrestrita a 
        $G$ se para qualquer caminho $\gamma$ com ponto inicial em $D$ existe uma continuação
        analítica $\{(f_t, D_t) : 0\leq t\leq 1\}$ ao longo de $\gamma$.
    \end{definicao}

É importante frisar, novamente, que $f$ admitir continuação analítica irrestrita 
\textbf{não significa} que as continuações por caminhos distintos irão coincidir.
Veja o seguinte exemplo.

    \begin{exemplo}
        Sejam $D = D(1,1) = \{ z\in\C : |z-1|<1 \}$ e $f:D\to\C$ dada por
        $f(z) = \displaystyle{\exp\left( \frac{1}{2}\log_0(z) \right)}$, onde
        $\log_0:D\to\C$ é a restrição do ramo principal do logaritmo ao disco aberto
        $D(1,1)$. Se $G = \C^*$, então $(f,D)$ admite continuação analítica irrestrita
        a $G$.
        
        Entretanto, as continuações analíticas podem ser distintas, como vimos quando
        construímos duas continuações analíticas do logaritmo tais que uma valia 
        $i\pi/2$ em $-1$ e outra valia $-i\pi/2$ em $-1$.
    \end{exemplo}

    \begin{observacao}
        Continuações analíticas irrestritas não são um fato da vida!
        
        Falar de
        %
        \begin{equation*}
            z\mapsto \int_{\gamma} \frac{f(w)}{w-z}dw
        \end{equation*}
        %
        definir uma função analítica com singularidades não isoladas 
        (conjunto de singularidades denso).
    \end{observacao}

    \begin{definicao}[Curvas FEP-homotópicas]
    \label{def-fep-homotopicas}
        Se $\gamma_0, \gamma_1 :[0,1]\to G\subseteq\C$ são curvas 
        suaves por partes tomando valores em $G$ e tais que 
        $\gamma_0(0) = \gamma_1(0) = a$ e $\gamma_0(1) = \gamma_1(1) = b$, 
        dizemos que elas são \textit{fixed-end-points}
        homotópicas (FEP-homotópicas) se existe uma aplicação contínua 
        $\Gamma:[0,1]\times [0,1] \to G\subseteq\C$ tal que
        %
        \begin{align*}
            \Gamma(t,0) &= \gamma_0(t) \qquad \Gamma(t,1) = \gamma_1(t) \\
            \Gamma(0,s) &= a \qquad \Gamma(1,s) = b
        \end{align*}
        %
        \begin{center}
            \textbf{diagrama slide 15 Aula 4}
        \end{center}
        %
    \end{definicao}

    \begin{definicao}
    \label{def-homotopica-a-zero}
        Seja $\gamma:[0,1]\to G\subseteq\C$ uma curva fechada suave por partes.
        Dizemos que $\gamma$ é homotópica a zero, denotando $\gamma\sim 0$, se $\gamma$
        é homotópica a uma curva constante, ou seja, existe uma aplicação contínua 
        $\Gamma:[0,1]\times [0,1] \to G\subseteq\C$ tal que
        %
        \begin{align*}
            \Gamma(0,s) &= \Gamma(1,s) \\
            \Gamma(t,1) &= \Gamma(t',1), \forall t, t'\in [0,1] \\
            \Gamma(t,0) &= \gamma(t)
        \end{align*}
        %
        \begin{center}
            \textbf{diagrama curva homotópica a zero}
        \end{center}
        %
    \end{definicao}

    \begin{definicao}
    \label{def-simplesmente-conexo}
        Um domínio $G\subseteq\C$ é simplesmente conexo se toda curva fechada
        suave por partes $\gamma: [0,1]\to G\subseteq\C$ é homotópica a 0.
        %
        \begin{center}
            \textbf{diagrama slide 17 Aula 4}
        \end{center}
        %
    \end{definicao}

    \begin{teorema}[Teorema de Monodromia]
    \label{teo-monodromia}
        Sejam $(f,D)$ um elemento de função e $G\subseteq\C$ um domínio tal que 
        $D\subseteq G$ e $(f,D)$ admite continuação analítica irrestrita em $G$. Sejam
        $a\in D, b\in G$, $\gamma_0, \gamma_1$ caminhos em $G$ e 
        $\{ (f_t, D_t) : 0\leq t\leq 1 \}$, $\{ (g_t, B_t) : 0\leq t\leq 1 \}$ continuações
        analíticas ao longo de $\gamma_0$ e $\gamma_1$, respectivamente. 
        Se $\gamma_0$ e $\gamma_1$ são FEP-homotópicas em $G$, então $[f_1]_b = [g_1]_b$.
    \end{teorema}

    \begin{proof}
        Já que $\gamma_0, \gamma_1$ são FEP-homotópicas em $G$, existe uma função contínua
        $\Gamma: [0,1]\times[0,1]\to G\subseteq\C$ tal que
        %
        \begin{align*}
            \Gamma(t,0) &= \gamma_0(t) \qquad \Gamma(t,1) = \gamma_1(t) \\
            \Gamma(0,s) &= a \qquad \Gamma(1,s) = b
        \end{align*}
        %
        para todo $s\in [0,1]$. Fixe um tal $s$ e considere o caminho 
        $\gamma_2(t) := \Gamma(t,s)$. Por hipótese, existe uma continuação analítica
        $\{ (h_{t,s}, D_{t,s}) : 0\leq t\leq 1 \}$ de $(f,D)$ ao longo de 
        $\gamma_2 \equiv \Gamma(\cdot, s)$. Pela 
        Proposição \ref{prop-unicidade-continuacao-analitica-caminho}, sabemos que
        %
        \begin{align*}
            [h_{1,1}]_b &= [g_1]_b, \\
            [h_{1,0}]_b &= [f_1]_b.
        \end{align*}
        %
        Portanto, para atingir o nosso objetivo (que é mostrar que $[f_1]_b = [g_1]_b$),
        basta mostrar que $[h_{1,0}]_b = [h_{1,1}]_b$. Para tanto, vamos usar a técnica
        de sempre: argumentar por conexidade.
        
        Defina
        %
        \begin{equation*}
            U = \left\{ u\in [0,1] : [h_{1,u}]_b = [h_{1,0}]_b \right\}.
        \end{equation*}
        %
        Por hipótese, temos $0\in U$ e, portanto, $U\neq\varnothing$. Queremos mostrar que
        $1\in U$ e, para tanto, vamos mostrar que $U$ é simultaneamente aberto e fechado
        relativamente a $[0,1]$.
        
        Primeiro, afirmamos que dado $u\in [0,1]$ existe $\delta > 0$ tal que se 
        $|u - v| < \delta$ então $[h_{1,u}]_b = [h_{1,v}]_b$. De fato, fixado $u\in [0,1]$,
        aplique o Lema \ref{lema-ext-epsilon-proximas} para encontrar $\varepsilon > 0$
        tal que se $\sigma$ é um caminho qualquer de $a$ para $b$ com 
        $|\gamma_u(t) - \sigma(t)| < \varepsilon$ para todo $t$, e se 
        $\{ (k_t, E_t): 0\leq t\leq 1 \}$ é uma continuação analítica qualquer de $(f,D)$
        ao longo de $\sigma$, então
        %
        \begin{equation*}
            [h_{1,u}]_b = [k_1]_b.
        \end{equation*}
        %
        Como $\Gamma$ é uniformemente contínua, existe $\delta > 0$ tal que se $|u-v|<\delta$
        então $|\gamma_u(t) - \gamma_v(t)| = |\Gamma(t,u) - \Gamma(t,v)| < \varepsilon$
        para todo $t$. Usando que $[h_{1,u}]_b = [k_1]_b$, a nossa afirmação segue.
        
        Tomemos $u\in U$ e seja $\delta > 0$ como na afirmação acima. Por definição de $U$,
        temos $(u-\delta, u+\delta) \subset U$, ou seja, $U$ é aberto. Por outro lado, dado
        $u\in\overline{U}$ e escolhendo $\delta$ como na afirmação acima novamente, então 
        existe $v\in U$ tal que $|u-v| < \delta$. Mas, pela afirmação, 
        $[h_{1,u}]_b = [h_{1,v}]_b$; e, como $v\in U$, $[h_{1,v}]_b = [h_{1,0}]_b$.
        Logo, $[h_{1,u}]_b = [h_{1,0}]_b$, isto é, $u\in U$ e, portanto, $U$ é fechado.
        
        Como $[0,1]$ é conexo, segue que $U = [0,1]$ e $1\in U$, como desejado.
    \end{proof}

    \begin{corolario}
        Seja $(f,D)$ um elemento de função que admite continuação analítica irrestrita no
        domínio simplesmente conexo $G$, com $D\subseteq G$. Existe uma extensão analítica
        $F:G\to\C$ de $f:D\to\C$, isto é, $F\big|_D \equiv f$.
    \end{corolario}

    \begin{proof}
        Fixe $a\in D$. Se $\gamma$ é um caminho suave ligando $a$ e $z$ e se 
        $\{ (f_t, D_t) : 0\leq t\leq 1 \}$ é uma continuação analítica de $(f,D)$ ao longo de
        $\gamma$, defina $F(z, \gamma) \equiv f_1(z)$. Já que $G$ é simplesmente conexo,
        $F(z, \gamma) = F(z,\beta)$ para quaisquer dois caminhos suaves em $G$ ligando $a$ e $z$.
        Portanto, $z\mapsto F(z)\equiv F(z, \gamma)$ é uma função bem definida em $G$ 
        (pois independe do caminho escolhido).
        
        Afirmamos que essa função é analítica. De fato, $F(w) = f_1(w), \forall w\in D(z,r)$
        para algum $r$ positivo. Pelo Lema \ref{lema-ext-epsilon-proximas} segue a afirmação.
        Para algum $\Tilde{r}>0$ temos $D(a, \Tilde{r}) \subseteq D$ e 
        $f\big|_{D(a, \Tilde{r})} \equiv F$, de modo que $F\big|_G \equiv f$.
        %
        \begin{center}
            \textbf{diagrama slide 24 Aula 4}
        \end{center}
        %
    \end{proof}
















