\chapter[Continuação analítica]{Continuação analítica}
\chaptermark{}

\hfill%
\begin{minipage}{10cm}
\begin{flushright}
\rightskip=0.5cm
\textit{``The greatest strategy is doomed if it's implemented badly.''}
\\[0.1cm]
\rightskip=0.5cm
--- Bernhard Riemann
\end{flushright}
\end{minipage}

\section{O princípio da reflexão de Schwarz}

% acho que aqui é melhor seguir o Stein, só dar uma complementada no argumento
% falando do caso de apenas o vértice estar sobre o eixo real

\section{Continuação analítica ao longo de caminhos}

% fazer a discussão introdutória (slides da Aula 1 e início dos slides da Aula 2)
% finalizar a parte da Aula 3 (comentar teo da uniformização e começar a parte de monodromia)

\begin{definicao}
\label{def-elemento-funcao}
Um elemento de função é um par $(f, G)$ onde $G$ é um domínio e $f$
é uma função analítica em $G$. Dado um elemento de função $(f,G)$, 
definimos o germe de $f$ em $a$, denotado por $[f]_a$, 
como a coleção de todos os elementos de função $(g,D)$ tais que 
$a\in D$ e $f(z) = g(z)$ para todo $z$ em uma vizinhança de $a$.
\end{definicao}


\begin{definicao}
\label{def-continuacao-analitica}
Seja $\gamma: [0,1] \to \mathbb{C}$ um caminho e suponha que para cada
$t\in[0,1]$ existe um elemento de função $(f_t, D_t)$ tal que
\begin{enumerate}[(a)]
    \item $\gamma(t) \in D_t$;
    \item para cada $t\in[0,1]$ existe $\delta > 0$ tal que se $|s-t| < \delta$
    então $\gamma(s) \in D_t$ e $[f_s]_{\gamma(s)} = [f_t]_{\gamma(s)}$.
\end{enumerate}
Nesse caso, dizemos que $(f_1, D_1)$ é a continuação analítica de $(f_0,D_0)$ ao 
longo de $\gamma$ ou ainda que $(f_1, D_1)$ é obtido a partir de $(f_0,D_0)$ por
continuação analítica ao longo de $\gamma$.
\end{definicao}


\begin{proposicao}
\label{prop-unicidade-continuacao-analitica-caminho}
Seja $\gamma: [0,1]\to\mathbb{C}$ um caminho de $a$ para $b$ e sejam
\begin{align*}
    \left\{ (f_t, D_t): 0\leq t\leq 1 \right\}, \\
    \left\{ (g_t, B_t): 0\leq t\leq 1 \right\}
\end{align*}
continuações analíticas ao longo de $\gamma$ tais que $[f_0]_a = [g_0]_a$.
Então $[f_1]_b = [g_1]_b$.
\end{proposicao}

\begin{proof}
A ideia é mostrar que
\begin{equation*}
    T = \left\{ t\in [0,1] : [f_t]_{\gamma(t)}= [g_t]_{\gamma(t)} \right\}
\end{equation*}
é aberto e fechado em $[0,1]$ (com a topologia induzida pela topologia usual da reta).
Isso, juntamente aos fatos de que $T$ é não-vazio ($0\in T$) e $[0,1]$ é conexo, 
nos permitirá concluir que $T = [0,1]$ e, em particular, $1\in T$.

Para mostrar que $T$ é aberto, seja $t\in T$ não nulo e diferente de 1 
(se $t=1$ terminamos). Por definição de continuação analítica, existe $\delta > 0$
tal que se $|s-t|<\delta$ então $\gamma(s) \in D_t\cap B_t$ e 
\begin{equation*}
    \begin{cases}
    [f_s]_{\gamma(s)} = [f_t]_{\gamma(s)} \\
    [g_s]_{\gamma(s)} = [g_t]_{\gamma(s)}.
    \end{cases}
\end{equation*}
Seja $H\subseteq D_t\cap B_t$ tal que $\gamma(s), \gamma(t)\in H$. 

\begin{center}
    \textbf{diagrama slide 17 aula 2}
\end{center}

Como $t\in T$,
$f_t(z) = g_t(z), \forall z\in H$. Logo, $[f_t]_{\gamma(s)} = [g_t]_{\gamma(s)}$
para todo $\gamma(s)\in D_t\cap D_s$. Segue então das duas igualdades acima que
$[f_s]_{\gamma(s)} = [g_s]_{\gamma(s)}$ se $|s-t|<\delta$. Logo, 
$(t-\delta, t+\delta)\subseteq T$ e, como $t$ é um ponto qualquer de $T$, temos
$T$ aberto em $[0,1]$.

Agora, para mostrar que $T$ é também fechado em $[0,1]$, tome $t$ um ponto limite 
de $T$ e escolha $\delta>0$ tal que $\gamma(s) \in D_t\cap B_t$ e
\begin{equation*}
    \begin{cases}
    [f_s]_{\gamma(s)} = [f_t]_{\gamma(s)} \\
    [g_s]_{\gamma(s)} = [g_t]_{\gamma(s)}.
    \end{cases}
\end{equation*}
sempre que $|s-t|<\delta$. Como $t$ é ponto limite de $T$, existe $s\in T$ com
$|s-t|<\delta$. Seja então $G$ um domínio tal que 
$\gamma((t-\delta, t+\delta))\subseteq G\subseteq D_t\cap B_t$.
Em particular, temos $\gamma(s)\in G$ e, por definição de $T$, segue que
$f_s(z) = g_s(z), \forall z\in G$. Usando novamente as duas igualdades anteriores, 
podemos garantir que
\begin{equation*}
    \begin{cases}
    f_s(z) = f_t(z) \\
    g_s(z) = g_t(z)
    \end{cases}, \forall z\in G
\end{equation*}
donde segue que $f_t(z) = g_t(z), \forall z\in G$. Ora, já que $G$ é aberto 
(e portanto possui um ponto limite em $D_t\cap B_t$), segue que 
$[f_t]_{\gamma(t)} = [g_t]_{\gamma(t)}$, ou seja, $t\in T$ e, consequentemente,
$T$ contém todos os seus pontos limites, i.e., é fechado.
\end{proof}


\begin{observacao}
Note que o caminho $\gamma$ na 
Proposição \ref{prop-unicidade-continuacao-analitica-caminho} é o
mesmo para $f$ e $g$! Isso será muito relevante a seguir devido ao seguinte exemplo.

Defina $\arg_0: \left\{ z\in\mathbb{C} : \Re(z) > 0 \right\} \to (-\pi/2, \pi/2)$,
que é uma restrição do ramo principal do argumento a um domínio mais ``restrito''.

\begin{center}
    \textbf{diagrama slide 4 aula 3}
\end{center}

Considere também a função analítica 
$f: \left\{ z\in\mathbb{C} : \Re(z) > 0 \right\} \to\mathbb{C}$
dada por $f(z) = \ln(z) + i\arg_0(z)$.

\begin{center}
    \textbf{diagrama slide 5 aula 3}
\end{center}

Vamos definir na região do diagrama da esquerda uma continuação analítica
$\left\{ (f_t, D_t), t\in [0,1] \right\}$ ao longo de $\gamma_1$, 
com $-1\in D_1$, $f_1(-1) = i\pi/2$,
$D_0 = \left\{ z\in\mathbb{C} : \Re(z) > 0 \right\}$ e $f_0 = f$.

Na região mostrada no diagrama da direita, vamos definir uma continuação analítica
$\left\{ (g_t, B_t), t\in [0,1] \right\}$ ao longo de $\gamma_2$,
com $-1\in B_1$, $g_1(-1) = -i\pi/2$, 
$B_0 = \left\{ z\in\mathbb{C} : \Re(z) > 0 \right\}$ e $g_0 = f$.

Note que podemos usar essas continuações analíticas para construir duas
\textbf{extensões analíticas} $F:U_1\to\mathbb{C}$ e $G:U_2\to\mathbb{C}$, 
com $F(z) = f_t(z), z\in D_t$ e $G(z) = g_t(z), z\in B_t$, 
sendo $U_1$ e $U_2$ as uniões do domínio original de $f$
com as regiões azul e vermelha, respectivamente. Ora, mas note que
$F$ e $G$, apesar de coincidirem no domínio original de $f$, são distintas!

Esse fato parece, à primeira vista, contradizer o princípio da identidade. 
Entretanto, isso não acontece porque $F$ e $G$ não estão definidas no mesmo
domínio. À segunda vista, pode parecer que a conclusão em que chegamos contradiz
a Proposição \ref{prop-unicidade-continuacao-analitica-caminho}. Mas isso também 
não ocorre, porque $F$ e $G$ foram construídas por caminhos distintos!
\end{observacao}


\begin{definicao}
\label{def-continuacao-analitica-germe}
Se $\gamma: [0,1]\to\mathbb{C}$ é um caminho de $a$ para $b$ e 
$\left\{ (f_t, D_t): 0\leq t\leq 1 \right\}$ é uma continuação analítica ao
longo de $\gamma$, então o germe $[f_1]_b$ é a continuação analítica de
$[f_0]_a$ ao longo de $\gamma$.
\end{definicao}

Note que é preciso usar a Proposição \ref{prop-unicidade-continuacao-analitica-caminho}
para verificar que a definição acima está bem posta. Ademais, note que ela torna
as coisas mais práticas, pois nos permite simplificar a descrição dos $D_t$'s.

\begin{observacao}
Sejam $U_1, U_2\subseteq\mathbb{C}$ domínios não-vazios tais que $U_1\cap U_2$
também é um domínio não-vazio.

\begin{center}
    \textbf{diagrama slide 9 aula 3}
\end{center}

Sejam ainda $f:U_1\to\mathbb{C}$ e $g:U_2\to\mathbb{C}$ funções analíticas
tais que $f(z) = g(z), \forall z\in U_1\cap U_2$. Nesse caso, dizemos que
$g$ é uma continuação analítica de $f$ a $U_2$ e vice-versa: $f$ é 
uma continuação analítica de $g$ a $U_1$.

Essas duas aplicações podem ser usadas para construir uma extensão analítica
$F: U_1\cup U_2\to\mathbb{C}$ dada por partes:
\begin{equation*}
    F(z) = \begin{cases}
    f(z), z\in U_1; \\
    g(z), z\in U_2.
    \end{cases}
\end{equation*}
Nesse caso, $(f,U_1)$ e $(g,U_2)$ são os elementos de função de $F$. 
Essa é a motivação para a definição de \textit{elemento de função}.

Vale reiterar que extensões analíticas por continuações analíticas disintas
não precisam coincidir. No diagrama abaixo temos $F: U\cup V\to\mathbb{C}$
e $G:U\cup W\to\mathbb{C}$ analíticas, com
\begin{align*}
    F(z) &= \begin{cases}
    f(z), z\in U; \\
    g(z), z\in V.
    \end{cases}, \\
    G(z) &= \begin{cases}
    f(z), z\in U; \\
    h(z), z\in W.
    \end{cases}
\end{align*}
mas $F\neq G$.

\begin{center}
    \textbf{diagrama slide 11 aula 3}
\end{center}

\end{observacao}

Damos sequência com duas definições que, à primeira vista, não parecem ter 
nada a ver com os termos por elas definidos.

\begin{definicao}
\label{def-funcao-analitica-completa}
Se $(f,G)$ é um elemento de função, então a função analítica completa obtida
a partir de $(f,G)$ é a coleção $\mathcal{F}$ de todos os germes $[g]_b$ 
para os quais existem $a\in G$ e um caminho $\gamma$ de $a$ para $b$ 
tais que $[g]_b$ é a continuação analítica de $[f]_a$ ao longo de $\gamma$.
\end{definicao}


\begin{definicao}
Uma coleção de germes $\mathcal{F}$ é chamada função analítica completa se
existe um elemento de função $(f,G)$ tal que $\mathcal{F}$ é a função analítica
completa obtida de $(f,G)$.
\end{definicao}

Note que esta definição nos permite ``esquecer'' do ponto $a$ mencionado na
Definição \ref{def-funcao-analitica-completa}. Podemos escolher qualquer ponto
de $G$, já que ele é um domínio.

A pergunta natural a se fazer é: ``por que a terminologia de 
\textbf{função analítica completa} é adequada quando nos referimos a uma
coleção de germes?'', e a resposta é surpreendente. Vamos mostrar que
podemos construir uma função a partir da coleção $\mathcal{F}$ de germes.

Inicialmente, consideremos a coleção
\begin{equation*}
    \mathcal{R} = \left\{ (z, [g]_z) : [g]_z \in \mathcal{F} \right\}.
\end{equation*}
Definimos então $\widetilde{\mathcal{F}}:\mathcal{R}\to\mathbb{C}$ por
\begin{equation*}
    \widetilde{\mathcal{F}}((z, [g]_z)) = g(z),
\end{equation*}
ou seja, escolhemos um ``representante'' do germe e calculamos sua imagem em $z$.

{\red Falar do teorema de uniformização...}

\section{Teorema de Monodromia}

% explicar o termo "monodromia" (wikipedia)

\begin{lema}
\label{lema-raio-convergencia-continuo}
Sejam $\gamma: [0,1]\to\mathbb{C}$ um caminho e 
$\left\{ (f_t, D_t): 0\leq t\leq 1 \right\}$ uma continuação analítica ao longo de
$\gamma$. Para $t\in[0,1]$, seja $R(t)$ o raio de convergência da expansão em 
série de potências de $f_t$ em torno de $z=\gamma(t)$. Então ou $R(t) \equiv\infty$
ou $R: [0,1]\to (0, \infty)$ é uma função contínua.
\end{lema}