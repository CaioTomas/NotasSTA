% !TeX spellcheck = pt_BR
\chapter[Semana 4]{}
\chaptermark{}

\hfill%
\begin{minipage}{13cm}
\begin{flushright}
\rightskip=0.5cm
\textit{``
	... in effect, if one extends these functions by allowing complex values for the arguments, then 
	there arises a harmony and regularity which whithout it would remain hidden ''}
\\[0.1cm]
\rightskip=0.5cm
---B. Riemann, 1851
\end{flushright}
\end{minipage}




\section{Os Ramos do Argumento e do Logaritmo}

Vamos começar esta seção estuando funções conhecidas como ramos do argumento.
Elas serão de grande utilidade no estudo da função logaritmo complexo, ou melhor,
no estudo dos diversos ramos do logaritmo, como ficará claro a frente.

\begin{definicao}[Ramo do Argumento]
\label{def-ramo-argumento}
\index{Ramo!do argumento}
Seja $U\subset \mathbb{C}^{*}\equiv \mathbb{C}\setminus\{0\}$ um conjunto aberto conexo. Uma função contínua $\arg:U\to\mathbb{R}$
satisfazendo 

\begin{align}\label{def-eq-ramo}
\frac{z}{|z|}=\exp(i\arg(z)), \quad \forall z\in U,
\end{align}

é chamada de um ramo do argumento.
\end{definicao}


Não é uma tarefa completamente trivial construir ramos do argumento. Apesar 
deles estarem totalmente ligados a ideia intuitiva de ângulo expressá-los
analiticamente é às vezes um pouco trabalhoso. Antes de construirmos um ramo
propriamente dito, vamos estudar quais propriedades tais funções têm para
ir entendendo melhor como eles funcionam do ponto de vista analítico. 

\begin{proposicao}
Não existe nenhum ramo do argumento definido em todo $\mathbb{C}^{*}$, isto é,
não existe nenhuma função contínua definida em todo o conjunto $\mathbb{C}^{*}$
satisfazendo a igualdade \eqref{def-eq-ramo} para todo $z\in\mathbb{C}^{*}$.
\end{proposicao}

\begin{proof}
A prova é por contradição. Suponha que exista tal função $\arg:\mathbb{C}^{*}\to\mathbb{R}$. 
Considere a curva $\gamma:[-\pi,\pi]\to\mathbb{C}$ dada por 
$\gamma(t)=\cos t+i\sen t$. É claro que $\gamma$ é uma função contínua
de $[-\pi,\pi]$ em $\mathbb{C}$ e portanto a composta $\arg\circ \gamma$ é uma função 
contínua de $[-\pi,\pi]$ em $\mathbb{R}$. Por \eqref{def-eq-ramo} para qualquer 
$t\in [-\pi,\pi]$ temos que 
\begin{align*}
\cos t+i\sen t = \frac{\cos t+i\sen t}{|\cos t+i\sen t|}
&=
\exp(i\arg(\gamma(t)))
\\[0.2cm]
&=
\cos(\arg(\gamma(t)))+ i\sen(\arg(\gamma(t))).
\end{align*}
Desta forma $\arg(\gamma(t))=t+2k(t)\pi$, onde para cada $t\in [-\pi,\pi]$ 
temos $k(t)\in\mathbb{Z}$. Já que $t\longmapsto \arg(\gamma(t))$ é 
contínua e $[-\pi,\pi]$ é conexo segue que $k(t)\equiv k$, 
onde $k$ é alguma constante inteira. Logo $\arg(\gamma(t))=t+2k\pi$
para todo $t\in [-\pi,\pi]$. 
Pela definição de $\gamma$ temos que se $t\to \pi$ então $\gamma(t)\to -1$
e assim 
\[
\arg(-1) = \lim_{t\to \pi} \arg(\gamma(t)) = \lim_{t\to \pi} t+2k\pi = (2k+1)\pi.
\] 
Novamente pela definição de $\gamma$ temos que se $t\to-\pi$ então 
$\gamma(t)\to -1$ e portanto 
\[
\arg(-1) = \lim_{t\to -\pi} \arg(\gamma(t)) = \lim_{t\to -\pi} t+2k\pi = (2k-1)\pi.
\]
O que é uma contradição com a equação anterior.
\end{proof}

\bigskip 


Uma inspeção cuidadosa da prova acima revela, na verdade, que nenhum ramo do argumento 
pode conter em seu domínio nenhum conjunto da forma $D(0,r)\setminus\{0\}$ 
não importa quão pequeno 
seja $r>0$.
Em outras palavras, dentro do domínio $U$ de um ramo qualquer do argumento, não pode ser 
possível ``circular a origem'', isto é, encontrar um círculo (totalmente contido em $U$) 
de forma que a origem esteja dentro da região delimitada por este círculo. 
Na verdade, com um pouco mais de trabalho poderíamos generalizar esta afirmação de
um circulo para uma curva de Jordan suave por partes. Portanto para encontrar ramos
do argumento devemos criar barreiras para que existam tais curvas circundando a origem.
A maneira mais natural de criarmos tais ``barreiras''
seria olhando para a família de todos os conjuntos da 
forma $D(0,r)\setminus\{0\}$, com $r>0$ e para cada raio $r>0$ fixado, 
remover um ponto de $\partial D(0,r)$. Entretanto isto não pode ser feito 
de maneira completamente aleatória. Más escolhas da remoção destes pontos poderiam
não ser efetivas para bloquear a existência de curvas de Jordan suave por partes
circundando a origem. 

Além do mais, é preciso levar em conta que 
a equação \eqref{def-eq-ramo} deve ser válida. 
Por falar na equação \eqref{def-eq-ramo}, observe que se $z\in U$ 
e se $\alpha z\in U$, onde $\alpha>0$ então
\[
\exp(i\arg(\alpha z))= \frac{\alpha z}{|\alpha z|} = \frac{z}{|z|} = \exp(i\arg(z)).
\]

Esta observação sugere que funções contínuas definidas em 
$\partial D(0,1)\setminus\{z_0\}$, onde $z_0=\cos\phi+i\sen\phi$ 
é um ponto qualquer em $\partial D(0,1)$,
satisfazendo \eqref{def-eq-ramo} podem eventualmente ser estendidas a funções contínuas
definidas em toda a região $\mathbb{C}^{*}\setminus L_{\phi}$, onde  
$L_{\phi}$ é o segmento de reta dado por 
$L_{\phi}\equiv \{t(\cos\phi+i\sen \phi)\in \mathbb{C}: t>0\}$.
Entretanto não é muito simples exibir uma tal função em termos da variável 
complexa $z$ e ou suas coordenadas. Esta será nossa próxima tarefa, descrever tal função.
Ela será a função que fornece um ângulo entre $-\pi$ e $\pi$ para qualquer 
número complexo em $\mathbb{C}^{*}\setminus L_{\pi}$. 

Antes de definir analiticamente esta função precisamos 
introduzir algumas notações. Vamos chamar de $\arccos$
a função inversa da função $\cos:(0,\pi)\to(-1,1)$. Vamos denotar por 
$\arcsen$ a inversa da função $\sen:(-\pi/2,\pi/2)\to(-1,1)$ e finalmente
denotamos por $\widetilde{\arccos}$ a inversa da função $\cos:(-\pi,0)\to (-1,1)$.

\begin{teorema}\label{teo-def-ramo-principal-argumento}
A função $\arg:\mathbb{C}\setminus L_{\pi}\to (-\pi,\pi)$ dada por 
\[
\arg(z)
=
\begin{cases}
\arccos\left( \displaystyle\frac{\Re(z)}{|z|} \right),
&\text{se}\ z\in U_1\equiv \{w\in\mathbb{C}: \Im(w)>0\};
\\[0.6cm]
\arcsen\left( \displaystyle\frac{\Im(z)}{|z|} \right),
&\text{se}\ z\in U_2\equiv \{w\in\mathbb{C}: \Re(w)>0\};
\\[0.6cm]
\widetilde{\arccos}\left( \displaystyle\frac{\Re(z)}{|z|} \right),
&\text{se}\ z\in U_3\equiv \{w\in\mathbb{C}: \Im(w)<0\};
\end{cases}
\]
é uma função bem-definida e contínua em $\mathbb{C}\setminus L_{\pi}$ e satisfaz
\[
\frac{z}{|z|} = \exp(i\arg(z)),\quad \forall z\in \mathbb{C}\setminus L_{\pi}.
\]
A função $\arg$ definida acima é chamada de ramo principal do argumento.
\index{Ramo!principal do argumento}
\end{teorema}


\begin{proof}
Primeiro é preciso mostra que a função $\arg$ está bem-definida. Isto é,

\begin{enumerate}[i)]
\item se $z\in U_1\cap U_2$ então 
\[
\arccos\left( \displaystyle\frac{\Re(z)}{|z|} \right)
=
\arcsen\left( \displaystyle\frac{\Im(z)}{|z|} \right);
\]

\item se $z\in U_2\cap U_3$ então 
\[
\arcsen\left( \displaystyle\frac{\Im(z)}{|z|} \right)
=
\widetilde{\arccos}\left( \displaystyle\frac{\Re(z)}{|z|} \right).
\]
\end{enumerate}

Já que a prova de ambos itens são análogas, vamos apresentar apenas a
prova do item i).

Seja $z\in U_1\cap U_2$ e assuma que 

\begin{align}\label{eq-aux2-arg-principal}
\arg(z) 
=
\arccos\left( \displaystyle\frac{\Re(z)}{|z|} \right).
\end{align}
Tomando o cosseno nos dois lados da igualdade acima e depois elevando o resultado ao quadrado 
ficamos com a seguinte identidade
\begin{align*}
\cos(\arg(z))^2
=
\cos^2 \left(  \arccos\left( \displaystyle\frac{\Re(z)}{|z|} \right) \right)
=
\frac{\Re(z)^2}{|z|^2}.
\end{align*}
Usando que $\sen^{2}x+\cos^{2}x=1$, na igualdade acima, obtemos
\begin{align*}
1-\sen(\arg(z))^2 = \frac{\Re(z)^2}{|z|^2}.
\end{align*}
Já que $|z|^2= \Re^2(z)+\Im^2(z)$ segue da igualdade acima que
\begin{align}\label{eq-aux1-arg-principal}
\sen(\arg(z))^2 
= 1 - \frac{\Re(z)^2}{|z|^2} 
= \frac{\Re(z)^2+\Im(z)^2}{|z|^2}-\frac{\Re(z)^2}{|z|^2}
= \frac{\Im(z)^2}{|z|^2}.
\end{align}
Como estamos considerando que $z\in U_1\cap U_2$, temos que $\Re(z)>0$ e $\Im(z)>0$. 
Pelo Lema \ref{lema-re-im-modulo} temos que $0<\Re(z)/|z|<1$.
Lembrando de como definimos a função $\arccos$, temos 
$\arccos([0,1))=(0,\pi/2]$ e daí segue que 
\[
\arg(z) 
=
\arccos\left( \displaystyle\frac{\Re(z)}{|z|} \right)
\in 
(0,\pi/2]
\quad \Longrightarrow\quad 
\sen(\arg(z))\geq 0.
\]
Destas observações e  de \eqref{eq-aux1-arg-principal}
temos
\[
\sen(\arg(z))
=
|\sen(\arg(z))|
=
\sqrt{\sen(\arg(z))^2}
=
\sqrt{
\frac{\Im(z)^2}{|z|^2}
}
=
\frac{|\Im(z)|}{|z|}
=
\frac{\Im(z)}{|z|}.
\]
Tomando agora $\arcsen$ em ambos lados na igualdade acima, ficamos com
\[
\arg(z) = \arcsen\left( \displaystyle\frac{\Im(z)}{|z|} \right)
\]
Desta igualdade e de \eqref{eq-aux2-arg-principal} 
concluímos finalmente a prova do item i).


A continuidade da função $\arg:\mathbb{C}^{*}\setminus L_{\pi}\to \mathbb{R}$
é simples de ser demonstrada. De fato, para qualquer $z_0\in U_1$ temos pela definição  
de $U_1$ que $\Im(z_0)>0$ e consequentemente $-1<\Re(z_0)/|z_0|<1$. Já que  
$\arccos$ é contínua no ponto $\Re(z_0)/|z_0|$ segue que
\begin{align*}
\lim_{z\to z_0}\arg(z) 
= 
\lim_{z\to z_0} \arccos\left( \displaystyle\frac{\Re(z)}{|z|} \right)
&=
\arccos\left( \lim_{z\to z_0}\displaystyle\frac{\Re(z)}{|z|} \right)
\\[0.2cm]
&=
\arccos\left( \displaystyle\frac{\Re(z_0)}{|z_0|} \right)
=
\arg(z_0).
\end{align*}

A continuidade nos pontos de $U_2$ e $U_3$ são provadas de maneira análoga.

Para concluir a prova deste teorema resta apenas mostrar a validade de
\eqref{def-eq-ramo}. Primeiro vamos mostrar que \eqref{def-eq-ramo} é 
válida para todo $z\in U_1\cap U_2$. De fato, neste caso podemos usar a 
identidade do item i) da primeira parte desta prova para verficar que 
\begin{align*}
\exp(i\arg(z))
&=
\cos( \arg(z))+i\sen (\arg(z))
\\[0.2cm]
&
=
\cos\left( \arccos\left( \displaystyle\frac{\Re(z)}{|z|} \right) \right)
+
i\sen \left( \arccos\left( \displaystyle\frac{\Re(z)}{|z|} \right) \right)
\\[0.2cm]
&
=
\cos\left( \arccos\left( \displaystyle\frac{\Re(z)}{|z|} \right) \right)
+
i\sen \left( \arcsen\left( \displaystyle\frac{\Im(z)}{|z|} \right) \right)
\\[0.2cm]
&=
\frac{\Re(z)}{|z|}+i \frac{\Im(z)}{|z|} 
=
\frac{z}{|z|}.
\end{align*}
Analogamente provamos a validade de \eqref{def-eq-ramo}
para os demais pontos de $\mathbb{C}^{*}\setminus L_{\pi}$. 
Com esta observação finalmente encerramos a prova do teorema.
\end{proof}

\bigskip 

Na prática, não usamos as fórmulas dada pelo teorema anterior. Pois o 
valor do ramo principal do argumento de um número complexo $z$ em 
$\mathbb{C}^{*}\setminus L_{\pi}$ é simplesmente o único ângulo entre $-\pi$
e $\pi$ formado vetor determinado por $z$ e pelo eixo real.
A fórmula obtida no teorema é importante para dar uma descrição analítica
precisa deste ângulo e também mais adequada, do que esta descrição 
heurística, para ser usada na prova de outros resultados. 




\bigskip 

Antes de prosseguir vamos apresentar algumas 
propriedades de ramos do argumento genéricos.
O primeiro destes resultados caracteriza todos os possíveis 
ramos do argumento definidos num mesmo domínio.

\begin{proposicao}\label{prop-arg1-arg2-2kpi}
Seja $U\subset\mathbb{C}^{*}$ um domínio. Se $\arg_1:U\to\mathbb{R}$ e 
$\arg_2:U\to\mathbb{R}$ são ramos do argumento em $U$, 
então existe uma constante $k\in\mathbb{Z}$
tal que $\arg_1(z)=\arg_2(z)+2k\pi$, para todo $z\in U$.
\end{proposicao}

\begin{proof}
Já que $\arg_1(z)$ e $\arg_2(z)$ são ramos do argumento em $U$ temos que
\[
\exp(i\arg_1(z))=\frac{z}{|z|}=\exp(i\arg_2(z)).
\]
Portanto $\cos(\arg_1(z))=\cos(\arg_2(z))$ e 
$\sen(\arg_1(z))=\sen(\arg_2(z))$. Segue das propriedades básicas das funções 
trigonométricas que existe um inteiro $k(z)$ tal que  
$\arg_1(z)=\arg_2(z)+2k(z)\pi$. Desta forma a aplicação
\[
z\longmapsto k(z)\equiv \frac{1}{2\pi}(\arg_1(z)-\arg_2(z)) 
\]
define uma função contínua. Já que $U$ é conexo e $k(z)\in\mathbb{Z}$,
segue da continuidade da função $z\longmapsto k(z)$ que ela é identicamente constante, isto é, existe 
$k\in\mathbb{Z}$ tal que $k(z)\equiv k$ para todo $z\in U$.
\end{proof}


\bigskip

Devemos observar que, em geral, é mais difícil estabelecer relações 
entre distintos ramos do argumento definidos em domínios distintos.
Assim, se $\arg_1:U\to\mathbb{R}$ e $\arg_2:V\to\mathbb{R}$ 
são ramos do argumento com $U\neq V$, não temos relações muito claras
entre este ramos. Não é possível em alguns casos dizer nem o que acontece 
em $U\cap V$ já que este conjunto poderia ser vazio ou ter uma infinidade
de componentes conexas. 


Além do ramo principal do argumento, vamos dar destaque nestas notas
a outros ramos definidos em domínios maximais de $\mathbb{C}^{*}$.
Para definir estes outros ramos primeiro fixamos um ângulo $\phi\in \mathbb{R}$.
Em seguida, consideramos o segmento de reta 
$L_{\phi}\equiv \{t(\cos\phi+i\sen \phi)\in \mathbb{C}: t>0\}$ e o 
aberto conexo $\mathbb{C}^{*}\setminus L_{\phi}$.
Definimos o ramo do argumento $\arg_{\phi}:\mathbb{C}^{*}\setminus L_{\phi}\to\mathbb{R}$
por $\arg_{\phi}(z)= \theta(z)$, onde $\theta(z)$ é o único ângulo satisfazendo 
\[
\frac{z}{|z|}=\exp(i\theta(z))\quad\text{e}\quad -2\pi+\phi<\theta(z)<\phi.
\]  
Infelizmente esta definição é bastante inadequada para provarmos que a função 
$z\longmapsto \arg_{\phi}(z)$ é contínua, como fizemos para o ramo principal.
Por outro lado, ela torna a discussão mais geométrica e intuitiva e com um pouco 
de esforço o leitor interessado pode adaptar a expressão do ramo principal do 
argumento, e obter uma descrição explícita para $\arg_{\phi}(z)$ em 
termos da variável complexa $z$; e eventualmente verificar que a mesma está bem
definida e é contínua em $\mathbb{C}^{*}\setminus L_{\phi}$. 


\bigskip 



\begin{definicao}[Ramo do Logaritmo]
\label{def-ramo-logaritmo}
\index{Ramo!do logaritmo}
Seja $U\subset \mathbb{C}^{*}\equiv \mathbb{C}\setminus\{0\}$ um conjunto aberto conexo e $\widetilde{\arg}:U\to\mathbb{R}$ um ramo do argumento definido em $U$. Então a função $f:U\subset \mathbb{C}^{*}\to\mathbb{C}$ dada por 
\begin{align}\label{def-eq-ramo-log}
f(z) = \ln |z|+ i\,\widetilde{\arg}(z), \quad \forall z\in U,
\end{align}
é chamada de um ramo do logaritmo em $U$. No caso especial em que  
$U=\mathbb{C}^{*}\setminus L_{\phi}$ 
e a função $\widetilde{arg}$ é tomada como sendo a função $\arg_{\phi}$, definida acima, então este ramo do logaritmo será denotado 
por $\log_{\phi}$. 
\end{definicao}



\begin{definicao}[Ramo Principal do Logaritmo]
\label{def-ramo-principal-log}
O ramo principal do logaritmo 
\index{Ramo!principal do logaritmo}
é definido como sendo a função 
$\log:\mathbb{C}^{*}\setminus L_{\pi}\to\mathbb{C}$
dada por 
\[
\log(z) = \ln |z| +i \arg(z),
\]
onde $arg: \mathbb{C}^{*}\setminus L_{\pi}\to \mathbb{R}$ 
é o ramo principal do argumento como definido no enunciado do Teorema
 \ref{teo-def-ramo-principal-argumento}. 
\end{definicao}


Observamos que a o ramo principal do logaritmo coincide com a função 
$\log_{\pi}$, apresentada na definição anterior. 
Isto é, para todo $z\in \mathbb{C}^{*}\setminus L_{\pi}$ 
temos $\log(z) = \log_{\pi}(z)$.


\begin{proposicao}\label{prop-explogz}
A função 
$\log_{\phi}:\mathbb{C}^{*}\setminus L_{\phi} 
\to \{z\in\mathbb{C}: -2\pi+\phi<\Im(z)<\phi \}$
define uma aplicação bijetiva cuja inversa é a função exponencial restrita à faixa 
$\{z\in\mathbb{C}: -2\pi+\phi<\Im(z)<\phi \}$. Isto é,
\[
\exp(\log_{\phi}(z))=z,  \qquad  \forall z\in \mathbb{C}^{*}\setminus L_{\phi}
\]
e 
\[
\log_{\phi}(\exp(z))=z,\qquad  \forall z\in \{z\in\mathbb{C}: -2\pi+\phi<\Im(z)<\phi \}.
\]
\end{proposicao}


\begin{proof}
Vamos mostrar inicialmente que $\log_{\phi}$ é uma inversa à direita,
em $\mathbb{C}^{*}\setminus L_{\phi}$, da função exponencial. Isto é,
$\exp(\log_{\phi}(z))=z$, para todo $z\in \mathbb{C}^{*}\setminus L_{\phi}$.
De fato, para qualquer que seja $z\in \mathbb{C}^{*}\setminus L_{\phi}$ temos
\[
\exp(\log_{\phi}(z))
=
\exp(\ln|z|+i\,\arg_{\phi}(z))
=
e^{\ln |z|}\cdot \exp(i\,\arg_{\phi}(z))
\\
=
|z|\cdot \frac{z}{|z|} = z.
\]
Por outro lado, para cada $z\in \{z\in\mathbb{C}: -2\pi+\phi<\Im(z)<\phi \}$
temos 
\begin{align*}
\log_{\phi}(\exp(z)) 
&= 
\ln|\exp(z)|+ i\arg_{\phi}(\exp(z))
\\
&=
\ln(\exp(\Re(z)))+ i\arg_{\phi}\Big(\exp(\Re(z))\exp(i\,\Im(z))\Big)
\\
&=
\Re(z)+ i\arg_{\phi}(\exp(i\,\Im(z)))
\\
&=
\Re(z)+ i\,\Im(z) = z.
\end{align*}
\end{proof}



\begin{lema}[Diferenciabilidade dos ramos do logaritmo]
Para qualquer $\phi\in\mathbb{R}$ fixado a função 
$\log_{\phi}:\mathbb{C}^{*}\setminus L_{\phi} 
\to \{z\in\mathbb{C}: -2\pi+\phi<\Im(z)<\phi \}$
define uma função holomorfa em $\mathbb{C}^{*}\setminus L_{\phi}$ e além do mais, 
sua derivada é dada por
\[
\frac{d}{dz}\log_{\phi}(z)= \frac{1}{z}, 
\qquad \forall z\in \mathbb{C}^{*}\setminus L_{\phi}.
\]
\end{lema}

\begin{proof}
Pela Proposição \ref{prop-explogz} sabemos que 
para quaisquer $z,w\in \mathbb{C}^{*}\setminus L_{\phi}$ temos
\begin{align}\label{eq-aux1-dif-logaritmo}
\frac{\log_{\phi}(z)-\log_{\phi}(w)}{z-w}
=
\frac{\log_{\phi}(z)-\log_{\phi}(w)}{\exp(\log_{\phi}(z)) -\exp(\log_{\phi}(w))}.
\end{align}
Para mostrar que $\log_{\phi}$ tem derivada complexa no ponto $z$ basta 
mostrar que a expressão acima tem limite quando $w\to z$.
Para ver que este limite existe, vamos considerar a seguinte mudança de variáveis:
$s=\log_{\phi}(z)$ e $t=\log_{\phi}(w)$. 
Já que $\log_{\phi}$ é uma função contínua em $\mathbb{C}^{*}\setminus L_{\phi}$
temos que se $w\to z$, então $t\to s$. Portanto segue da diferenciabilidade
da função exponencial complexa e da Proposição \ref{prop-explogz} 
que o seguinte limite existe
\begin{align*}
\frac{d}{dz}\log_{\phi}(z)
&=\lim_{w\to z}
\frac{\log_{\phi}(z)-\log_{\phi}(w)}{z-w}
\\[0.3cm]
&=
\lim_{w\to z}
\frac{\log_{\phi}(z)-\log_{\phi}(w)}{\exp(\log_{\phi}(z)) -\exp(\log_{\phi}(w))}
\\[0.3cm]
&=
\lim_{t\to s}
\frac{s-t}{\exp(s) -\exp(t)}
\\[0.2cm]
&=
\frac{1}{\exp(s)}
=
\frac{1}{\exp(\log_{\phi}(z))}
=
\frac{1}{z}.
\end{align*}

\end{proof}













Dados $\alpha,\beta\in \mathbb{R}$ é possível estabelecer uma relação simples entre
$\arg_{\alpha}(z)$ e $\arg_{\beta}(z)$, para qualquer 
$z\in (\mathbb{C}^{*}\setminus L_{\alpha})\cap(\mathbb{C}^{*}\setminus L_{\beta})$.

Como ilustrado na figura abaixo, podemos ter para alguns pontos do plano complexo
$w\in (\mathbb{C}^{*}\setminus L_{\alpha})\cap(\mathbb{C}^{*}\setminus L_{\beta})$
a igualdade $\arg_{\alpha}(w) = \arg_{\beta}(w)$.
Basta que estes valores do argumento estejam na interseção dos intervalos abertos 
$(-2\pi+\alpha,\alpha)$ e $(-2\pi+\beta,\beta)$ temos que 
\begin{figure}[h]
\centering
\includegraphics[width=0.60\linewidth]{"Figuras/ramos-argumento1"}
\caption{Relação entre valores de ramos distintos do argumento. A região hachurada em
verde é onde temos $\arg_{\alpha}(w)=\arg_{\beta}(w)$. Nos demais pontos deste intervalos
os valores deste ramos do argumento diferem de exatamente $2\pi$.}
\label{fig:ramos-argumento1}
\end{figure}

Se temos $\alpha<\beta$ como na Figura \ref{fig:ramos-argumento1} 
($|\beta-\alpha|<2\pi$) e $z$ é um ponto em 
$(\mathbb{C}^{*}\setminus L_{\alpha})\cap(\mathbb{C}^{*}\setminus L_{\beta})$ 
tal que $\arg_{\beta}(z)>\alpha$, então $\arg_{\alpha}(z)=\arg_{\beta}(z)-2\pi$.


Na verdade, dados $\alpha,\beta\in\mathbb{R}$ quaisquer, podemos mostrar que sempre existe $k\in\mathbb{Z}$ tal que para todo 
$z\in (\mathbb{C}^{*}\setminus L_{\alpha})\cap(\mathbb{C}^{*}\setminus L_{\beta})$  
temos 
\begin{align}\label{eq-arg-alpha-arg-beta}
\arg_{\alpha}(z)=\arg_{\beta}(z)+2k\pi
\end{align}

\begin{figure}[h]
\centering
\includegraphics[width=1\linewidth]{"Figuras/ramos-argumento2"}
\caption{Neste exemplo podemos ver como calcular o valor da constante que aparece na equação 
\ref{eq-arg-alpha-arg-beta}. Aqui temos $\arg_{\alpha}(z)=\arg_{\beta}(z)+2k\pi$, onde $k=-4$.}
\label{fig:ramos-argumento2}
\end{figure}

Dado $x\in\mathbb{R}$ vamos denotar o menor inteiro maior que $x$
por $\lceil x\rceil$. Às vezes, nos referimos a $\lceil x\rceil$
como o ``teto de $x$''. O teto de $x$ também pode ser caracterizado por
$\lceil x\rceil = \sup\{k\in \mathbb{Z}: x\leq k \}$. Analogamente,
definimos o ``piso de $x$'' como sendo o maior inteiro menor que $x$,
notação $\lfloor x\rfloor$. 
Alternativamente, podemos caracterizar o piso de $x$ pela expressão 
$\lfloor x\rfloor = \inf\{k\in\mathbb{Z}: k\leq x\}$.


\begin{proposicao}\label{prop-arg_beta-arg_alpha}
Dados $\alpha,\beta\in\mathbb{R}$ e
$z\in (\mathbb{C}^{*}\setminus L_{\alpha})\cap(\mathbb{C}^{*}\setminus L_{\beta})$  
temos que 
\[
\arg_{\alpha}(z)=\arg_{\beta}(z)+2k\pi,\qquad  \text{onde}\ 
k = 
\left\lfloor \frac{\alpha-\arg_{\beta}(z)}{2\pi} \right\rfloor
=
\left\lceil \frac{\arg_{\alpha}(z)-\beta}{2\pi} \right\rceil. 
\]
\end{proposicao}

\begin{proof}
Já que 
$z\in (\mathbb{C}^{*}\setminus L_{\alpha})\cap(\mathbb{C}^{*}\setminus L_{\beta})$
segue da definição de ramo do argumento que
\begin{align*}
\cos(\arg_{\alpha}(z))+i \sen(\arg_{\alpha}(z))
&=
\exp(i\,\arg_{\alpha}(z))
\\ 
&= 
\frac{z}{|z|} 
\\
&= 
\exp(i\,\arg_{\beta}(z)) 
=
\cos(\arg_{\beta}(z))+i \sen(\arg_{\beta}(z)).
\end{align*}
Portanto 
\[
\begin{cases}
\cos(\arg_{\alpha}(z))=\cos(\arg_{\beta}(z));
\\[0.2cm]
\sen(\arg_{\alpha}(z))=\sen(\arg_{\beta}(z)).
\end{cases}
\]
Logo, existe $k\in\mathbb{Z}$ tal que 
$\arg_{\alpha}(z)= \arg_{\beta}(z)+2k\pi$.
Desta forma temos que
\begin{align*}
k
&=
\frac{\arg_{\alpha}(z)-\arg_{\beta}(z)}{2\pi}
\\
&=
\frac{\arg_{\alpha}(z)-\alpha+\alpha-\arg_{\beta}(z)}{2\pi}
\\
&=
\frac{\arg_{\alpha}(z)-\alpha}{2\pi}
+
\frac{\alpha-\arg_{\beta}(z)}{2\pi}.
\end{align*}
Lembrando que $-2\pi+\alpha<\arg_{\alpha}(z)<\alpha$,
podemos concluir que 
\[
-1<\frac{\arg_{\alpha}(z)-\alpha}{2\pi}<0.
\]
Deste fato e da igualdade anterior, segue que
\[
k< \frac{\alpha-\arg_{\beta}(z)}{2\pi}<k+1
\quad \Longrightarrow \quad 
\left\lfloor \frac{\alpha-\arg_{\beta}(z)}{2\pi} \right\rfloor = k.
\]

Analogamente, temos 
\begin{align*}
k
=
\frac{\arg_{\alpha}(z)-\beta}{2\pi}
+
\frac{\beta-\arg_{\beta}(z)}{2\pi}.
\end{align*}
Como $-2\pi+\beta<\arg_{\beta}(z)<\beta$ segue que 
\[
0<\frac{\beta-\arg_{\beta}(z)}{2\pi}<1.
\]
Portanto
\[
k-1<\frac{\arg_{\alpha}(z)-\beta}{2\pi}<k
\quad \Longrightarrow \quad
k= \left\lceil \frac{\arg_{\alpha}(z)-\beta}{2\pi} \right\rceil.
\]
\end{proof}



\begin{observacao}
A Proposição \ref{prop-arg_beta-arg_alpha} pode ser vista como 
uma generalização da Proposição \ref{prop-arg1-arg2-2kpi} 
já que nesta última os ramos argumento não precisam estar 
definidos em domínios idênticos.
\end{observacao}



\begin{corolario}
Dados $\alpha,\beta\in\mathbb{R}$ e
$z\in (\mathbb{C}^{*}\setminus L_{\alpha})\cap(\mathbb{C}^{*}\setminus L_{\beta})$  
temos que 
\[
\log_{\alpha}(z)=\log_{\beta}(z)+i2k\pi,\qquad  \text{onde}\ 
k = 
\left\lfloor \frac{\alpha-\arg_{\beta}(z)}{2\pi} \right\rfloor
=
\left\lceil \frac{\arg_{\alpha}(z)-\beta}{2\pi} \right\rceil. 
\]
\end{corolario}
\begin{proof}
A prova é uma aplicação direta da definição dos ramos do logaritmo e da Proposição 
\ref{prop-arg_beta-arg_alpha}. De fato, para qualquer 
$z\in (\mathbb{C}^{*}\setminus L_{\alpha})\cap(\mathbb{C}^{*}\setminus L_{\beta})$  
temos
\begin{align*}
\log_{\alpha}(z)
&=
\ln|z|+i\,\arg_{\alpha}(z)
\\[0.3cm]
&=
\ln|z|+i( \arg_{\beta}(z) + 2k\pi)
\\[0.3cm]
&=
[\ln|z|+i\,\arg_{\beta}(z)] + i2k\pi
\\[0.3cm]
&=
\log_{\beta}(z)+ i2k\pi.
\end{align*}
\end{proof}

\section{Potências Arbitrárias}

Já que definimos a exponencial complexa e os ramos do logaritmo, 
podemos agora introduzir o conceito de potencias arbitrárias. 

\begin{definicao}\label{def-potencias-arbitrarias}
Fixe $w\in\mathbb{C}$ e $\phi\in\mathbb{R}$. 
Dado $z\in \mathbb{C}\setminus L_{\phi}$ definimos $z^w$ (ou  $z^{w|_{\phi}}$
quando for necessário especificar o valor de $\phi$) como
sendo o número complexo $f_{w,\phi}(z)$, onde 
$f_{w,\phi}: \mathbb{C}\setminus L_{\phi}\to \mathbb{C}$ 
é a função dada por $f_{w,\phi}(z) = \exp(w\log_{\phi}(z))$.  
\end{definicao}


Para $w\in\mathbb{C}$ fixado e $\phi\in\mathbb{R}$ escolhido, temos que a aplicação
$(\mathbb{C}\setminus L_{\phi}) \ni z\longmapsto z^{w}=\exp(w\log_{\phi}(z))$
define uma aplicação holomorfa em $\mathbb{C}\setminus L_{\phi}$. O ramo 
principal da função $z\longmapsto z^{w}$ é obtido quando escolhemos $\phi=\pi$ 
e consequentemente $\log_{\phi}$ como o ramo principal do logaritmo. 

Pela regra da cadeia (Teorema \ref{teo-regra-da-cadeia}),
Proposição \ref{prop-explogz} e equação \ref{eq-razao-exps} temos que
\begin{align*}
\frac{d}{dz}z^w 
= 
\frac{d}{dz}\exp(w\log_{\phi}(z)) 
&=
\exp(w\log_{\phi}(z))\frac{d}{dz}[w\log_{\phi}(z)]
\\
&=
\exp(w\log_{\phi}(z))\frac{w}{z}
\\
&=
\exp(w\log_{\phi}(z))\frac{w}{\exp(\log_{\phi}(z))}
\\
&=
w\frac{\exp(w\log_{\phi}(z))}{\exp(\log_{\phi}(z))}
\\
&=
w \exp((w-1)\log_{\phi}(z)).
\\
&=
wz^{w-1}.
\end{align*}

O que mostra que, independentemente da escolha do ramo do logaritmo temos sempre 
\[
\frac{d}{dz}z^w = wz^{w-1}.
\]

Outra observação importante é que a função introduzida na 
Definição \ref{def-potencias-arbitrarias} generaliza, em um certo sentido, 
a noção usual de potência já que para qualquer $n\in\mathbb{N}$ e 
$z\in \mathbb{C}\setminus L_{\pi}$ temos: 
\[
z^n  
=  
\exp(n\log(z))
=
\underbrace{\exp(\log(z))\cdot\ldots\cdot\exp(\log(z))}_{n-\mathrm{vezes}}
= z\cdot\ldots\cdot z.
\]

Outra observação importante é que estas funções nos permitem construir raízes 
$n$-ésimas de um número complexo $w$. Para isto basta observar que 
escolhido o ramo principal do logaritmo e
$w\in \mathbb{C}\setminus L_{\pi}$ temos que $w^{\frac{1}{n}}$ é uma
solução da equação $z^n=w$. De fato, 
\[
w^{\frac{1}{n}}
=
\exp\Big(\frac{1}{n}\log(w)\Big)
=
\exp\Big(\frac{1}{n}(\ln|w|+i\arg(w))\Big)
=
|w|^{\frac{1}{n}}\exp\Big(i\frac{\arg(w)}{n}\Big).
\]


Na verdade, podemos construir todas as raízes $n$-ésimas de $w$
considerando todos os possíveis ramos de $w^{\frac{1}{n}}$.
Para simplificar a discussão vamos introduzir a seguinte notação para os
ramos da raíz $n$-ésima. 
\index{Raíz!$n$-ésima de $z$}
Dado $\phi\in\mathbb{R}$ e $w\in \mathbb{C}\setminus L_{\phi}$
vamos denotar $\exp((1/n)\log_{\phi}(w))$ por $\sqrt[n]{w}^{\,\phi}$.
\begin{proposicao}\label{prop-ramos-raizes-raizes-nesimas}
Para qualquer $n\in\mathbb{N}$ e $w\in\mathbb{C}$ fixados temos 
\[
\{ \sqrt[n]{w}^{\,\phi} \in \mathbb{C}: \phi\in\mathbb{R}\ \text{e}\ w\in 
\mathbb{C}\setminus L_{\phi} \}
=
\{ z\in\mathbb{C}: z^n=w\}.
\]
\end{proposicao}

Em seguida, mostramos as principais ideias envolvidas na prova da 
proposição acima para o caso especial em que $n=2$ e $w=-1$. 

\begin{exemplo}
\label{exemplo-raiz-menos-um} 
Vamos mostrar que 
\[
\{ \sqrt{-1}^{\,\phi} \in \mathbb{C}: \phi\in\mathbb{R}\ \text{e}\ (-1)\in 
\mathbb{C}\setminus L_{\phi} \}
=
\{ z\in\mathbb{C}: z^2=-1\}
=
\{-i,i\}.
\]
\end{exemplo}


Inicialmente escolhemos um ramo qualquer do argumento onde temos mais facilidade 
para determinar o valor de $\arg_{\phi}(-1)$. Por exemplo, vamos tomar $\phi=0$.
Neste caso, $\arg_{0}(-1)=-\pi$. Para qualquer $\phi\neq (2k+1)\pi$, com $k\in\mathbb{Z}$,
temos que $(-1)\in (\mathbb{C}\setminus L_{\phi})\cap (\mathbb{C}\setminus L_{0})$.
Desta forma podemos aplicando a Proposição \ref{prop-arg_beta-arg_alpha} e assim concluir que
para quaisquer tais valores de $\phi$ temos 
\[
\arg_{\phi}(-1) = \arg_{0}(-1) + 2\pi
\left\lfloor \frac{\phi-\arg_{0}(-1)}{2\pi}\right\rfloor 
=
-\pi + 2\pi
\left\lfloor \frac{\phi+\pi}{2\pi}\right\rfloor 
\]

Logo, se $\phi\in\mathbb{R}$ não é um múltiplo ímpar
de $\pi$ então temos 
\begin{align*}
\sqrt{-1}^{\,\phi} 
&=
\exp\Big(\frac{1}{2}\ln|-1|+i\frac{\arg_{\phi}(-1)}{2}  \Big)
\\[0.3cm]
&=
\exp\Big(i\frac{\arg_{\phi}(-1)}{2}  \Big)
\\[0.3cm]
&=
\exp\Big(i \big(-\frac{\pi}{2} + \pi
\left\lfloor \frac{\phi+\pi}{2\pi}\right\rfloor \big) \Big).
\end{align*}
Fazendo $\phi$ variar em $\mathbb{R}\setminus\{(2k+1)\pi: k\in\mathbb{Z}\}$
podemos observar que a expressão acima assume os seguintes valores
\[
\left\{\exp\Big(i \big(-\frac{\pi}{2} + k\pi \big)\Big) : k\in\mathbb{Z}\right\} 
=
\{ e^{-i\frac{\pi}{2}}, e^{i\frac{\pi}{2}} \}
=
\{-i,i\}.
\]


\bigskip 

Observamos que para qualquer que seja $w\in \mathbb{C}\setminus L_{\phi}$ sempre temos 
\begin{align*}
\sqrt{w}^{\,\phi} \sqrt{w}^{\,\phi} 
&= 
\left[\exp\Big(\frac{1}{2}\ln|w|+i\frac{1}{2}\arg_{\phi}(w)\Big)\right]^2
\\
&=
\exp\Big(\ln|w|+i\arg_{\phi}(w)\Big)
\\
&=
\exp(\log_{\phi}(w))
\\
&=
w.
\end{align*}

Mas, em geral, podemos ter
\[
\sqrt{zw}^{\,\phi} \neq  \sqrt{z}^{\,\phi} \sqrt{w}^{\,\phi} 
\quad\text{e também}\quad 
\sqrt{z^2}^{\,\phi} \neq  \sqrt{z}^{\,\phi} \sqrt{z}^{\,\phi}.
\]
mesmo que $z, w$, $zw$ e $z^2$ sejam pontos de $\mathbb{C}\setminus L_{\phi}$.