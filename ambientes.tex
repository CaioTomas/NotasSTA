%-----------------
%	THEOREM STYLES
%-----------------
% \usepackage{amsgen}
\definecolor{violetaclaro}{RGB}{10, 10, 150}
\definecolor{marromclaro}{RGB}{150, 60, 60}

% Boxed/framed environments
\newtheoremstyle{bluenumbox}% Theorem style name
	{0pt}% Space above
	{0pt}% Space below
	{\normalfont}% Body font
	{}% Indent amount
	{\small\bf\sffamily\color{violetaclaro}}% Theorem head font
	{\;}% Punctuation after theorem head
	{0.25em}% Space after theorem head
	{\small\sffamily\color{violetaclaro}\thmname{#1}\nobreakspace\thmnumber{#2}\thmnote{\nobreakspace\sffamily\bfseries\color{black}---\nobreakspace#3.}} 
	% Optional theorem note

\newtheoremstyle{blacknumex}% Theorem style name
	{5pt}% Space above
	{5pt}% Space below
	{\normalfont}% Body font
	{} % Indent amount
	{\small\bf\sffamily}% Theorem head font
	{\;}% Punctuation after theorem head
	{0.25em}% Space after theorem head
	{\small\sffamily{\tiny\ensuremath{\blacksquare}}%
		\nobreakspace\thmname{#1}\nobreakspace
		\thmnumber{#2}% Theorem text (e.g. Theorem 2.1)
		\thmnote{\nobreakspace\sffamily\bfseries---\nobreakspace#3.}
	}% Optional theorem note

\newtheoremstyle{blacknumbox} % Theorem style name
	{0pt}% Space above
	{0pt}% Space below
	{\normalfont}% Body font
	{}% Indent amount
	{\small\bf\sffamily}% Theorem head font
	{\;}% Punctuation after theorem head
	{0.25em}% Space after theorem head
	{\small\sffamily\thmname{#1}%
		\nobreakspace\thmnumber{#2}% Theorem text (e.g. Theorem 2.1)
		\thmnote{\nobreakspace\sffamily\bfseries---\nobreakspace#3.}
	}% Optional theorem note

% Non-boxed/non-framed environments
\newtheoremstyle{bluenum}% Theorem style name
	{5pt}% Space above
	{5pt}% Space below
	{\normalfont}% Body font
	{}% Indent amount
	{\small\bf\sffamily\color{violetaclaro}}% Theorem head font
	{\;}% Punctuation after theorem head
	{0.25em}% Space after theorem head
	{\small\sffamily\color{violetaclaro}\thmname{#1}%
		\nobreakspace\thmnumber{#2}% Theorem text (e.g. Theorem 2.1)
		\thmnote{\nobreakspace\sffamily\bfseries\color{black}---\nobreakspace#3.}
	} % Optional theorem note

\newtheoremstyle{brownnumbox}% Theorem style name
	{5pt}% Space above
	{5pt}% Space below
	{\normalfont}% Body font
	{}% Indent amount
	{\small\bf\sffamily\color{marromclaro}}% Theorem head font
	{\;}% Punctuation after theorem head
	{0.25em}% Space after theorem head
	{\small\sffamily\color{marromclaro}\thmname{#1}%
		\nobreakspace\thmnumber{#2}% Theorem text (e.g. Theorem 2.1)
		\thmnote{\nobreakspace\sffamily\bfseries\color{black}---\nobreakspace#3.}
	} % Optional theorem note
\makeatother

% Defines the theorem text style for each type of theorem to one of the three styles above
\newcounter{dummy} 
\numberwithin{dummy}{chapter}

\theoremstyle{bluenumbox}
	\newtheorem{theoremeT}[dummy]{Teorema}
	\newtheorem{exerciseT}{Exercício}[chapter]

\theoremstyle{blacknumex}
	\newtheorem{exampleT}{Exemplo}[chapter]

\theoremstyle{blacknumbox}
	\newtheorem{definitionT}{Definição}[chapter]
	
\theoremstyle{brownnumbox}
	\newtheorem{corollaryT}[dummy]{Corolário}
	\newtheorem{lemmaT}{Lema}[chapter]

\theoremstyle{bluenum}
	\newtheorem{proposicaoT}[dummy]{Proposição}

%------------------------------
%	DEFINITION OF COLORED BOXES
%------------------------------

\RequirePackage[framemethod=default]{mdframed} 
% Required for creating the theorem, definition, exercise and corollary boxes

% Theorem box
\newmdenv[
	skipabove=7pt,
	skipbelow=7pt,
	backgroundcolor=black!5,
	linecolor=violetaclaro,
	innerleftmargin=5pt,
	innerrightmargin=5pt,
	innertopmargin=5pt,
	leftmargin=0cm,
	rightmargin=0cm,
	innerbottommargin=5pt]{tBox}

% Exercise box	  
\newmdenv[
	skipabove=7pt,
	skipbelow=7pt,
	rightline=false,
	leftline=true,
	topline=false,
	bottomline=false,
	backgroundcolor=violetaclaro!10,
	linecolor=violetaclaro,
	innerleftmargin=5pt,
	innerrightmargin=5pt,
	innertopmargin=5pt,
	innerbottommargin=5pt,
	leftmargin=0cm,
	rightmargin=0cm,
	linewidth=4pt]{eBox}	

% Definition box
\newmdenv[
	skipabove=7pt,
	skipbelow=7pt,
	rightline=false,
	leftline=true,
	topline=false,
	bottomline=false,
	linecolor=violetaclaro,
	innerleftmargin=5pt,
	innerrightmargin=5pt,
	innertopmargin=0pt,
	leftmargin=0cm,
	rightmargin=0cm,
	linewidth=4pt,
	innerbottommargin=0pt]{dBox}	

% Corollary/Lemma box
\newmdenv[
	skipabove=7pt,
	skipbelow=7pt,
	rightline=false,
	leftline=true,
	topline=false,
	bottomline=false,
	linecolor=marromclaro,
	backgroundcolor=marromclaro!10,
	innerleftmargin=5pt,
	innerrightmargin=5pt,
	innertopmargin=2pt,
	leftmargin=0cm,
	rightmargin=0cm,
	linewidth=4pt,
	innerbottommargin=8pt]{cBox}

% Remark box
\newmdenv[
	skipabove=7pt,
	skipbelow=7pt,
	rightline=false,
	leftline=false,
	topline=false,
	bottomline=false,
	backgroundcolor=marromclaro!10,
	innerleftmargin=5pt,
	innerrightmargin=5pt,
	innertopmargin=5pt,
	leftmargin=0cm,
	rightmargin=0cm,
	linewidth=4pt,
	innerbottommargin=5pt]{rBox}

% Creates an environment for each type of theorem and assigns it a theorem text style from the "Theorem Styles" section above and a colored box from above
\newenvironment{teorema}
    {\begin{tBox}\begin{theoremeT}}
    {\end{theoremeT}\end{tBox}}
%
\newenvironment{exercicio}
    {\begin{eBox}\begin{exerciseT}}{
	    \hfill{\color{violetaclaro}\tiny\ensuremath{\blacksquare}}
	    \end{exerciseT}\end{eBox}
    }	
%
\newenvironment{definicao}
    {\begin{dBox}\begin{definitionT}}
    {\end{definitionT}\end{dBox}}	
%
\newenvironment{exemplo}
    {\begin{exampleT}}
    {\hfill{\tiny\ensuremath{\blacksquare}}\end{exampleT}}	
%
\newenvironment{corolario}
    {\begin{cBox}\begin{corollaryT}}
    {\end{corollaryT}\end{cBox}}
%
\newenvironment{lema}
    {\begin{cBox}\begin{lemmaT}}
    {\end{lemmaT}\end{cBox}}
%
\newenvironment{observacao}
    {\begin{rBox}{\noindent\small\sffamily\color{marromclaro}\textbf{Observação.}}}
    {\end{rBox}}
%
\newenvironment{proposicao}
    {\begin{proposicaoT}}
    {%
        \hfill{%
            \color{violetaclaro}\tiny\ensuremath{\blacksquare}
        }\end{proposicaoT}
    }
%
\renewenvironment{proof}[1][Demonstração]
    {\paragraph{{#1.\ }}}
    {\hfill\ensuremath{\blacksquare}\\}
%

%%%%%%%%%%%%%%%%%%%%
% As linhas abaixo tiram os bad breaks dos warnings. Comente-as se quiser saber se/onde os bad breaks ocorrem
\usepackage{silence}

\WarningFilter{mdframed}{You got a bad break}

\makeatletter

\mdf@PackageWarning{You got a bad break\MessageBreak
  because the last split box is empty\MessageBreak
  You have to change the settings}

\makeatother
%