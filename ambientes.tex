%----------------------------------------------------------------------------------------
%	THEOREM STYLES
%----------------------------------------------------------------------------------------

\usepackage{amsmath, amsfonts, amssymb, amsthm, mathrsfs, nccmath} 
% For math equations, theorems, symbols, etc

\newcommand{\intoo}[2]{\mathopen{]}#1\,;#2\mathclose{[}}
\newcommand{\ud}{\mathop{\mathrm{{}d}}\mathopen{}}
\newcommand{\intff}[2]{\mathopen{[}#1\,;#2\mathclose{]}}
\renewcommand{\qedsymbol}{$\blacksquare$}
\newtheorem{notation}{Notation}[chapter]

% Boxed/framed environments
\newtheoremstyle{bluenumbox}% Theorem style name
	{0pt}% Space above
	{0pt}% Space below
	{\normalfont}% Body font
	{}% Indent amount
	{\small\bf\sffamily\color{marinho}}% Theorem head font
	{\;}% Punctuation after theorem head
	{0.25em}% Space after theorem head
	{\small\sffamily\color{marinho}\thmname{#1}\nobreakspace\thmnumber{\@ifnotempty{#1}{}\@upn{#2}}
	% Theorem text (e.g. Theorem 2.1)
	\thmnote{\nobreakspace\the\thm@notefont\sffamily\bfseries\color{black}---\nobreakspace#3.}} 
	% Optional theorem note

\newtheoremstyle{blacknumex}% Theorem style name
	{5pt}% Space above
	{5pt}% Space below
	{\normalfont}% Body font
	{} % Indent amount
	{\small\bf\sffamily}% Theorem head font
	{\;}% Punctuation after theorem head
	{0.25em}% Space after theorem head
	{\small\sffamily{\tiny\ensuremath{\blacksquare}}%
		\nobreakspace\thmname{#1}\nobreakspace
		\thmnumber{\@ifnotempty{#1}{}\@upn{#2}}% Theorem text (e.g. Theorem 2.1)
		\thmnote{\nobreakspace\the\thm@notefont\sffamily\bfseries---\nobreakspace#3.}
	}% Optional theorem note

\newtheoremstyle{blacknumbox} % Theorem style name
	{0pt}% Space above
	{0pt}% Space below
	{\normalfont}% Body font
	{}% Indent amount
	{\small\bf\sffamily}% Theorem head font
	{\;}% Punctuation after theorem head
	{0.25em}% Space after theorem head
	{\small\sffamily\thmname{#1}%
		\nobreakspace\thmnumber{\@ifnotempty{#1}{}\@upn{#2}}% Theorem text (e.g. Theorem 2.1)
		\thmnote{\nobreakspace\the\thm@notefont\sffamily\bfseries---\nobreakspace#3.}
	}% Optional theorem note

% Non-boxed/non-framed environments
\newtheoremstyle{bluenum}% Theorem style name
	{5pt}% Space above
	{5pt}% Space below
	{\normalfont}% Body font
	{}% Indent amount
	{\small\bf\sffamily\color{marinho}}% Theorem head font
	{\;}% Punctuation after theorem head
	{0.25em}% Space after theorem head
	{\small\sffamily\color{marinho}\thmname{#1}%
		\nobreakspace\thmnumber{\@ifnotempty{#1}{}\@upn{#2}}% Theorem text (e.g. Theorem 2.1)
		\thmnote{\nobreakspace\the\thm@notefont\sffamily\bfseries\color{black}---\nobreakspace#3.}
	} % Optional theorem note
\makeatother

% Defines the theorem text style for each type of theorem to one of the three styles above
\newcounter{dummy} 
\numberwithin{dummy}{chapter}

\theoremstyle{bluenumbox}
	\newtheorem{theoremeT}[dummy]{Teorema}
	\newtheorem{problem}{Problema}[chapter]
	\newtheorem{exerciseT}{Exercício}[chapter]

\theoremstyle{blacknumex}
	\newtheorem{exampleT}{Exemplo}[section]

\theoremstyle{blacknumbox}
	\newtheorem{vocabulary}{Vocabulário}[chapter]
	\newtheorem{definitionT}{Definição}[chapter]
	\newtheorem{corollaryT}[dummy]{Corolário}
	\newtheorem{lemmaT}{Lema}[chapter]

\theoremstyle{bluenum}
	\newtheorem{proposition}[section]{Proposição}

%----------------------------------------------------------------------------------------
%	DEFINITION OF COLORED BOXES
%----------------------------------------------------------------------------------------

\RequirePackage[framemethod=default]{mdframed} 
% Required for creating the theorem, definition, exercise and corollary boxes

% Theorem box
\newmdenv[
	skipabove=7pt,
	skipbelow=7pt,
	backgroundcolor=black!5,
	linecolor=marinho,
	innerleftmargin=5pt,
	innerrightmargin=5pt,
	innertopmargin=5pt,
	leftmargin=0cm,
	rightmargin=0cm,
	innerbottommargin=5pt]{tBox}

% Exercise box	  
\newmdenv[
	skipabove=7pt,
	skipbelow=7pt,
	rightline=false,
	leftline=true,
	topline=false,
	bottomline=false,
	backgroundcolor=marinho!10,
	linecolor=marinho,
	innerleftmargin=5pt,
	innerrightmargin=5pt,
	innertopmargin=5pt,
	innerbottommargin=5pt,
	leftmargin=0cm,
	rightmargin=0cm,
	linewidth=4pt]{eBox}	

% Definition box
\newmdenv[
	skipabove=7pt,
	skipbelow=7pt,
	rightline=false,
	leftline=true,
	topline=false,
	bottomline=false,
	linecolor=marinho,
	innerleftmargin=5pt,
	innerrightmargin=5pt,
	innertopmargin=0pt,
	leftmargin=0cm,
	rightmargin=0cm,
	linewidth=4pt,
	innerbottommargin=0pt]{dBox}	

% Corollary box
\newmdenv[
	skipabove=7pt,
	skipbelow=7pt,
	rightline=false,
	leftline=true,
	topline=false,
	bottomline=false,
	linecolor=gray,
	backgroundcolor=black!5,
	innerleftmargin=5pt,
	innerrightmargin=5pt,
	innertopmargin=5pt,
	leftmargin=0cm,
	rightmargin=0cm,
	linewidth=4pt,
	innerbottommargin=5pt]{cBox}

% Creates an environment for each type of theorem and assigns it a theorem text style from the "Theorem Styles" section above and a colored box from above
\newenvironment{theorem}{\begin{tBox}\begin{theoremeT}}{\end{theoremeT}\end{tBox}}
\newenvironment{exercise}{\begin{eBox}\begin{exerciseT}}{
	\hfill{\color{marinho}\tiny\ensuremath{\blacksquare}}
	\end{exerciseT}\end{eBox}
}				  
\newenvironment{definition}{\begin{dBox}\begin{definitionT}}{\end{definitionT}\end{dBox}}	
\newenvironment{example}{\begin{exampleT}}{\hfill{\tiny\ensuremath{\blacksquare}}\end{exampleT}}		
\newenvironment{corollary}{\begin{cBox}\begin{corollaryT}}{\end{corollaryT}\end{cBox}}
\newenvironment{lemma}{\begin{cBox}\begin{lemmaT}}{\end{lemmaT}\end{cBox}}

\newenvironment{remark}{{\noindent\small\sffamily\textbf{Observação.}}}{\hfill{\tiny\ensuremath{\blacksquare}}} 
\newenvironment{solution}{{\noindent\small\sffamily\textbf{Solução.}}}{} 