\documentclass[12pt,a4paper]{article}
\usepackage[utf8]{inputenc}
\usepackage{amssymb,amsthm,amsmath}
\usepackage{graphicx}
\usepackage{verbatim}
\usepackage{booktabs}
\usepackage{xfrac}
\usepackage[pdfstartview=FitH,backref,colorlinks,bookmarksnumbered,bookmarksopen,linktocpage,urlcolor=blue,linkcolor=cyan]{hyperref}
\usepackage{pdfpages}
\usepackage{enumitem}
\usepackage{pifont}

\newcommand{\cmark}{\ding{51}}%
\newcommand{\xmark}{\ding{55}}%
\newlist{todolist}{itemize}{2}
\setlist[todolist]{label=$\square$}

\newcommand{\done}{\rlap{$\square$}{\raisebox{2pt}{\large\hspace{1pt}\cmark}}%
\hspace{-2.5pt}}
\newcommand{\wontfix}{\rlap{$\square$}{\large\hspace{1pt}\xmark}}

\title{A fazer}
\author{}
\date{\today}

\begin{document}

\maketitle

\begin{todolist}
    % pra marcar o item como feito é só colocar \item[\done]
    \item dar a motivação no início da seção 5.2 (continuação analítica ao longo
    de caminhos);
    \item falar do teorema da uniformização de Riemann (final da seção 5.2); 
    \item colocar diagramas;
    \item demonstrar o teorema de Monodromia;

\end{todolist}

\end{document}