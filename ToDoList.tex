\documentclass[12pt,a4paper]{article}
\usepackage[utf8]{inputenc}
\usepackage{amssymb,amsthm,amsmath}
\usepackage{graphicx}
\usepackage{verbatim}
\usepackage{booktabs}
\usepackage{xfrac}
\usepackage[pdfstartview=FitH,backref,colorlinks,bookmarksnumbered,bookmarksopen,linktocpage,urlcolor=blue,linkcolor=cyan]{hyperref}
\usepackage{pdfpages}
\usepackage{enumitem}
\usepackage{pifont}

\newcommand{\cmark}{\ding{51}}%
\newcommand{\xmark}{\ding{55}}%
\newlist{todolist}{itemize}{2}
\setlist[todolist]{label=$\square$}

\newcommand{\done}{\rlap{$\square$}{\raisebox{2pt}{\large\hspace{1pt}\cmark}}%
\hspace{-2.5pt}}
\newcommand{\wontfix}{\rlap{$\square$}{\large\hspace{1pt}\xmark}}

\title{A fazer}
\author{}
\date{\today}

\begin{document}

\maketitle

\begin{todolist}
    % pra marcar o item como feito é só colocar \item[\done]
    \item dar a motivação no início da seção 5.3 (continuação analítica ao longo
    de caminhos);
    \item falar do teorema da uniformização de Riemann (final da seção 5.3); 
    \item colocar/fazer diagramas;
    \item[\done] demonstrar o teorema de Monodromia;
    \item[\done] escrever a observação sobre funções com conjunto de singularidades denso;
    \item[\done] falar da continuação do logaritmo;
    \item[\done] incrementar a seção do Princípio da Reflexão de Schwarz;
    \item[\done] escrever sobre o Teorema da Aproximação de Runge;
    \item[\done] escrever sobre integração complexa e o início de transformada de Fourier;
    \item[\done] demonstrar a proposição e o corolário na subseção de existência da transformada de Fourier;
    \item[\done] demonstrar o lema da integral da exponencial (auxiliar para a inversão da transformada);
    \item demonstrar o teorema da inversão da transformada;
    \item demonstrar a fórmula da soma de Poisson;
    \item pensar num lema que junte as demonstrações das boas definições da transf. de Fourier,
          da integral da exponencial no lema auxiliar (caso especial da transf. de Laplace) e
          da integral do shift;

\end{todolist}

\end{document}

% Vou colocar esse aviso aqui pq não achei lugar melhor. No arquivo .bib, deixar de colocar 
% "language = {brazilian}"
% nas referências citadas gerará warning(s). 
% --- Thiago